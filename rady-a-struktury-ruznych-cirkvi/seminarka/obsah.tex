\chapter{Úvod}
\label{kap:uvod}

\section{Prameny}

Tuto práci píšu z~pohledu hrdého a oddaného člena Církve československé
husitské, a adepta na kněžství v~ní. Pokusil jsem se nastudovat potřebné
materiály svojí církve tak, abych se mohl skromně domnívat, že jejím stanovám
alespoň povrchně a v~rámci možností své pramalé vyzrálosti v~oboru, jakož i
svého omezeného intelektu rozumím. Naopak komplexní porozumění stanovám jiných
církví, včetně té Římskokatolické, není mojí ambicí, a proto jsem do jejích
materiálů nahlížel jen bodově za konkrétním účelem vydobytí faktů nezbytných pro
sepsání tohoto krátkého pojednání.

Čerpal jsem tedy ze Základů víry Církve československé husitské, z její ústavy,
řádů a agendy. Pro fakta o Římskokatolické církvi jsem nabral odrazový můstek
v~knize Církevní právo od autorů Tretera, Horák vydané roku Páně dvoutisícího
sedmnáctého, a dále jsem se opíral o kanonické právo, dostupné na stránkách
katolické církve.

\section{Kněžství}

Kněžství je vágně definovaný pojem namnoze závislý na úhlu pohledu. Kněží se
vyskytovali od nejranějších historických dob podnes a vyskytují se
v~nejrozličnějších směrech, tedy ve velké šíři jak na časové ose tak na ose
lokálně-kulturní. A na obou z~nich se pojetí značně liší. Pokusu o uchopení
pojmu se vyhnu, protože moje ambice není v~pojednání o kněžství jako takovém,
nýbrž ve srovnání jeho konkrétních dvou realizací, a to současného
římskokatolického a Církve československé husitské.

Zaměřme se nejdříve na společnou základnu. Církev československá husitská se
odloupla od církve Římskokatolické před sto jedním rokem, zdědivše od ní
v~podstatě všechno, co ji definovalo, včetně pojmu kněžství. Zakotvené je
biblicky hlavně v~pastorálních epištolách, které vyjmenovávají potřebné
vlastnosti knězovy, a můžeme si jako členové reformační církve s~libostí
povšimnout, že například v~otázce celibátu stojí biblické instrukce zcela jasně
na ,,naší`` straně. Je až úsměvné, že právě tento bod, který tak okatě odporuje
katolickému kanonickému právu, je výplodem té části bible, kterou má na svědomí
raný katolicismus. Pravdou však zůstává, že vývoj církve včetně kněžství a
kněžstva, byl společný oběma našim zkoumaným subjektům ještě dlouho předlouho po
dobách vzniku pastorálních epištol. To, že se Církev československá husitská
rozhodla (a velmi dobře!) po kněžích celibát nevyžadovat, nelze tedy vnímat
odloučené od faktu, že mnoho jiných nebiblických praktik od římských katolíků
převzala. A i tady si dovolím toto hodnotit jako dobré, protože z~oné
dvoutisícileté tradice by se měly zavrhnout jen špatné věci, ne všechny jen
proto, že nejsou v~bibli, která při vší zbožné úctě nemůže být jediným kanálem
Ducha svatého v~lidském životě.

Zpět však k~naší otázce společného základu: Kněží v~obou
církvích mají jiný status nežli ostatní členové církve. Jsou osoby pověřené a
výlučně oprávněné ke konání svátostných úkonů a jsou ke své církvi ve služebním
poměru. Skládají slib věrnosti, podepisují smlouvu, dostávají mzdu. Dostávají
liturgický oděv, který smějí nosit při svátostných
obřadech\footnote{Bohoslužebný řád CČSH část A bod 5; CIC 929}.

\section{Apoštolská posloupnost a obecné kněžství}
V~samotném pojetí najdeme i zásadní rozdíly. Především jde o to, že kněžství
v~Římskokatolické církvi se opírá o apoštolskou posloupnost\footnote{viz
Církevní právo s. 36}, zatímco u Církve
československé husitské o obecné kněžství všech křesťanů spojené s~pověřením
církve k~profesionální službě\footnote{Základy víry CČSH 342-347; 3. část agendy
s. 23}.

Apoštolská posloupnost je fascinující fenomén --
jeden z~těch, u kterých není ani zdravě pochybujícímu skeptikovi zatěžko
přijmout její historickou nepřerušenost. Vždyť už římský biskup Klement v~prvním
století se ve svém prvním listu korintským o důsledné péči o apoštolskou
posloupnost zmiňuje\footnote{1Clem 44}, a Tertullian o Klementovi píše, že byl
vysvěcen samotným Petrem\footnote{De praescriptione haereticorum 32,2}. Pro
ochranu a pojištění apoštolské posloupnosti se později začalo dbát na to, aby
každého biskupa světilo několik biskupů najednou.

Na druhou stranu domnívat se, že by Duch svatý sestupoval na lidi podle toho,
jestli na ně vložil ruce někdo, na koho před tím někdo vložil ruce... řekněme,
že to z~Boha svrchovaného a nekonečně moudrého dělá štafetový kolík. Vycházet
z~obecného kněžství je tedy v~duchu reformačním jednoznačně více na straně pravé
zbožnosti.

Nebyl bych však spravedlivý, kdybych nepodotkl, že opora Církve československé
husitské v~kněžství obecném namísto v~apoštolské posloupnosti je v~pravém slova
smyslu znouzectnost, neboť získání legitimní apoštolské posloupnosti
byla v~začátcích církve palčivou, leč marnou touhou.

Nemohu říci, že bych to nechápal. Jakkoliv je absurdní domnívat se, že by
vkládání rukou mohlo ovládat Ducha svatého, nelze ani vyloučit, že rituál
skutečně nějakým způsobem obcování s~Duchem svatým zprostředkuje. Jako nelze
Duchu svatému přikázat, aby se vložením rukou vlil, ani mu to nelze zakázat.
Nemohu v~této souvislosti nepomyslet na anekdotu s~Richardem Feynmanem, jenž jsa
navštíven novinářem, byl dotázán, proč že má nade dveřmi zavěšenou podkovu.
Řekl, že pro štěstí. Novinář se zeptal, zdali na to věří. Feynman odpověděl:
,,Nevěřím. Ale to neznamená, že to nefunguje.``

Přesto vidím v~absenci formální apoštolské posloupnosti pro naši církev
požehnání: Kdo se nemůže spolehnout na formální prostředky, musí se obrátit
k~živému Duchu svatému a dosvědčovat jeho působení svým životem.

\chapter{Svěcení}

Svěcení je v~obou církvích aktem uvedení osoby do stavu kněžství. Výsledek je
doživotní a nezvratný status svěcence, ovšem faktické vykonávání úřadu
samozřejmě doživotně trvat nemusí, ba mohou pro to vzniknout překážky.
Podívejme se na konkrétní dílčí rozdíly.

Církev československá husitská rozlišuje dva \textbf{stupně kněžského svěcení},
a to jáhenské a samotné kněžské. Církev Římskokatolická přidává k~těmto dvěma
ještě svěcení biskupské. Historicky bylo stupňů více, ale i tehdy byly
zjednodušeně řečeno právě tyto tři povážovány za svátostné. Církev
československá husitská oplývá taktéž biskupy (a svůj charakter s~oblibou
dekoruje přívlastkem \textit{,,s prvkem episkopálním``}), ale biskupství je pro
ni úřadem, či pověřením, nikoliv stavem, jenž by vyžadoval zvláštní svěcení.

U Římskokatolické církve se rozlišuje mezi jáhenstvím přechodným jako
mezistupněm kněžství a jáhenstvím trvalým. Na přechodné jáhenství jsou kladeny
obecně nižší nároky, ale kandidát musí splňovat předpoklady pro následné plné
kněžství.

\section{Předpoklady}

Obecných podmínek pro kněžké svěcení neuvaluje Církev československá husitská
mnoho, a ty, které uvaluje, skýtají prostor pro interpretaci, přesto však jsou
smysluplné a troufám si konstatovat, že jejich dodržování odfiltruje mnoho
nevhodných kandidátů. Jsou to
\begin{itemize}
\item{křest\footnote{Preambule řádu duchovenské služby CČSH},}
\item{zakotvenost v~církvi a ve víře, včetně náležení k~některé náboženské obci,}
\item{občanská a morální bezúhonnost\footnote{Ústava CČSH čl. 10, odst. 1}}
\item{a dobrý zdravotní stav.\footnote{Řád duchovenské služby CČSH §13, odst. 1}}
\end{itemize}

Všechny tyto předpoklady platí i pro kandidáty na svěcení u církve
Římskokatolické, která navíc vyžaduje minimální věk dvaceti pěti let pro
kněžství a trvalé jáhenství a dvaceti tří let pro jáhenství
přechodné.\footnote{CIC 1031 §1} Dále je svěcení umožněno toliko svobodným
mužům\footnote{CIC 1024} -- pouze o trvalé jáhenství se může ucházet muž ženatý
se souhlasem manželčiným. Vyžadováno je také biřmování a absolvování mezistupňů
lektora a akolyty, vždy s~odstupem alespoň šesti měsíců.

\section{Vzdělání}

V~otázce potřebného vzdělání vyžaduje Církev československá husitská
pro jáhenské svěcení završení bakalářského studia husitské teologie (HT1 nebo HT2) na HTF UK. Pro
kněžské svěcení pak totéž na magisterské úrovni.\footnote{Řád duchovenské služby
CČSH §2, odst. 1} S~povolením ústřední rady
církve na doporučení svěcencova biskupa lze uznat i jiné srovnatelné vzdělání po
případném složení rozdílových zkoušek.\footnote{tamtéž odst. 2}

Kromě akademického vzdělání též stanovy vyžadují osobnostní formaci kandidátovu
v~době studia. Studenta má oslovit spirituál a na studentovu seminární formaci
dohlédnout podle zvláštní církevní směrnice.\footnote{tamtéž §14}
Po vysvěceném knězi zastávajícím úřad se vyžaduje další
vzdělávání.\footnote{tamtéž §2, odst. 4, písm. k}
Před svěcením je také stanovena zkouška před církevní komisí pověřenou
biskupy.\footnote{tamtéž §16, odst. 3}

U Římskokatolické církve je pro kněžské svěcení potřeba čtyřletá seminární
formace, dva roky studia filozofie a čtyři roky studia teologie.\footnote{CIC
232-264} Po pěti letech studia lze udělit svěcení na přechodého
jáhna.\footnote{CIC 1031, § 2}
Seminární řád je detailně rozveden ve výše citovaných zákonech. 

Katolické požadavky jsou v~jednom ohledu méně specifické: CČSH vyžaduje
konkrétní studijní program z~konkrétního institutu. V~ostatních ohledech jsou
náročnější. Vyžaduje se šestileté studium filozofie a teologie, což se zdá
srovnatelné s~magisterským studiem teologie na HTF, ovšem je třeba neopomenout,
že v~současné akreditaci je podmínkou pro přijetí do navazujícího magisterského
programu libovolné předchozí studium bakalářského či vyššího stupně. Přitom
formálně ani fakticky při patřičném studijním nasazení nic nebrání tomu, aby se
navazující magisterské studium zvládlo během jediného roku, což je oproti
povinným šesti opravdu značný rozdíl. Nevidím toto ovšem jako problém, neboť se
nezdá, že by svěcení po jednoletém studiu byla nějak rozšířená praktika a hlavně
je zde kontrolní element ve formě světícího biskupa, který, dá se snad
spolehnout, nevysvětí člověka, kterému chybí potřebné teologické znalosti.
Naopak to, že v~systému je flexibilita, vidím jako pozitivní element harmonující
s~tím, že se jako církev neopíráme tolik o formální aparát, ale více o organické
naplňování autority Ducha Kristova.

Za politováníhodné však považuji, když se dobrá pravidla nedodržují ne snad
protože by bylo vhodné je porušit, ale protože k~jejich plnění chybí vůle. Mé
srdce zaplesalo, když jsem se dozvěděl, že naši kněží mají povinnost se dále
vzdělávat a mají od církve právo na veškerou podporu k~tomuto vzdělávání. Kolik
ale kněží tohoto práva využívá a od kolika kněží někdo tuto jejich povinnost
očekává?

Dovolím si zabrousit na tenký led akademického psaní přimícháním subjektivních
teologických postojů. Domnívám se, že právě zanedbávání dalšího vzdělávání je
jednou z~příčin toho, že se v~naší církvi stále daří liberalistickým blábolům,
které zpochybňují věčný život, přítomnost Krista v~chlebu a víně a dokonce
Ježíšovo božství.

\chapter{Závěr}

Je vidět, že podmínky kněžského svěcení jsou obecně méně přísné v~Církvi
československé husitské než v~církvi Římskokatolické. Můžeme v~tom vidět naší
církvi od počátku vlastní svobodomyslnost a důraz na autoritu Ducha Kristova
oproti formálnímu aparátu, jehož vnímaná zkostnatělost a nabobtnalost u
Římskokatolické církve jistě mnoha moderním husitům byla důležitým bodem
vymezení. Nelze však opominout i pohled z~druhé strany, který naši církev a
vlastně skoro všechny menší církve ukazuje jako štěňátka, která sbírají drobty u
pánova stolu. Nemůžeme mít tak přísná pravidla, protože lidé, kteří by rádi ke
katolíkům jako k~,,primární`` církvi, ale neprojdou jejich formálními požadavky,
tvoří velkou část naší členské i duchovenské základny. Za úsměvný, ale výstižný
považuji výraz pro Starokatolickou cíkev jako ,,katolicismus s~lidskou tváří``.

Tento pohled zajisté není lichotivý a je na nás, abychom s~touto realitou
správně zacházeli. Je sice nezbytné se jakkoliv nelichotivé realitě podívat do
očí a je-li vskutku realitou, přijmout ji jako součást Boží vůle, ovšem hřivnu
nezakopávat, nýbrž s~ní hospodařit. Domnívám se, že odkaz, který v~Církvi
československé husitské od svých zakladatelů neseme -- a to včetně návaznosti na
katolickou tradici -- je skutečným pokladem, který může přinést užitek třicátý,
šedesátý i stý.

Jestliže tolik kněží bude i nadále zasmrádlými úředníčky, kteří si sedí na svých
nuzných postech a vystačí si s~tím, že přednesou kázání, kterým sebe i svých pět
oveček utvrdí v~dosavadních mentálních klecích, zahořkle krčíce rameny nad
řídnoucími počty věřících, můžeme se spolehnout, že budeme jako koukol hozeni do
ohně neuhasitelného. V~době skutečné, probíhající apokalypsy musíme vynaložit
všechno úsilí na plnění Boží vůle, nelze jet dál ve stejných kolejích.

Kněží, kteří mají celou dvoutisíciletou tradici k~dispozici, ale nejsou jí
striktně vázáni, kněží, kteří se nezpovídají římskému diktátorovi, ale mají
zodpovědnot vůči svému národu a vůči svému Mistrovi pánu Ježíšu Kristu, kněží,
kteří si vzali za úkol prodchnout současné poznání Duchem Kristovým, jsou
nejenže nejlepší, ale možná i jedinou nadějí na proměnu naší doby
z~přetechnizované, strachem ovládané a ekonomickým růstem zotročené tyranie do
předobrazu království Božího, které se i nyní přiblížilo.

