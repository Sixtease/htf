\chapter{Úvod}
\label{kap:uvod}

\section{Prameny}

Tuto práci píšu z~pohledu hrdého a oddaného člena Církve československé
husitské, a adepta na kněžství v~ní. Pokusil jsem se nastudovat potřebné
materiály svojí církve tak, abych se mohl skromně domnívat, že jejím stanovám
alespoň povrchně a v~rámci možností své pramalé vyzrálosti v~oboru, jakož i
svého omezeného intelektu rozumím. Naopak komplexní porozumění stanovám jiných
církví, včetně té římskokatolické, není mojí ambicí, a proto jsem do jejích
materiálů nahlížel jen bodově za konkrétním účelem vydobytí faktů nezbytných pro
sepsání tohoto krátkého pojednání.

Čerpal jsem tedy ze Základů víry Církve československé husitské, z její ústavy,
řádů a agendy. Pro fakta o Římskokatolické církvi jsem nabral odrazový můstek
v~knize Církevní právo od autorů Tretera, Horák vydané roku Páně dvoutisícího
sedmnáctého, a dále jsem se opíral o kanonické právo, dostupné na stránkách
katolické církve.

\section{Kněžství}

Kněžství je vágně definovaný pojem namnoze závislý na úhlu pohledu. Kněží se
vyskytovali od nejranějších historických dob podnes a vyskytují se
v~nejrozličnějších směrech, tedy ve velké šíři jak na časové ose tak na ose
lokálně-kulturní. A na obou z~nich se pojetí značně liší. Pokusu o uchopení
pojmu se vyhnu, protože moje ambice není v~pojednání o kněžství jako takovém,
nýbrž ve srovnání jeho konkrétních dvou realizací, a to současného
římskokatolického a Církve československé husitské.

Zaměřme se nejdříve na společnou základnu. Církev československá husitská se
odloupla od církve Římskokatolické před sto jedním rokem, zdědivše od ní
v~podstatě všechno, co ji definovalo, včetně pojmu kněžství. Zakotvené je
biblicky hlavně v~pastorálních epištolách, které vyjmenovávají potřebné
vlastnosti knězovy, a můžeme si jako členové reformační církve s~libostí
povšimnout, že například v~otázce celibátu stojí biblické instrukce zcela jasně
na ,,naší`` straně. Je až úsměvné, že právě tento bod, který tak okatě odporuje
katolickému kanonickému právu, je výplodem té části bible, kterou má na svědomí
raný katolicismus. Pravdou však zůstává, že vývoj církve včetně kněžství a
kněžstva, byl společný oběma našim zkoumaným subjektům ještě dlouho předlouho po
dobách vzniku pastorálních epištol. To, že se Církev československá husitská
rozhodla (a velmi dobře!) po kněžích celibát nevyžadovat, nelze tedy vnímat
odloučené od faktu, že mnoho jiných nebiblických praktik od římských katolíků
převzala. A i tady si dovolím toto hodnotit jako dobré, protože z~oné
dvoutisícileté tradice by se měly zavrhnout jen špatné věci, ne všechny jen
proto, že nejsou v~bibli, která při vší zbožné úctě nemůže být jediným kanálem
Ducha svatého v~lidském životě.

Zpět však k~naší otázce společného základu: Kněží v~obou
církvích mají jiný status nežli ostatní členové církve. Jsou osoby pověřené a
výlučně oprávněné ke konání svátostných úkonů a jsou ke své církvi ve služebním
poměru. Skládají slib věrnosti, podepisují smlouvu, dostávají mzdu. Dostávají
liturgický oděv, který smějí nosit při svátostných
obřadech\footnote{Bohoslužebný řád CČSH část A bod 5; CIC 929}.

\section{Apoštolská posloupnost a obecné kněžství}
V~samotném pojetí najdeme i zásadní rozdíly. Především jde o to, že kněžství
v~Římskokatolické církvi se opírá o apoštolskou posloupnost\footnote{viz
Církevní právo s. 36}, zatímco u Církve
československé husitské o obecné kněžství všech křesťanů spojené s~pověřením
církve k~profesionální službě\footnote{Základy víry CČSH 342-347; 3. část agendy
s. 23}.

Apoštolská posloupnost je fascinující fenomén --
jeden z~těch, u kterých není ani zdravě pochybujícímu skeptikovi zatěžko
přijmout její historickou nepřerušenost. Vždyť už římský biskup Klement v~prvním
století se ve svém prvním listu korintským o důsledné péči o apoštolskou
posloupnost zmiňuje\footnote{1Clem 44}, a Tertullian o Klementovi píše, že byl
vysvěce samotným Petrem\footnote{De praescriptione haereticorum 32,2}. Pro
ochranu a pojištění apoštolské posloupnosti se později začalo dbát na to, aby
každého biskupa světilo několik biskupů najednou.

Na druhou stranu domnívat se, že by Duch svatý sestupoval na lidi podle toho,
jestli na ně vložil ruce někdo, na koho před tím někdo vložil ruce... řekněme,
že to z~Boha svrchovaného a nekonečně moudrého dělá štafetový kolík. Vycházet
z~obecného kněžství je tedy v~duchu reformačním jednoznačně více na straně pravé
zbožnosti.

Nebyl bych však spravedlivý, kdybych nepodotkl, že opora Církve československé
husitské v~kněžství obecném namísto v~apoštolské posloupnosti je v~pravém slova
smyslu znouzectnost, neboť získání legitimní apoštolské posloupnosti
byla v~začátcích církve palčivou, leč marnou touhou.

Nemohu říci, že bych to nechápal. Jakkoliv je absurdní domnívat se, že by
vkládání rukou mohlo ovládat Ducha svatého, nelze ani vyloučit, že rituál
skutečně nějakým způsobem obcování s~Duchem svatým zprostředkuje. Jako nelze
Duchu svatému přikázat, aby se vložením rukou vlil, ani mu to nelze zakázat.
Nemohu v~této souvislosti nepomyslet na anekdotu s~Richardem Feynmanem, jenž jsa
navštíven novinářem, byl dotázán, proč že má nade dveřmi zavěšenou podkovu.
Řekl, že pro štěstí. Novinář se zeptal, zdali na to věří. Feynman odpověděl:
,,Nevěřím. Ale to neznamená, že to nefunguje.``

Přesto vidím v~absenci formální apoštolské posloupnosti pro naši církev
požehnání: Kdo se nemůže spolehnout na formální prostředky, musí se obrátit
k~živému Duchu svatému a dosvědčovat jeho působení svým životem.

\chapter{Svěcení}

Svěcení je v~obou církvích aktem uvedení osoby do stavu kněžství. Výsledek je
doživotní a nezvratný status svěcence, ovšem faktické vykonávání úřadu
samozřejmě doživotně trvat nemusí, ba mohou pro to vzniknout překážky.
Podívejme se na konkrétní dílčí rozdíly.

Církev československá husitská rozlišuje dva \textbf{stupně kněžského svěcení},
a to jáhenské a samotné kněžské. Církev římskokatolická přidává k~těmto dvěma
ještě svěcení biskupské. Historicky bylo stupňů více, ale i tehdy byly
zjednodušeně řečeno právě tyto tři povážovány za svátostné. Církev
československá husitská oplývá taktéž biskupy (a svůj charakter s~oblibou
dekoruje přívlastkem \textit{,,s prvkem episkopálním``}), ale biskupství je pro
ni úřadem, či pověřením, nikoliv stavem, jenž by vyžadoval zvláštní svěcení.

\section{Vzdělání}

V~otázce potřebného vzdělání vyžaduje Církev československá husitská
pro jáhenské svěcení završení bakalářského studia husitské teologie (HT1 nebo HT2) na HTF UK. Pro
kněžské svěcení pak totéž na magisterské úrovni.\footnote{Řád duchovenské služby CČSH §2
odst. 1} S~povolením ústřední rady
církve na doporučení svěcencova biskupa lze uznat i jiné srovnatelné vzdělání po
případném složení rozdílových zkoušek.\footnote{tamtéž odst. 2}

Kromě akademického vzdělání též stanovy vyžadují osobnostní formaci kandidátovu
v~době studia. Studenta má oslovit spirituál a na studentovu seminární formaci
dohlédnout podle zvláštní církevní směrnice.\footnote{tamtéž §14}
Po vysvěceném knězi zastávajícím úřad se vyžaduje další
vzdělávání.\footnote{tamtéž §2, odst. 4, písm. k}

V 
