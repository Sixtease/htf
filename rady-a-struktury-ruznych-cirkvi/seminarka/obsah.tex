\chapter{Kněžství}
\label{kap:uvod}

Kněžství je vágně definovaný pojem namnoze závislý na úhlu pohledu. Kněží se
vyskytovali od nejranějších historických dob podnes a vyskytují se
v~nejrozličnějších směrech, tedy ve velké šíři jak na časové ose tak na ose
lokálně-kulturní. A na obou z~nich se pojetí značně liší. Pokusu o uchopení
pojmu se vyhnu, protože moje ambice není v~pojednání o kněžství jako takovém,
nýbrž ve srovnání jeho konkrétních dvou realizací, a to současného
římskokatolického a Církve československé husitské.

Zaměřme se nejdříve na společnou základnu. Církev československá husitská se
odloupla od církve Římskokatolické před sto jedním rokem, zdědivše od ní
v~podstatě všechno, co ji definovalo, včetně pojmu kněžství. Zakotvené je
biblicky hlavně v~pastorálních epištolách, které vyjmenovávají potřebné
vlastnosti knězovy, a můžeme si jako členové reformační církve s~libostí
povšimnout, že například v~otázce celibátu stojí biblické instrukce zcela jasně
na ,,naší`` straně. Je až úsměvné, že právě tento bod, který tak okatě odporuje
katolickému kanonickému právu, je výplodem té části bible, kterou má na svědomí
raný katolicismus. Pravdou však zůstává, že vývoj církve včetně kněžství a
kněžstva, byl společný oběma našim zkoumaným subjektům ještě dlouho předlouho po
dobách vzniku pastorálních epištol. To, že se Církev československá husitská
rozhodla (a velmi dobře!) po kněžích celibát nevyžadovat, nelze tedy vnímat
odloučené od faktu, že mnoho jiných nebiblických praktik od římských katolíků
převzala. A i tady si dovolím toto hodnotit jako dobré, protože z~oné
dvoutisícileté tradice by se měly zavrhnout jen špatné věci, ne všechny jen
proto, že nejsou v~bibli, která při vší zbožné úctě nemůže být jediným kanálem
Ducha svatého v~lidském životě.

Zpět však k~naší otázce společného základu: Kněží v~obou
církvích mají jiný status nežli ostatní členové církve. Jsou osoby pověřené a
výlučně oprávněné ke konání svátostných úkonů a jsou ke své církvi ve služebním
poměru. Skládají slib věrnosti, podepisují smlouvu, dostávají mzdu. Dostávají
liturgický oděv, který smějí nosit při svátostných
obřadech\footnote{Bohoslužebný řád CČSH část A bod 5; CIC 929}.

V~samotném pojetí se liší

\chapter{Svěcení}

Svěcení je v~obou církvích aktem uvedení osoby do stavu kněžství. 
