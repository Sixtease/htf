\def\Abstrakt{%
Český mystik Karel Makoň (*1912 \textdagger1993) po sobě zanechal rozsáhlé a významné dílo v~psané a
mluvené podobě.
%Po mystickém zážitku v~koncentračním táboře strávil zbytek života vedením ostatních k~zasvěcení života hledání Království Božího.
V~desítkách
knih, které napsal, rozvádí návod k~využití pozemského života pro vstup do
vědomého spojení s~Věčností. Hovořil ke skupinkám příznivců, z~čehož se
zachovalo přes tisíc hodin magnetofonových záznamů.
Pro šíři jeho hovorů je těžké určit, jaké jsou hlavní body jeho poselství, a
dostupná literatura o něm je velmi sporadická.
Vyskytují se závažné výroky, které v~obměnách opakuje po více
než dvacet let svého auditivně zaznamenaného působení. Mezi tyto opakující se
výroky patří například: ,,Trpnost je nezbytnou součástí cesty k~Bohu.``,
,,Modlitba nesmí být mechanická, aby měla spojovací účinek.``, ,,Milost Boží je
zákonitým jevem.``
Tyto lze s~jistotou zařadit mezi stěžejní pilíře jeho nauky.
Jaké jsou ale ty ostatní?
Pomocí počítačové analýzy přepisů nahrávek lze dospět k~seznamu kandidátských témat,
jenž může být základem pro vyčerpávající seznam pokrývající celý mluvený korpus.
V~mluveném korpusu Karla Makoně lze nalézt odpovědi na všechny základní otázky systematické teologie.
Mnohé z~těchto odpovědí jsou v~kontroverzi k~církevním naukám.
Esence Makoňova poselství lze shrnout starověkým citátem ,,Tento život je mostem do věčnosti.``
Jiní jeho příznivci uvádějí ale jiná shrnutí.
}
