\chapter{Dostupná literatura o Karlu Makoňovi}

O Karlu Makoňovi téměř žádná literatura neexistuje. Makoň sám zanechal řádově
více materiálu, než kolik bylo napsáno o něm někým jiným. Je běžné, že
z~autorské literatury lze vyčíst informace o autorově životě. V~případě Karla
Makoně je tomu tak ještě více, než je obvyklé. Svá sdělení Makoň často opírá o
vlastní zkušenostia zážitky, bere sám sebe jako ukázku toho, jak v~lidském
životě působí Boží milost. Proto Makoňovy spisy a záznamy přednášek jsou zdaleka
nejbohatším dostupným zdrojem informací o něm samém.

\section{Mozaika mystického života Karla Makoně}

Na Makoňovu památku byla vydána
kniha \textit{Mozaika mystického života Karla Makoně}\cite{mozaikaKM}, kterou
jeho příznivci vydali deset let po jeho smrti.
Osmdesátistránková knížka má formu osobních vzpomínek na Karla Makoně. Jsou zde
vylíčeny základní prvky Makoňova poselství, některé výrazné příhody, které s ním
jeho žáci zažili a především je zde zachycen duch, ve kterém Makoně jeho žáci
vnímali.

Hlavním hybatelem vzniku Mozaiky byl Stáňa Kalibán, který má asi právem reputaci
Makoňova nejoddanějšího žáka. Kalibán se po Makoňově smrti snažil popularizovat
jeho dílo. V~Mozaice je autorem úvodu, doslovu, stručného Makoňova životopisu,
výběru ukázek z~Makoňova díla a stejně jako u ostatních spoluautorů osobní
vzpomínky. Ta je v~jeho případě rozdělena do kapitol, kde vypravuje, jakou
moudrostí ho Karel Makoň obohacoval, a uvádí anekdoty ilustrující Makoňovu
výjimečnost.

Druhý přispěvším je Jenda Baxa. Ten se ve svém svědectví soustředí na
předurčenost svého setkání s~Makoněm. Vypráví příběh svého setkání s~ním, jaké
tam byli pro něho nepochopitelné zdánlivé náhody a líčí samozřejmost, se kterou
Makoň konstatuje Boží vedení.

Třetím přispěvším je MUDr. Vít Elger. On také pořídil, katalogizoval a po
desetiletí uchovával drtivou většinu nahrávek Makoňových slov, než jsem je
digitalizoval. Jako jediný ze spoluautorů Mozaiky v~době psaní tohoto textu
žije. Dr. Elger představuje ve své krátké stati Makoňovo dílo poměrně
systematicky. Soustředí se na jeho knihy a ke každému aspektu Makoňovy nauky
uvádí, ve které knize je akcentován. Zdůrazňuje závažnost a závaznost Makoňovy
nauky a opírá se i o biblické citáty. Jeho svědectví na rozdíl od ostatních není
opřeno o anekdotické zkušenosti, nýbrž zčásti skrytě zčásti otevřeně přirovnává
Makoně osobním významem k~Ježíši Kristu.

Čtvrtým přispěvším je Ruda Müller. Ten ve svém kraťoučkém příspěvku na necelé
dvě strany zasazuje Makoňovo dílo do kontextu svojí doby. Zdůrazňuje, že přišel
mezi lidi, kteří měli načteného Bruntona, Weinfurtera, dostupnou indickou
filozofii a další duchovní potravu, která byla v~té době na veřejném jídelníčku.
Makoň jim podle jeho slov ukázal, že ničemu nerozumějí, a odhalil jim ještě
mnohem větší moudrost skrývající se v~křesťanství, které bylo ,,u nosu``, ale
v~němž ji nehledali.

Pátým přispěvším je Miloš Skalický. Bez výhrad ho mohu nazvat nejpokročilejším
Makoňovým žákem. Skalický se sám dobral osobního bohatého styku s~Boží
přirozeností, o čemž svědčí jeho zatím nevydané spisy jako např. Existenční
cesta. Stal se duchovním vůdcem na setkáních přátel Karla Makoně, která
pokračují dodnes v~návaznosti na setkání s~Makoněm samotným. Byl jsem svědkem
jeho duchovní pohotovosti, hlubokého vhledu a stále větší naléhavosti, se kterou
nás odváděl od piety k~Makoňovi ke skutečné cestě k~Bohu tady a teď. Skalický už
v~Mozaice, dávno před tím, než jsem se s~ním mohl setkat, píše o Makoňovi
pohledem člověka, který se na něho dívá s~úctou, ale kriticky. Popisuje Makoňovu
tělesnost, vývoj v~jeho přístupu a jako ostatní předává svoje nejvýraznější
zkušenosti v~Makoňově přítomnosti.

Posledním, šestým přispěvším je Pavel Válek. V~jeho stati je mně jediné známé
nezávislé psané svědectví o Makoňově schopnosti hovořit zvířecí řečí. Pavel
Válek se pozastavuje nad Makoňovým vztahem k~vlastní rodině v~kontextu velkého
množství času, které věnoval duchovním přátelům. Vyzdvihuje návodnost Makoňových
rad, zejména v~přístupu k~eucharistii.

Společným prvkem všech příspěvků v~knize Mozaika mystického života Karla Makoně
je vděčnost. Také lze konstatovat, že většina přispěvších zařazuje Makoňovo dílo
i své vlastní zaměření do oblasti křesťanské mystiky a dávají tento pojem do
kontrastu jednak s~běžným rituálním či mravoučným křesťanství, jak ho tehdy
vnímali, a jednak s~mystikou orientální.

Za souhrnné poselství Mozaiky tedy můžeme považovat to, že Makoň svou naukou a
svým životem odhaluje křesťanskou mystiku jako cestu k~Bohu, která se nabízí
soudobému člověku. Osobně považuji tuto knížku za cenný doplněk k~dokreslení
představy o tom, kým Karel Makoň byl, ale pro seznámení s~ním se lze lépe
obrátit na Makoňovo vhodně volené autorské dílo dle založení zájemcova.

\section{Závěrečné studentské práce}

V~roce 2015 byla na Fakultě humanitních studií UK obhájena diplomová práce Kláry
Pokorné s~názvem \textit{,,Čeští ezoterici v~proměnách soudobých dějin``}. Zde
je o Karlu Makoňovi jen několikerá letmá zmínka v~poznámkách pod čarou, kde je
uveden jako člen generace předcházející té, o které práce pojednává. Zároveň je
zde uvedeno, že na Makoně existuje složka v~archivu StB, ovšem konstatuje jeho
zničení.

Celá věnovaná Karlu Makoňovi je bakalářská práce Radmily Müllerové obhájené roku
2019 na Filozofické fakultě Masarykovy univerzity. Jejím titulem je
\textit{,,Vyprávění o utrpení: Formování identity a důvěryhodnosti mystika Karla
Makoně``}, což harmonuje se zaměřením na element utrpení a na formování Makoňovy
identity mystika. Práce čerpá z~výseku Makoňových děl, kde se soustředí 

% okomentovat spolupráci; zmínit Makoňovo vyprávění o výslechu

% Makoň a Kočí: úplnost a správnost stačí k~zásahu, ale očekávání limituje
% účinek, viz Zlatý klíč
% Pokouší se Makoňovo poznání souvislostí odůvodnit zasazováním událostí do
% "symbolického světa", čili souvislosti implikuje jako Makoňovy konstrukty

%*mozaika
%bakalářka
%dingir
%web
%moje
