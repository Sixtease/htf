\chapter{Dostupná literatura o Karlu Makoňovi}

O Karlu Makoňovi téměř žádná literatura neexistuje. Makoň sám zanechal řádově
více materiálu, než kolik bylo napsáno o něm někým jiným. Je běžné, že
z~autorské literatury lze vyčíst informace o autorově životě. V~případě Karla
Makoně je tomu tak ještě více, než je obvyklé. Svá sdělení Makoň často opírá o
vlastní zkušenosti a zážitky, bere sám sebe jako ukázku toho, jak v~lidském
životě působí Boží milost. Proto Makoňovy spisy a záznamy přednášek jsou zdaleka
nejbohatším dostupným zdrojem informací o něm samém.

\section{Mozaika mystického života Karla Makoně}

Na Makoňovu památku byla vydána
kniha \textit{Mozaika mystického života Karla Makoně}\cite{kaliban2002mozaika}, kterou
jeho příznivci vydali deset let po jeho smrti.
Osmdesátistránková knížka má formu osobních vzpomínek na Karla Makoně. Jsou zde
vylíčeny základní prvky Makoňova poselství, některé výrazné příhody, které s ním
jeho žáci zažili, a především je zde zachycen duch, ve kterém Makoně jeho žáci
vnímali.

Hlavním hybatelem vzniku Mozaiky byl Stáňa Kalibán, který má asi právem reputaci
Makoňova nejoddanějšího žáka. Kalibán se po Makoňově smrti snažil popularizovat
jeho dílo. V~Mozaice je autorem úvodu, doslovu, stručného Makoňova životopisu a
výběru ukázek z~Makoňova díla. Stejně jako ostatní spoluautoři přispěl též
osobní vzpomínkou. Ta je v~jeho případě rozdělena do kapitol, kde vypravuje, jakou
moudrostí ho Karel Makoň obohacoval, a uvádí anekdoty ilustrující Makoňovu
výjimečnost.

Druhým přispěvším je Jenda Baxa. Ten se ve svém svědectví soustředí na
předurčenost svého setkání s~Makoněm. Vypráví příběh svého setkání s~ním, jaké
tam byly pro něho nepochopitelné zdánlivé náhody, a líčí samozřejmost, se kterou
Makoň konstatuje Boží vedení.

Třetím přispěvším je MUDr. Vít Elger. On také pořídil, katalogizoval a po
desetiletí uchovával drtivou většinu nahrávek Makoňových slov, než jsem je
digitalizoval. Jako jediný ze spoluautorů Mozaiky v~době psaní tohoto textu
žije. Dr. Elger představuje ve své krátké stati Makoňovo dílo poměrně
systematicky. Soustředí se na jeho knihy a ke každému aspektu Makoňovy nauky
uvádí, ve které knize je akcentován. Zdůrazňuje závažnost a závaznost Makoňovy
nauky a opírá se i o biblické citáty. Jeho svědectví na rozdíl od ostatních není
opřeno o anekdotické zkušenosti, nýbrž zčásti skrytě zčásti otevřeně přirovnává
Makoně osobním významem k~Ježíši Kristu.

Čtvrtým přispěvším je Ruda Müller. Ten ve svém kraťoučkém příspěvku na necelé
dvě strany zasazuje Makoňovo dílo do kontextu svojí doby. Zdůrazňuje, že přišel
mezi lidi, kteří měli načteného Bruntona\cite{brunton1951hidden},
Weinfurtera\cite{weinfurter1923ohnivy}, dostupnou indickou
filozofii\cite{weinfurter1935bhagavadgita} a další duchovní potravu, která byla v~té době na veřejném jídelníčku.
Makoň jim podle jeho slov ukázal, že ničemu nerozumějí, a odhalil jim ještě
mnohem větší moudrost skrývající se v~křesťanství, které bylo ,,u nosu``, ale
v~němž ji nehledali.

Pátým přispěvším je Miloš Skalický. Bez výhrad ho mohu nazvat nejpokročilejším
Makoňovým žákem. Skalický se sám dobral osobního bohatého styku s~Boží
přirozeností, o čemž svědčí jeho kniha Tajemství života / Tajemství
smrti\cite{skalicky2022tajemstvi} i zatím nevydané spisy jako např. Existenční
cesta. Stal se duchovním vůdcem na setkáních přátel Karla Makoně, která
pokračují dodnes v~návaznosti na setkání s~Makoněm samotným. Byl jsem svědkem
jeho duchovní pohotovosti, hlubokého vhledu a stále větší naléhavosti, se kterou
nás odváděl od piety vůči~Makoňovi ke skutečné cestě k~Bohu tady a teď. Skalický už
v~Mozaice, dávno před tím, než jsem se s~ním mohl setkat, píše o Makoňovi
pohledem člověka, který se na něho dívá s~úctou, ale kriticky. Popisuje Makoňovu
tělesnost, vývoj v~jeho přístupu a jako ostatní předává svoje nejvýraznější
zkušenosti v~Makoňově přítomnosti.

Posledním, šestým přispěvším je Pavel Válek. V~jeho stati je mně jediné známé
nezávislé, psané svědectví o Makoňově schopnosti hovořit zvířecí řečí. Pavel
Válek se pozastavuje nad Makoňovým vztahem k~vlastní rodině v~kontextu velkého
množství času, které věnoval duchovním přátelům. Vyzdvihuje návodnost Makoňových
rad, zejména v~přístupu k~eucharistii.

Společným prvkem všech příspěvků v~knize Mozaika mystického života Karla Makoně
je vděčnost. Také lze konstatovat, že většina přispěvších zařazuje Makoňovo dílo
i své vlastní zaměření do oblasti křesťanské mystiky a dávají tento pojem do
kontrastu jednak s~běžným rituálním či mravoučným křesťanstvím, jak ho tehdy
vnímali, a jednak s~mystikou orientální.

Za souhrnné poselství Mozaiky tedy můžeme považovat to, že Makoň svou naukou a
svým životem odhaluje křesťanskou mystiku jako cestu k~Bohu, která se nabízí
soudobému člověku. Osobně považuji tuto knížku za cenný doplněk k~dokreslení
představy o tom, kým Karel Makoň byl, ale pro seznámení s~ním se lze lépe
obrátit na Makoňovo vhodně volené autorské dílo dle založení zájemcova.

\section{Závěrečné studentské práce}

V~roce 2015 byla na Fakultě humanitních studií UK obhájena diplomová práce Kláry
Pokorné s~názvem \textit{,,Čeští esoterici v~proměnách soudobých dějin``}\cite{pokorna2015cesti}. Zde
je o Karlu Makoňovi jen několikerá letmá zmínka v~poznámkách pod čarou, kde je
uveden jako člen generace předcházející té, o které práce pojednává. Zároveň je
zde uvedeno, že na Makoně existuje složka v~archivu StB, ovšem konstatuje jeho
zničení.

Celá věnovaná Karlu Makoňovi je bakalářská práce Radmily Müllerové\cite{mullerova2019vypraveni} obhájená roku
2019 na Filozofické fakultě Masarykovy univerzity. Jejím titulem je
\textit{,,Vyprávění o utrpení: Formování identity a důvěryhodnosti mystika Karla
Makoně``}, což harmonuje se zaměřením na element utrpení a na formování Makoňovy
identity mystika. Práce čerpá z~výseku Makoňových děl, kde se Makoň věnuje
srovnání jógy s~křesťanskou mystikou a Zjevení Sv. Jana, což ústí v~dojem, že je
taková celá jeho nauka.

V~práci se několikrát představují dva hlavní momenty utrpení v~Makoňově životě:
Operace ramene v~předškolním věku a pobyt v~koncentračním táboře vyvrcholivší
v~jistotu vlastní smrti. Po stručném životopisu uvádí jako osobnosti, které měly
na Makoně vliv, Karla Weinfurtera, Šrí Aurobinda Ghóše (který je konzistentně
s~Makoňovým vlastním zvykem uváděn jako \textit{,,Ghos``}) a Alana Wattse.
To pravdě odpovídá: Weinfurter Makoně formoval, Makoň přeložil Ghóšovu knihu Syntéza
jógy\cite{aurobindo1999synthesis} a komentuje jeho dílo ve své knize Srovnání
  jógy s~křesťanskou mystikou\cite{KaMaGhos}, a od Wattse přeložil hned čtyři
  tituly, jimiž jsou Cesta
zen\cite{watts1957way}, Moudrost nejistoty\cite{watts1951wisdom}, Příroda, muž a
žena\cite{watts1973nature} a Svrchovaná totožnost\cite{watts1950supreme}.

Poté již rozvádí odtělesňování jako předmět Makoňovy interpretace. Uvádí
Selvaradžana Jesudiana jako Makoňova jógového mistra, k~čemuž já doplňuji, že
oni dva se v~tělech nepotkali. Makoň však opět přeložil dvě jeho knihy: Jógu ve
dvou světech napsanou s~Elisabeth Haich\cite{yesudian1951yoga} a Sebevýchovu
jógou\cite{yesudian1974self}. U odtělesňování představuje Müllerová Makoňův pojem
obapolnosti, se kterým Makoň napracuje často, ale tím zajímavější je jeho
pojednání.

V~další sekci si všímá Makoňova opouštění těla, kterému se spontánně
naučil při extrémní bolesti. Poukazuje, že Makoň se ve výkladu k těmto událostem
opírá o nauku o čakramech z~rádžajógy, jak ji měl nastudovanou, a o biblické
příběhy. Dává jeho přístup také do kontrastu s~přístupem lidí, kteří se
odtělesňit nedokážou a pro které nebylo utrpení odrazovým můstkem, ale kamenem úrazu.

V~následující sekci autorka srovnává Makoně s~Bedřichem Kočím. U obou pozoruje
\textit{,,úplné``} a \textit{,,správné``} sebeodevzdání Bohu a zákonitě
přicházející Boží pomoc, přičemž pozoruje, že Makoň zdůrazňuje u sebe absenci
očekávání pomoci od Boha, zatímco Kočí očekával Boží pomoc v~uzdravení. Vyvozuje
z~toho, že kvalita sebeodevzdání je postačující podmínkou, zatímco absence
očekávání již nikoliv. Případnému zájemci o detailní rozvinutí této problematiky
samotným Makoněm lze doporučit jeho překlad a komentář ke knize Zlatý
klíč\cite{fox1931golden} v~rámci Cesty vědomí\cite{KaMaCV2}. Obě
svědectví o Boží pomoci jsou pak zpochybněna a relativizována v~následujícím
odstavci, kde se na příběhu žen lotyšských alkoholiků naznačuje, že člověk
kauzální souvislosti mezi událostmi ve svém životě a vyšší mocí sám přisuzuje.
V~závěru sekce je vyzdvižena Makoňova absence nenávisti vůči působitelům
utrpení.

Další sekce pojednává o principu příčiny a následku. Porovnává se zde přístup
Roberta F. Murphyho\cite{murphy2001body}, který podle autorky odmítá objektivnost principu karmy, a
Makoňův přístup, který s~karmou pracuje, ale nikoliv jako s~vládnoucím
principem. Uvádí jednu ze základních Makoňových pouček, že \textit{každá krize
je způsobena nerovnováhou mezi vnitřním a vnějším životem}. Autorka poučku volně
parafrázuje: \textit{,,Utrpení [...] vzniká, když nastává rozpor mezi duchovní
složkou života [...] a fyzickými činy nebo jednáním ve fyzickém světě.``}
Ve zbytku práce se Makoňovo zácházení s~utrpením shrnuje.

\section{Dingir 2007/4}

Článek v~religionistickém časopise Dingir z~roku 2007\cite{hajek2007cesky} byl
až do napsání výše zmíněné bakalářské práce jedinou mně známou oficiálně vydanou
publikací zabývající se Karlem Makoněm ve třetí osobě. Autor článku Jurik Hájek
představuje na dvou stránkách Makoňovo dílo a opět stručně předkládá Makoňův
životní příběh.

Autor zmiňuje, že motivem jeho příspěvku je především to, že záznam o Karlu Makoňovi
nenachází v~dostupných pojednáních o české mystice 20. století. Považuje přitom
Karla Makoně za důležitou osobnost tohoto proudu. Důvod k~opomenutí Makoně
v~publikacích vidí v~tom, že Makoň působil mimo veřejnost a média. Poukazuje
také na fakt, že Makoňova díla se relativně těžko čtou, což přispívá k~jeho
mizivé popularitě.

Článek několika větami představuje díla Pohádky pro děti, vyšedší pod titulem
Odkrytá moudrost starých pravd\cite{makon1992odkryta},
Blahoslavenství\cite{makon2000blahoslavenstvi} a Otázky a odpovědi V, které
také dostaly knižní podobu s~titulem Slabikář na cestu životem, ovšem oficiálně
nikdy nevyšly. U Blahoslavenství vyzdvihuje důležité Makoňovo poselství, že
evangelní Ježíšův život je návodem na vstup do věčnosti rozvedeným do nejmenších
podrobností. Celkově oceňuje hloubku Makoňova pojetí. Závěrem vyjadřuje naději,
že Makoňovo dílo může pozitivně ovlivnit dialog mezi vědou a náboženstvím,
přičemž problém náboženského přístupu vidí v~doslovném pojetí Bible.

V~článku též zaznívá jedna myšlenka, která stojí za citaci:
\begin{quote}
Makoň [...] hovoří o tom, že tento stav [nebe] je
nutno dosáhnout a zažít již zde na zemi ve
své pravdivé osobní zkušenosti. [...]
%Celé Písmo, a zvláště evangelijní texty Ježíšovy nauky nám ukazují cestu k dosažení.
%Každé slovo posvátného textu má svùj hlubinný smysl, který je nutno nejen pochopit, ale realizovat.
Toto pojetí zajisté není výlučně Makoňovo, [...]
%setkáme se s ním v celé linii mystického myšlení - znovu opakuji: mám na mysli
%mystiku, jako hledání a nalézání skry- tého smyslu našeho života, jako cestu,
%jejíž itineráø je ukryt v posvátných textech kaž- dého náboženství. Obdobné
%základní myš- lenky najdeme i v èeské mystice, u W ein- furtera, u J. Vacka aj.,
nicméně u Karla Makoně je celá tato koncepce podána systematickým, vnitřně
koherentním výkladem, takže skoro můžeme hovořit o ,,ontologii`` duchovního
vývoje člověka -- i když Makoň nikde nezabředá do samoúčelných filosofických
úvah. Jasná koncepce, znalost duchovních proudů Východu i Západu, hluboká osobní
zkušenost -- upřednostňují Makoňovo duchovní dílo před většinou ostatních
představitelů české mystiky (a nejen české).
\end{quote}

\section{Web}

Existuje webová stránka věnovaná Makoňovi, \url{makon.cz}, kde je uloženo celé
jeho dílo a několik slov o jeho životě. Krom toho jsem na webu o Makoňovi našel
jen útržkovité informace, většinou shrnutí jeho života s~důrazem na příhody
s~operacemi ramene a z~koncentračního tábora.
Skupiny a stránky věnované Makoňovu odkazu jsem na sociálních sítích nenašel prakticky
žádné. Sporadické zmínky v~diskuzních fórech nejsou pomocí vyhledávačů
k~dohledání.

\section{Moje předchozí práce}

Makoňovu dílu je věnována moje disertační práce na Matematicko-fyzikální fakultě
UK\cite{kruuza2021iterativni}, na které jsem pracoval v~letech 2011 až 2021. Druhá kapitola je věnovaná
popisu Makoňova díla v~rozsahu, jaký umožňoval technický obor, v~rámci něhož se
práce obhajovala. Úvod kapitoly obsahuje stručnou charakteristiku Makoňovy nauky
z~mého subjektivního pohledu. Vymezuji ji oproti moderním duchovním autorům,
oproti římskokatolické církevní nauce a oproti vědeckému bádání. Následuje sekce
o Makoňově osobě. V její první podsekci vyprávím Makoňův život, opět zdůrazňuje
milníky, které se nejvýrazněji podílely na zformování Makoně jako mystika.
V~další podsekci představuji Makoňovo psané dílo.  Vyjmenovávám vydané knihy,
nastiňuji proces jejich vzniku a šíření, kvantitativní charakteristiky a uvádím
seznam zejména obsáhlých a jinak vyčnívajících titulů. Závěr sekce o Makoňově
osobě je věnován jeho nauce.

Druhá sekce kapitoly věnované Makoňovu dílu pojednává o tématech, kterých se
dotýkají záznamy přednášek.

Především však vznikl webový archiv zvukových nahrávek, kde jsou
prakticky všechny nahrávky k~poslechu, stažení a opatřeny kompletním přepisem,
jenž umožňuje vyhledávání.

%*mozaika
%*bakalářka
%*dingir
%web
%moje
