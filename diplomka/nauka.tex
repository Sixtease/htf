\chapter{Nauka Karla Makoně}

Z~toho mála, co o Karlu Makoňovi bylo napsáno a publikováno, se většina týká
nevšedních epizod jeho života, popřípadě zhodnocení jeho díla či nástin některé
jednotlivosti jeho obsáhlé nauky. Moje disertační práce z~tohoto vybočuje tím,
že se snaží aspoň nástínit seznam témat, kterým se Makoňovo dílo věnuje. Toto
úsilí bych rád rozvedl v~této kapitole a rád bych, aby pokus o zmapování
hlavních prvků Makoňova poselství bylo hlavním přínosem této diplomové práce.

Vzhledem k~samotné kvantitě Makoňova díla je tento můj záměr téměř bláhový. Sám
jsem přečetl jen šest Makoňových knih a poslechl necelou polovinu jeho
dostupných nahrávek. Vzhledem k~tomu, že přístupnost Makoňových nahrávek je
z~velké části mým dílem, na kterém jsem strávil více než dekádu, a k~tomu, že
prakticky všechny dosavadní zmínky o Makoňově díle se soustředí na jeho knihy,
dává mi smysl na moje předchozí úsilí navázat a soustředit se na nahrávky.

Záměr zmapovat množinu hlavních poselství v~Makoňově díle vychází krom jiného
také z~toho, že po poslechnutí většího množství nahrávek docházím k~pozorování,
že navzdory velké rozmanitosti otázek, které Makoň zodpovídá, se vývody
v~odpovědích s~obměnami opakují. Do jaké míry se opakují, jak moc zůstávají
výpovědi napříč vývojem Makoňovy nauky obsahově totožné, a kolik vlastně
takových opakujících se sdělení je, jsou hlavní otázky, které si pokládám.

\section{Vybraná témata}

libovolný systém koncentrace, např. písmenová cvičení

libovolný systém sebezáporu, např. judaismus nebo křesťanská morálka

dharmou každého je vědomá věčnost

ježíšův život je dokonalý návod do všech podrobností

katolická tradice je nedoceněná pomůcka, ale potřebuje korekci

Otčenáš je řád hodnot

automatismus II

\section{}

\subsection{trpnost}

Vidíme, že Karel Makoň klade opakovaně důraz na rovnováhu protikladů:
činnost - trpnost,
modlitba - život,
vnitřní život - vnější život.
Mohlo by se zdát, že jeho nauka spočívá v~tom, žít v~rovnováze. To by však bylo
velmi zavádějící, protože samotná rovnováha může být statická. Už samotné slovo
evokuje představu vah, na nichž v~klidu spočívají \textit{rovné váhy}. Makoň ale
krom rovnováhy zdůrazňuje také nutnost absolutního úsilí. Máme podle něho jít za
jediným cílem celou svojí bytostí. Činnost vyvíjet na samé hranice svých
možností a odevzdat se i za cenu vlastní smrti.

Rovnováha je tedy u něho principem, který je třeba zachovat, ale samotné kráčení
po cestě bez odboček je pro něho v~podstatě stěžejnější.

Rovnováhu mezi modlitbou a životem inde nazývá rovnováhou mezi vnitřním
a vnějším životem, a to obvykle v~kontextu toho, že porušení této rovnováhy je
příčinou veškerého utrpení, a že tedy klíčem k~životu bez utrpení je tato
rovnováha. To je jistě pozoruhodné. Ne každý podá návod na život bez utrpení.

\subsection{Modlitba}

Kategorické odmítnutí
mechanicky prováděné modlitby podává jako něco, co místo aby k~Bohu přivádělo, od něj
odvádí, takže není jen modlitbou podřadnou, nekvalitní, ale něčím, co je
s~modlitbou v~přímém protikladu. Podává ji to jako zlo, kterého je potřeba se
varovat jako něčeho škodlivého.

\subsection{Dharma}

Makoň hlásá, že máme vzít za svůj cíl vědomé spojení s~Bohem. ,,Hledejte
Království Boží především.`` I ti, kdo tak nečiní, však jdou nedobrovolně
tamtéž, neboť Království Boží je jediná společná dharma všem lidem, tedy jediný
nejvyšší úkol, který mají naplnit všichni lidé, a to po dobrém či po zlém. Po
zlém však ne z~nějakého násilí ze strany Boží, nýbrž protože v~hloubi duše po
tom člověk touží, a tak pokud na tom nepracuje, připraví mu sama jeho duše
donucovací prostředky, kterými ho přiměje na cestě třeba těžkopádně pokračovat
vpřed.
