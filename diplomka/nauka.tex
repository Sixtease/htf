\chapter{Nauka Karla Makoně}

Z~toho mála, co o Karlu Makoňovi bylo napsáno a publikováno, se většina týká
nevšedních epizod jeho života, popřípadě zhodnocení jeho díla či nástin některé
jednotlivosti jeho obsáhlé nauky. Moje disertační práce z~tohoto vybočuje tím,
že se snaží aspoň nástínit seznam témat, kterým se Makoňovo dílo věnuje. Toto
úsilí bych rád rozvedl v~této kapitole a rád bych, aby pokus o zmapování
hlavních prvků Makoňova poselství bylo hlavním přínosem této diplomové práce.

Vzhledem k~samotné kvantitě Makoňova díla je tento můj záměr téměř bláhový. Sám
jsem přečetl jen šest Makoňových knih a poslechl necelou polovinu jeho
dostupných nahrávek. Vzhledem k~tomu, že přístupnost Makoňových nahrávek je
z~velké části mým dílem, na kterém jsem strávil více než dekádu, a k~tomu, že
prakticky všechny dosavadní zmínky o Makoňově díle se soustředí na jeho knihy,
dává mi smysl na moje předchozí úsilí navázat a soustředit se na nahrávky.

Záměr zmapovat množinu hlavních poselství v~Makoňově díle vychází krom jiného
také z~toho, že po poslechnutí většího množství nahrávek docházím k~pozorování,
že navzdory velké rozmanitosti otázek, které Makoň zodpovídá, se vývody
v~odpovědích s~obměnami opakují. Do jaké míry se opakují, jak moc zůstávají
výpovědi napříč vývojem Makoňovy nauky obsahově totožné, a kolik vlastně
takových opakujících se sdělení je, jsou hlavní otázky, které si pokládám.

\section{Vybraná témata}

libovolný systém koncentrace, např. písmenová cvičení

libovolný systém sebezáporu, např. judaismus nebo křesťanská morálka

dharmou každého je vědomá věčnost

ježíšův život je dokonalý návod do všech podrobností

katolická tradice je nedoceněná pomůcka, ale potřebuje korekci

Otčenáš je řád hodnot

automatismus II
