\def\AbstraktEN{%
Karel Makoň (1912-1993) has left behind an extensive and significant opus, both
written and spoken. After a mystical experience in a concentration camp, he
spent the rest of his life urging others to devote their lives to seeking the
Kingdom of God. In the dozens of books he has written he elaborates on a how-to
for entering the conscious Life Eternal. He was talking to several groups of
followers, which resulted in over a thousand hours of magnetophone tape
recordings. Available literature about him is very sporadical. The breadth of
his talks makes it hard to determine the main points of his message. There are
repeated significant statements appearing over the span of more than twenty
years of his recorded oral activity. Some of these repeated statements are:
``Passivity is an essential part of the path to God.'', ``Prayer must not be
mechanical if it is to have its connecting effect.'', ``God's grace is an
inevitable phenomenon.'' A list of candidate topics in Makoň's spoken corpus can
be obtained through computational analysis of his transcribed recordings. This
can be a basis for an exhaustive list of topics covering the entire material.
All foundational questoins of systematical theology can be answered using
Makoň's talks. Many of these answers are controversial towards ecclesiastical
doctrines. The essence of Makoň's message can be summed up using the ancient
quote ``The life is a bridge into the Eternity''. Others come up with different
summaries though.
}
