\chapter{Osobnost Karla Makoně}

Pokusím se přiblížit, co za člověka Karel Makoň byl tím, že převyprávím jeho
život podle toho, jak a s~jakými důrazy ho podává on sám. Je to příběh plný
zázraků, což odpovídá tomu, že Makoň na svém životě demonstruje působení
zázračného tažení Božího. Nic mu to ale aspoň v~mých očích neubírá na
pravdivosti.

\section{Poznávání správného}

Karel Makoň se narodil 12.12.1912 v~Ošelíně u Plzně. Jeho otec několik měsíců na
to zemřel na tuberkulózu a Karel sám dostal zánět do levého ramene. Ten dospěl
do takového stádia, kdy lékař doporučil amputaci ruky. Matka toto odmítla a tak
započala první fáze Makoňovy duchovní formace. Rameno bylo nutno operovat.
Jelikož se ještě nepoužívala transfúze, musela se operace provádět na
několikrát, protože jinak by dítě vykrvácelo. A kvůli Makoňově útlému věku a
stavu tehdejší anesteziologie musely operace probíhat za vědomí. Dítě bylo
vystaveno takové bolesti, že se naučilo vystupovat z~těla. Toto se podle
Makoňových slov opakovalo čtrnáctkrát, což malého Karla Makoně nezvratně
změnilo.

Ve stavech mimo tělo, kdy necítil bolest, se setkával s~něčím, co nazývá
,,žlutým světlem``.  Důsledkem bylo, že dítě začalo samovolně rozlišovat, co je
správné, a byl vnitřně puzen správné následovat. Správné od nesprávného dokázal
rozpoznat, ale neměl schopnost správné řešení situace sám vymyslet. Důvěřoval
matce a tak ji častoval otázkami, jak se má zachovat. Mnohokrát pak její návrh
odmítl s~tím, že tohle dělat nebude a trval na tom, že musí dostat jiný. Jeho
matce se zdálo, že dítě si svévolně vybírá a matku trápí.

V~období raného dětství vyčníval Karel Makoň nejen puzením ke správnému, ale
také intenzivním stykem se zvířaty. Aby se operované rameno neporanilo, měl
Karel Makoň zapovězeno hrát si s~jinými dětmi. Trávil proto svůj volný čas
v~předškolním věku téměř výhradně o samotě se zvířaty, obzvláště s~husami.
Naučil se rozumět jejich řeči a ony ho přijali mezi sebe jako jednoho z~nich.

Makoň poznal, že husy ve stavu spánku nejsou nečinné, nýbrž žijí intenzivním
životem na jiné úrovni. Vzaly ho oďobáváním na správných místech s~sebou a on
s~nimi prožíval ,,stav zvířecího ráje``, kde bylo přítomno celé stádo, ale
nikoliv v~tělech, nýbrž bez nich. Prožívali pospolitost, viděli na sobě, jaké
mají neduhy a co je potřeba k~jejich překonání.

Poznání skutečného zvířecího ráje, ke kterému nemá běžný člověk přístup, a
puzení vyhnout se nesprávnému, vedly Karla Makoně k~pevnému rozhodnutí vystříhat
se veškerého ,,pohlavního života``, jak sexuální aktivitu konzistentně nazývá.

Ve škole prožíval Karel Makoň těžké loučení se zvířecím světem a sžívání s~tím
lidským. Poznával ale jak v~matce tak v~učitelích autority, takže se převychovat
nechal. Vedlo to nakonec k~tomu, že Karel Makoň si v~životě vytvořil jedinou
skutečnou lásku, a tou byly přírodní vědy. Toužil stát se přírodovědcem víc než
cokoliv jiného. Když ho pak rodinná finanční situace donutila se tohoto snu
vzdát, ztratil svoji jedinou lásku a s~tím jakoby sama sebe.
Tímto dospěl do druhého zlomu po operacích ramene.

\section{Modlitba za lásku}

Makoň byl nucen studovat ekonomii a aby aspoň nějak ukojil touhu po přírodních
vědách, chodil si číst přírodovědné knihy do knihovny. Při jedné takové
příležitosti namátkou otevřel knihu a oči mu padly na větu, která ho uchvátila a
strhla do prvního vytržení. Stálo tam \textit{,,tento život je mostem do
věčnosti``}.

%Koluje legenda, že ve dvanáct hodin a
%dvanáct minut, ovšem sám Karel Makoň zmiňuje v~Umění žít\cite{KaMaUZ}, bylo to
%podle záznamu jeho otce v jednu hodinu a devatenáct minut v noci.
