\chapter{Osobnost Karla Makoně}

Karel Makoň se narodil 12.12.1912 v~Ošelíně u Plzně. Jeho otec několik měsíců na
to zemřel na tuberkulózu a Karel sám dostal zánět do levého ramene. Ten dospěl
do takového stádia, kdy lékař doporučil amputaci ruky. Matka toto odmítla a tak
započala první fáze Makoňovy duchovní formace. Rameno bylo nutno operovat.
Jelikož se ještě nepoužívala transfúze, musela se operace provádět na
několikrát, protože jinak by dítě vykrvácelo. A kvůli Makoňově útlému věku a
stavu tehdejší anesteziologie musely operace probíhat za vědomí. Dítě bylo
vystaveno takové bolesti, že se naučilo vystupovat z~těla. Toto se podle
Makoňových slov opakovalo čtrnáctkrát, což malého Karla Makoně nezvratně
změnilo.

Ve stavech mimo tělo, kdy necítil bolest, se setkával s~něčím, co nazývá
,,žlutým světlem``.  Důsledkem bylo, že dítě začalo samovolně rozlišovat, co je
správné, a byl vnitřně puzen správné následovat. Důvěřoval svojí matce a 

%Koluje legenda, že ve dvanáct hodin a
%dvanáct minut, ovšem sám Karel Makoň zmiňuje v~Umění žít\cite{KaMaUZ}, bylo to
%podle záznamu jeho otce v jednu hodinu a devatenáct minut v noci.
