\chapter{Kontext Makoňova působení}

\section{Makoňova doba}

Karel Makoň prožil skoro celé dvacáté století, takže jeho doba byla velmi
různorodá. Narodil se ještě v~Rakousku-Uhersku, čemuž odpovídá jeho tradiční
katolická výchova. Do základní školy chodil v~čerstvě vzniklém
Československu, takže zakusil ještě školství v tereziánském duchu. Budeme-li u
něho rozeznávat tři hlavní období:
\begin{enumerate}
\item{puzení ke správnému, tj. od operací ramene do první extáze (1930),}
\item{modlitbu za lásku k~Bohu, tj. od první extáze do mystické smrti
v~koncentračním táboře (1939) a}
\item{aktivní tvorbu, tj. od propuštění z~KZ do smrti,}
\end{enumerate}
zkonstatujeme, že první období -- dětství prožíval v~době první světové války a
rané první republiky. Druhé období prožíval v~době pozdní první republiky a
během prvních dnů protektorátu. Za druhé světové války ho zastihl nejvyznamnější
formující prožitek v~koncentračním táboře. Většinu války i další události už pak
prožil ve stavu praktické odpoutanosti od vnějších poměrů.

Je zřejmé, že vnější okolnosti hrály v~jeho formování do roku 1939 zásadní roli:
\begin{itemize}
\item{stav medicíny, který umožnil operace, ale neumožnil anestezii ani
transfúzi,}
\item{prostý venkovský život, který umožňoval strávit většinu raného dětství
s~husami,}
\item{trvající vliv rakousko-uherského katolicismu,}
\item{rostoucí popularita okultismu a mystiky}
\item{a konečně nacistický teror vedoucí k~prožitku mystické smrti.}
\end{itemize}

Makoňovo zaměření nebylo do záležitostí tohoto světa a jeho fundamentální
poselství je nadčasové, nezávislé na poměrech. Přesto by bylo mylné se domnívat,
že vnější okolnosti po roce 1939 Makoně neovlivňovaly. Ovlivňovaly, a to
zásadně, byť se ho nemusely niterně dotýkat. Jednak sice jeho zaměření nemuselo
být do záležitostí tohoto světa, ale lidské duše, kterým chtěl dopomoci ke
spáse, se v~tomto světě nacházely, a ve světě viděl ke spáse nezbytný
prostředek, jednak se za jeho působení výrazně změnily podmínky pro šíření
náboženských nauk. Po skončení druhé světové války v~Československu, na krátko
zase svobodném, mohl Makoň svoje svědectví šířit jako středoškolský učitel
otevřeně i mezi žáky a samozřejmě mezi známými. Když pak došlo v~roce 1948
k~bolševickému puči, stalo se postupně náboženství i duchovno obecně
tabuizovaným oborem\cite{kaplan1993stat}. Makoň v~souladu se svojí naukou, že nemá smysl plýtvat
silami na boj s~režimem -- podobně jako sv. Josef utekl od Heroda, místo aby
s~ním bojoval -- se stáhl do ústraní. Působil dál, ale v~uzavřených kruzích.

Dočasné rozvolnění v~šedesátých letech ani pád socialistického bloku v~roce 1989
tento Makoňův styl nezměnily. Na veřejnost ani do médií se nijak nehrnul a
pokračoval ve svém psaní a přednášení a celkovém nenápadném působení do konce
života.

\section{Předchůdci}

Makoň se sám řadí do tradice křesťanské mystiky. Bylo by tedy vhodné zařadit ho
do kontextu této tradice. Odvolává se na mnoho autorů. Dá se ale říci, že jen
dva jsou podle všech kritérií jeho předchůdci, a těmi jsou Mistr Eckhart a svatá
Terezie z~Avily.

Makoň mezi autory, na které navazuje, rozeznává dvě kategorie:
\begin{enumerate}
\item{ty, kteří prožili mystickou smrt a}
\item{ty, kteří mystickou smrt neprožili.}
\end{enumerate}
Nechci hodnotit, nakolik je to legitimní rozdělení a nakolik bylo
z~Makoňovy strany přesné. Jednak se k~tomu necítím povolán a jednak jsem coby
Makoňův příznivce pozitivně předpojat. Nejde mi zde však o nějaké objektivní
vymezení toho, kdo jaký byl, ale o to, ke komu se jakým způsobem Makoň hlásí.
Jeho předchůdce tedy míním v~tom smyslu, koho by za svého předchůdce sám asi
považoval.

Z~těch, kteří se křesťanské mystice věnovali, ale kteří podle Makoně smrtí na
kříži neprošli, jmenuje např. Molinose nebo Bernarda z~Clairvaux, kterého rád
nazývá kvůli podílu na křížové výpravě prasákem\footnote{Viz \texttt{91-05A} od
pozice 12:45.}. Odkazuje se naopak na některé
světce, které skutečně za svaté (a tedy prošedší mystickou smrtí) považuje, ovšem
kteří se nezabývali křesťanskou mystikou tak vyhraněně, aby mohli s~ním být
postaveni do řady. Z~nich jmenujme např. svatého Antonína Velikého nebo svatou
Anežku Českou. V~podobné kategorii by mohli být Makoňovy vzory z~nekřesťanských
kultur: Lao C', Šrí Aurobindo Ghóš, Ramakrišna. Lze zmínit i Makoňovy vzory
biblické: Například proroka Eliáše a pisatele Starého zákona považuje za lidi
prošedší také smrtí na kříži.

Dva stěžejní Makoňovi předchůdci, kterým připisuje nemenší úroveň poznání, než jakou
rozpoznával u sebe, a kteří se také věnují křesťanské mystice, se samozřejmě od
Makoně značně odlišují. Jednak oba dva měli už za svého života značný vliv a
pozornost, jednak je nauka každého z~nich posazená do zcela jiné doby. Jejich
porovnání by bylo na samostatné práce, ale zmiňme aspoň ty nejpatrnější rozdíly:
Většina dochovaných Eckhartových děl jsou kázání, takže zcela jiný žánr, než do
jakého můžeme zařadit Makoňovo dílo. Eckhartova řeč je také mnohem více
postavena na šokujících a těžko srozumitelných výrocích, které mají vytrhnout
posluchačovu mysl a unést ji k~Bohu, zatímco Makoň píše systematicky a, mám-li
parafrázovat jeho vlastní slova, píše tak, aby přesvědčil rozum, že se má dát do služeb
Božích. Svatá Terezie z~Avily naproti tomu musela harmonovat, aspoň do jisté
míry, s~tehdejší církevní doktrínou.

\section{Současníci}

Kategorii Makoňových současníků můžeme vymezit bez ohledu na jeho vlastní postoj
vůči nim, i když samozřejmě Makoňovy postoje jsou jedním z~hlavních kritérií,
podle kterých můžeme rozhodnout, kterého z~jeho současníků máme jmenovat. Makoň
začínal působit v~době, kdy česká mystika byla prakticky definovaná osobností
Karla Weinfurtera\cite{sanitrak2006dejiny1}. Je proto logické, že se na něj
často odvolává, že v~jejich učeních nalezneme mnohou shodu, i že se vůči
Weinfurterovi Makoň vymezuje. Vymezení je především vůči praktikám jeho
následovatelů a podobnosti nalezneme velmi hojně a netřeba pro ně chodit
daleko. Hned v~úvodu k~Ohnivému keři\cite{weinfurter1923ohnivy} nalézáme
myšlenku, že je potřeba uvést na pravou míru chybně interpretované křesťanské
učení. Makoň, stejně jako Weinfurter, pokládá jógu za ekvivalent mystické cesty,
odvolává se na rádža jógu, Paula Bruntona atd.

Druhým výrazným Makoňovým současníkem je Eduard Tomáš. Tomáš byl starší než
Makoň, ale zemřel později (*1908 \textdagger2002). I jejich nauky se překrývají
v~mnoha bodech, a to i těch hlavních: Není trvalého štěstí v~žádném předmětu
tohoto světa; naše skutečná podstata je božská; atd.

Hlavní rozdíly oproti Weinfurterovi i Tomášovi vidím dva shodné: Oba byli
exponovaní na veřejnosti, zatímco Makoň nikoliv, to za prvé. Za druhé, Makoň
sice nezapadá do žádné církevní doktríny, ale je vyhraněně křesťanský.
Zdůrazňuje, že jediným mistrem je Ježíš Kristus. Jediným vzorem hodným
následování je Ježíš Kristus. Makoň se vlastně snaží přivést lidi k~tomu, aby
plodně následovali Ježíšova příkladu. V~tomto se jak od Weinfurtera, pro kterého
je podstatná mystická zkušenost, i od Tomáše, který klade důraz na Absolutno,
liší. Pro Tomáše je království nebeské stav štěstí a klidu\footnote{Paměti
mystika 1. díl, od 02:52 \url{youtu.be/4nVY5RvOjs4?t=172}.}. Pro Makoně je to
podřízení vlastní vůle vůli Boží. A z~mého hlediska to Makoně staví na pevnější
půdu.

\section{Shrnutí}

Většina velkých osobností dojde svého zařazení a hodnocení až s~jistým odstupem.
Jak dlouho to bude trvat u Karla Makoně, nelze předvídat. Některé osobnosti,
jako svatá Hildegarda z~Bingen, došly širšího uznání až skoro po mnoha
staletích\footnote{Srovnej počty publikací o sv. Hildegardě podle roku jejich
vzniku.}. Jiné možná nikdy, což nevíme, protože kdybychom věděli, nebyly by
neznámé. Dával jsem Karla Makoně do srovnání s~osobnostmi, od kterých všech se
odlišoval tím, že nedospěl popularity. Je to logické, neboť ho nemůžeme
srovnávat s~neznámými osobnostmi, opět protože o nich nevíme. Navíc je tato
práce psána z~pohledu člověka, který v~Makoňovi velkou a vyčnívající osobnost
spatřuje.

Je jasné, že vliv okolí na Makoňovu tvorbu nelze zhodnotit: Každá událost je
organicky zařazena do svého prostředí a v~jiných podmínkách ji zkrátka nelze
zkoumat. Snad se však podařilo nastínit aspoň nejhrubší souvislosti, bez jejichž
povědomí nelze Makoňovo dílo hodnotit ani mu rozumět.
