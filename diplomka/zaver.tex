\chapter{Závěr}

\section{Esence Makoňova poselství}

Nemělo by být překvapením, že po napsání této práce docházím ke shrnutí Makoňova
poselství v~tomto smyslu:
\begin{quote}
,,Smyslem lidského života je vědomé spojení s~Bohem,
což se rovná vědomému prožívání vlastní věčné podstaty. Vše má sloužit jako
prostředek k~tomuto cíli.``
\end{quote}
Elegantněji by věc šla vyjádřit slovy starověkého citátu:
\begin{quote}
,,Tento život je mostem do věčnosti.``
\end{quote}
Jakkoliv se může zpochybňovat jeho historická vazba na rané
křesťanství\cite{dus2001neznama}, jeho pravdivost byla pro Makoně natolik
strhující, že se stal jádrem jeho nauky.

Tento závěr je každopádně subjektivní, jakkoliv je podložen četnými ukázkami.
Abych si ověřil validitu svého závěru, poprosil jsem další příznivce Makoňovy
nauky, aby mi sdělili v~jedné či několika málo větách, co oni považují za esenci
Makoňova poselství. Každý z~dotázaných na rozdíl ode mne Makoně osobně zažil a
byli jím formováni po mnoho let svých životů. Zde jsou jejich odpovědi:

\begin{itemize}
  \item{\textit{,,Cíl vždycky nejvyšší.``}}
  \item{\textit{,,Nelpěte na událostech, aby dopadaly podle vašich představ. Nechte věci, ať
    se dějí.``}}
\item{\textit{,,Zapomínat na sebe. Jsme povinováni láskou ke všemu a ke všem. Nenávist
  není přípustná. I ten, kdo nejde cestou lásky, musí zapomínat sám na sebe, aby
    vystoupil ze svého já.``}}
  \item{\textit{,,Nepřekážejte Bohu.``}}
  \item{\textit{,,V~každém živém tvoru musí být Bůh neustále přítomen: ,Bohové jste.`
    Největší překážkou na cestě k~Bohu je vlastnění všeho druhu.``}}
\end{itemize}

Je vidět, že rozmanitost jednotlivých shrnutí vysoko převyšuje jejich
příbuznost. Je to zajímavá ukázka toho, jak je každý prakticky tímtéž ovlivněn
rozdílně dle svého založení. Každopádně lze s~radostí konstatovat, že jednotlivé
varianty si snad neodporují.

\section{Karel Makoň jako náboženská autorita}

Karel Makoň se může jevit jako kacíř -- a je-li kacířství odchýlení se od
církevní
nauky\footnote{\href{https://www.oxfordreference.com/view/10.1093/oi/authority.20110803095932433}{oxfordreference.com/view/10.1093/oi/authority.20110803095932433}},
pak je kacířem zcela přiznaně a otevřeně. Sám tvrdí, že definicí
kacířství je vydávání části učení za celek, proto sebe za kacíře nepovažuje. Je
důležité vzít v~potaz, že Makoň se vůbec nesnaží zaštiťovat autoritou nějaké
církve. Nemluví jako křesťan, ale jako člověk, který čerpá z~vlastního vhledu.
Církev vidí často zkresleně, nebo aspoň dnešní církve neodpovídají tomu, jak o
nich Makoň mluví, což staví jejich vývoj za poslední dekády do velmi dobrého
světla. Makoň vidí v~katolické tradici nedoceněný a nesmírně propracovaný
prostředek na cestě k~vědomému spojení s~Bohem. Ostatně všechno na světě, i
Ježíše samotného a celý lidský život považuje za prostředek na cestě ke spojení
s~Bohem. Biblický text a katolickou tradici ale hodnotí jako obzvláštní
prostředky, protože nejsou tak kontextově závislé jako ostatní. Může se jimi
řídit každý a v~každé fázi\rref{76-01-Kaly}{01:20:53}. Biblickému textu je potřeba správně rozumět a
katolickou tradici je třeba náležitě opravit.

Z~katolického hlediska je tedy kacířem zcela jasně. Z~hlediska reformovaného
možná také. Ovšem zda má Makoň pravdu, to je zcela jiná otázka. Každopádně
slibuje něco, co církev neslibuje. Nejde mu o spásu po smrti, ani o víru v~to,
že už jsme spaseni. Jde mu o spásu reálnou, zcela jasně rozlišenou, u které
absolutně není pochyb, zda nastane či nastala. A celý život zasvěcuje tomu, aby
jí došli pokud možno všichni lidé. Věří, že to lze. Sám ji zažívá, spatřuje v~ní
jedinou skutečnou hodnotu a navádí k~ní s~nadlidskou vytrvalostí, přesvědčivostí
a mnohdy i zázračně\footnote{Viz např. Mozaika mystického života Karla Makoně
str. 29 - 48.}.

Mohl by se nám jevit jako egocentrik. Mluví sám o sobě a svých zážitcích častěji
než kdo jiný. Svoje zážitky bere jako model pro ostatní. Na některých místech je
patrno, že si je vědom svojí převahy. Vidět ho ale jako egoistu by byl
fatální omyl. Svoji výjimečnost zapřahá nekompromisně do svého poslání přimět
svoje posluchače, aby se stali dokonalými následovníky Kristovými. Říká, že
Satan je jáčko a radí ho přemoct tím, že ho zapojíme do spasitelského úkolu. Ať
má sám jáčko velké jakkoliv (a po pravdě, vzhledem k~jeho objektivním kvalitám,
se mně osobně zdá příkladně skromným), aplikuje na sobě tuto svoji poučku
nekompromisně.

Mohl by se jevit jako jakýsi \textit{,,hyperpelagianista``}. Vždyť neguje prvek
Božího výběru a Boží svobody v~lidské spáse. Predestinaci unifikuje na
univerzální predestinaci ke spasení a jediné, na čem záleží, je lidský prvek.
Zde je však nutné vzít v~potaz, že to jediné, co člověk může udělat, aby se
disponoval spáse, je podle Makoně odstoupení od sebe. Mnohokrát zdůrazňuje, že
vlastní silou se člověk nikdy nemůže dostat dál než k~extázi, kterou zavrhuje
jako překonaný fenomén\rref{78-05A-Kaly}{07:12}. I pro Makoně vše záleží na Bohu, který nám však dal
svobodu bránit se spáse.

Mohli bychom ho obvinit z~ezoterismu. Vždyť hovoří o rozmlouvání se záhrobním
světem\rref{kotouc-D01-d}{08:10}, o vílách\rref{80-07B}{24:46} a o
magii\rref{80-12}{00:00}. Nikdy však obcování s~\textit{,,vyššími sférami``}
nemá za cíl a nevede k~nim. Hovoří o nich buď pro odlehčení nebo častěji pro
dokreslení kontextu, vždy s~jasným cílem pomoci člověku co nejvíce na cestě do
Království Božího.

Jeho práce s~Biblí je velmi svérázná a podle mě ji nelze napodobit či z~ní
učinit metodu, ačkoliv se jeví, že Makoň po svých posluchačích právě toto
chtěl\footnote{\rid{kotouc-77-brezen01-d}{01:25:56}, \rid{82-07}{01:01:00}}.
Vychází někdy z~pochybných či zcela mylných stanovisek. Například že Máří
Magdaléna je totožná s~prostitutkou odsouzenou k~ukamenování (J 8, 3-11)\rref{83-22B}{29:46}, nebo že jsme
všichni potomci Kainovi\rref{83-27}{31:09}, nebo že \textit{,,Bůh tak svět miloval, že dal svého jediného
Syna, aby žádný, kdo v~něho věří, nezahynul, ale měl věčný život,``} (J 3, 16) jsou slova
svatého Pavla\rref{80-01B}{12:24}. Několikrát tvrdí, že dokáže vyložit jakékoliv
podobenství\rref{86-25A}{27:20}, ovšem
prakticky se omezuje jen na hrstku z~nich. Radí vidět Bibli jako dokonalý návod
na cestu do Království Božího rozvedený do všech detailů, ovšem při svých
exegezích vrhá příběhy do jiného světla, než v~jakém jsou v~samotné
Bibli\rref{85-06B}{24:43}.

Zajisté to má souvislost s~tím, že dostal filozofické a ekonomické vzdělání a
nikoliv teologické a biblistické. Makoň křesťanství zná z~výchovy v~tradičním
římskokatolickém prostředí a detaily dostudovával později z~dostupných zdrojů na
vlastní pěst. Znalosti má rozsáhlé, ale zcela pochopitelně mu schází moderní
poznatky v~biblistice.

Pro mne osobně toto bylo asi největším rozčarováním, neboť jsem dlouho Makoňova
slova měl za ryzí pravdu a cestu, kterou lze doslova následovat a splnit tak
výzvu, kterou před nás předkládá: Vejít v~rámci tohoto života do vědomého
spojení s~Věčností, do Království Božího. Stále tuto jeho výzvu chci přijmout,
ovšem už si nemyslím, že by stačilo se řídit Makoňovou naukou coby jediným
vodítkem. Makoňovu práci s~Biblí převzít nedokážu a ani nechci. Stěžejní přístup
ale ano, a nemohu se tak vzdát svého vlastního přínosu, svojí vlastní cesty,
která se nutně od té Makoňovy musí lišit. Budiž proto Bohu vzdána čest a dík za
to, že Makoňova nauka má i své nedostatky. Ty jsou však vůči hodnotě, kterou
Makoňovo dílo má, naprosto nicotné.

Karel Makoň nám odkázal něco, co palčivě chybí: Životní program s~nejvyšším
možným cílem, strhávající celé naše mnohovrstevnaté a roztěkané osobnosti, a využívající
všechno, co svět nabízí. Nezavrhuje nic. Tradici, vědu, technologii, utrpení,
požitky, zkrátka všechno je pro něho prostředkem na cestě k Bohu, a žádného
prostředku se na této cestě nemáme štítit.

Přidávám se k~autorům Mozaiky mystického života Karla Makoně v~hluboké vděčnosti
za to, že jsem se s~jeho odkazem mohl setkat. Kéž tento skvost prospěje mnohým
dalším.
