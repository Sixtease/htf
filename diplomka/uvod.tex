\chapter{Úvod}
\label{kap:uvod}

Karel Makoň je téměř neznámou postavou české literatury a kupodivu i české
mystiky. Jeho knihy nejsou prakticky v~žádné knihovně o knihkupectvích nemluvě.
Není o něm zmínka v~encyklopediích, v~době psaní tohoto textu (duben 2022) ani
na Wikipedii. Jeho dílo je přitom jedinečné jak rozsahem (27 autorských spisů,
28 překladů a komentářů k~nim, přes 1000 hodin záznamů přednášek) tak i
závažností obsahu.

Když jsem se s~jeho dílem poprvé setkal v~roce 2008, uchvátilo mě natolik, že
jsem jeho zpracování bez nadsázky zasvětil život. Dva roky jsem digitalizoval
magnetofonové pásky se záznamy jeho promluv, pak jsem jedenáct let pracoval
v~rámci doktorského studia na MFF UK na jejich přepisu a zpřístupnění. Zpracovat
jeho dílo nejen z~hlediska informatického jakožto sadu dat, nýbrž z~hlediska
obsahu, bylo mým snem od momentu, kdy jsem si uvědomil, že v~rámci disertace
v~oboru komputační lingvistiky se musím více či méně držet mantinelů technického
zaměření pracoviště.

Rozsah Makoňova díla je takový, že celé je přečíst a poslechnout je na mnoho
let. Sám jsem se k~němu dostal nejprve přes nahrávky. Tato preference mi
zůstala. Poslechl jsem zatím přibližně třetinu Makoňových hovorů, zatímco knih
jsem přečetl jenom hrstku. Svoji disertační práci jsem též věnoval nahrávkám. Zatímco
knihy byly úsilím předchozích Makoňových příznivců vydávány, nahrávkám se
dosud nikdo systematicky nevěnoval. Proto tomuto aspektu Makoňovy tvorby zůstávám věren i
zde v~této diplomové práci.

Za její úkol jsem si vytyčil zodpovědět otázku \textit{,,Co vlastně říká Karel
Makoň?{}``} Soustředím se tedy především na záznamy Makoňových hovorů a knihy
zmiňuji jen okrajově. Samozřejmě by se dalo postupovat i zcela opačně a postavit
práci na knihách. Věřím tomu, že takové práce se v~dohlednu někdo ujme s~nemenším
zápalem, než jsem se já ujal zpracování nahrávek. Kéž k~tomu tato práce pomůže.

\section{Citace}

Jelikož se tato práce zabývá Makoňovými nahrávkami, často se odkazuji na
konkrétní pasáže. Nahrávky nejsou běžným materiálem k~citování a těžko na ně lze
uplatnit standardní citační praxi. Soubor digitalizovaných nahrávek je
%coby výsledek mojí předchozí snahy
přístupný ze dvou hlavních kanálů:
\begin{enumerate}
\item{
  Z~internetové stránky \href{http://radio.makon.cz/}{radio.makon.cz}, kde se nachází mnou vytvořená
  webová služba, která umožňuje vyhledávání a poslech se simultánním zobrazením
  přepisů, a
}
\item{
  z~repozitáře Lindat/Clariah, který je výsledkem společného snažení několika
  institucí včetně Univerzity Karlovy a jejímž záměrem je otevřený přístup
  k~jazykovým zdrojům: \href{https://lindat.cz/repository/xmlui/handle/11234/1-3422}{lindat.cz/repository/xmlui/handle/11234/1-3422}.
}
\end{enumerate}
Přístup z~webové aplikace je pohodlnější, ovšem u Lindat/Clariah je garantována
trvalost přístupu.

Citace z~Makoňova mluveného korpusu udávám ve formě párů \textit{identifikátor
nahrávky} - \textit{časová pozice}. Identifikátory vznikaly při digitalizaci tím
způsobem, že se opisovaly popisky na nosičích (hlavně kazetách a kotoučích). U
identifikátorů je garantováno, že neobsahují mezery, ale jinak jejich formát
nemá omezení. V~textu indentifikátor vždy uvádím pomocí monospacového písma, např.
\texttt{85-05A}. Pozice je od začátku nahrávky a to v~běžném formátu
,,\textit{minuty}:\textit{sekundy}``, popř.
,,\textit{hodiny}:\textit{minuty}:\textit{sekundy}``.
