\chapter{Teologie Karla Makoně}
\label{kap:teologie}

V~kapitole~\ref{kap:temata} jsem se na Makoňovy nahrávky díval jako na zavřenou
krabici neznámého obsahu a pokoušel jsem se do ní z~různých úhlů vrhnout světlo,
aby se ukázalo, co se v~ní nachází. V~této kapitole se chci nadále pokoušet
odpovědět na otázku, co Karel Makoň říká, ovšem z~jiného východiska. Tentokrát
mi nejde o to, pokrýt věrně celý obsah pomyslné krabice, nýbrž si vzít sadu
konkrétních otázek a zjistit, jak na ně Makoň odpovídá.

Za tyto otázky vezmu tradiční body systematicko-teologické nauky:
\begin{enumerate}
    \item{%
        Nauka o Bohu
        \begin{enumerate}
            \item{O Bohu jediném (boží podstata a vlastnosti)}
            \item{Boží trojjedinost}
        \end{enumerate}
    }
    \item{%
        Kosmologie, nauka o stvoření, o Bohu stvořiteli
        \begin{enumerate}
            \item{Bůh stvořitel}
            \item{Boží prozřetelnost: svět je řízen}
        \end{enumerate}
    }
    \item{%
        Antropologie: K~čemu je člověk, jaký má smysl a jak ho porušil
    }
    \item{%
        Christologie
        \begin{enumerate}
            \item{Christologie: o osobě, podstatě, přirozenostech}
            \item{%
                Soteriologie: o Kristově díle; o ospravedlnění v Kristu; o posvěcení
            }
        \end{enumerate}
    }
    \item{%
        Ekleziologie: o církvi, jaká církev je pravá, viditelná / neviditelná církev
        \begin{enumerate}
            \item{%
                Sakramentologie: nauka o svátostech; otázka a Božího slova a svátostí
            }
        \end{enumerate}
    }
    \item{Eschatologie}
\end{enumerate}

\section{O Bohu}

\textit{%
,,Teď bych chtěl říci: Tam hrozilo a hrozí dosud, že když takhle někdo si čte ten
Starý zákon i Nový, tak z Boha udělá bytost. Bůh nikdy bytostí nebyl. Takhle se
nám taky dodneška v~křesťanství Bůh definuje. Je to nejdokonalejší bytost, že
ano, všemohoucí, vševědoucí a tak dále. Kdyby tomu takhle bylo, tak by vůbec
nebylo možno se jinak spojit s~tím Bohem než bytostně. A já, který jsem nechápal
Boha také jinak než bytostně, třebas ne jako starce, to jedno, tak jsem
narazil podle starých Židů a jiných předpisů třebas indických na bytostný
způsob dosahování. Existenční já tomu říkám. Takže moje zkušenosti jsou také
z~tohoto oboru jako bytostné, jsou existenční. Ale jaké v~tom je nebezpečí? Že
totiž ten člověk, který si předělá Boha na bytost, nikdy nepochopí, Bůh tedy je
také stavem. Nebo považuje to za vedlejší, že je stavem. Že je stavem vědomí,
stavem lásky, stavem existence, stavem poznání.``
%My totiž poznáváme takhle lidsky
%tím způsobem, že když to vidíme, cítíme nebo smyslem. Smysly nám to předkládají a
%my to rozumně zpracujeme, tak teprve to považujeme za poznané. To ještě poznané
%vůbec není. Tedy to je jenom porozumění věci, ale my to považujeme za poznané.
%Kdežto kdybychom správně mysleli, jak se myslet má, tak bychom věděli, že abychom
%mohli něco tímto způsobem poznávat, je k~tomu třeba mít
%obecnou schopnost poznávací. Takže když už něco
%konkrétně poznávám, tak to je jenom používání té obecné schopnosti poznávací. A je to
%tedy poměr jako projevu k~neprojevenému. Že my obecný způsob nebo obecnou
%schopnost poznávací nepoznáváme. My víme, že poznáváme, teprve když něco
%konkrétního poznáváme, nebo abstraktního, to už je jedno. Prostě poznáme něco, co
%se nám předkládá k~poznání. ale myslíme si, že kdyby by se nám nic
%nepředkládalo, že by prostě potom nezbylo vůbec nic z~toho, co je v oblasti
%poznání, to je omyl.
}

\textit{
,,Co se myslí tím, že jdeme k~Bohu jako k ničemu jako k~,,nic``, že ten Duch je
nic? A my, co považujeme za něco, nic vlastně není a tamto nic, co považujeme
za nic, je fakticky všechno? No, něco na tom je. Já tomu nemohu odporovat, této
myšlence, ale já bych rád to ještě z~hlubšího hlediska vysvětlil. Ježíš Kristus
se o tom vyjádřil tak, že on je úhelným kamenem a že ho zavrhli a že potom
nemohou postavit budovu bez toho úhelného kamene, že se to všechno zřítí a že to
nemá základy a podobně. My jsme totiž postaveni svou existencí na tomhletom
,,nic``. To znamená na něčem, co bychom nemohli konkretizovat jako existenci. Pro
nás přeci Bůh neexistuje. To si musíme přiznat. Já aspoň jako beznaboh si to
můžu přiznat. Pro mě do těch sedmnácti let Bůh neexistoval, to jsem byl hotový
beznaboh. A svou silou jsem jím pořád. Jenom mocí Boží chápu to jinak. Ale ze
své schopnosti víry, kterou nemám, bych to nepochopil. Já vás obdivuju. No ale
jestliže tomu takhle je, že ten člověk jde k~něčemu, co neexistuje na žádné
úrovni, nýbrž co ty úrovně jenom řídí, a kdyby to existovalo na úrovni
kterékoliv, tak by to nemohlo řídit. Tak je to to pravé nic, ke kterému se jde.
Po každé úrovni si můžete kráčet, ať je to sebevyšší. Třetí nebe ať je to, tak
po ní můžete kráčet. Se tam procházím mezi... může mi tam být dobře. Ale pravé
Tao není cesta, po které se dá kráčet. To je, to je to pravé nic. Tam až
vstoupíte, do toho nic, tak si uvědomíte s~Lao'c-em, ,,Tao je při všem. Ale do
ničeho nezasahuje. A chceš-li se přiblížiti k tomu Tao, nerozlišuj mezi
příjemným a nepříjemným. Kdo rozlišuje mezi příjemným a nepříjemným, nemůže se
přiblížiti k~pravé moudrosti.`` No, ono se tomu lidově říká: ,,Slunce Boží svítí i na
špinavou kaluž,`` ne? Nerozlišuje. Nesvítí víc do té
čisté kaluže, než do té špinavé, že ano? A v~tom smyslu je to nic. Že je to
vrcholně milosrdné. Ale tak nerozlišeně milosrdné, že už to z~lidského hlediska
jako milosrdenství nemůžeme poznávat. To je tak totálně milující, že z~hlediska
lidského, z~hlediska člověka, který je zvyklý, že láska někam teče, konkrétně
nějakým korytem k~něčemu, k~něčemu se vzpíná, tato láska to není, takže my to
jako lásku poznávat nemůžeme. My i tu lásku Boží poznáváme, jakože neexistuje,
že to nic. A zrovna tak je to s~tím bytím a s~tím poznáním a se vším. Tak bych
to takhle chtěl rozpitvat, abyste viděli, co je to nic. To je všeobsažné.
Na pozadí všeho existující. Ale na \textbf{pozadí} všeho existující. To
znamená, když rozbijete atom na nejmenší ještě další částky a ty zase,
tak Boha nenajdete. Ten není v~něčem existujícím. Ten není
tím existujícím, lépe řečeno. Ale je na pozadí toho. On to všechno udržuje, on
to všechno udržuje při vědomí, při existenci, podle toho, na jaké úrovni se to
vyskytuje, ne? Tak k~tomuto nic jdeme, takže když potom se dostaneme do toho
nic, tak je to pro nás ohromující zkušenost, protože my víme: ,,Je to nic, ale
tím jenom proto, že je to nic, tak je to všecko.`` Já se nedivím, že někdo je
takový pantheista, řekne, tak Bůh je všechno. Je to nesprávný postoj, to já
uznávám, to je velice, velice zvrhlý postoj. On není všecko. Ale všecko je
ustavičně z~něho. Ale má-li být z~něčeho ustavičně něco, jenom z~toho má
pocházet, tak ono to nemůže být tím nebo oním, to nemůže být.% To by se, to by
%potom z toho něčeho potom vycházelo se a Z toho druhého, to druhé už by nemohlo,
%to je, má jinou povahu, z toho vycházet.
To musí být opravdu nic, aby z~toho
všechno mohlo vycházet. Já nevím, to asi nesrozumitelné. Je to srozumitelné? Je
to těžko srozumitelné, no ale je to tak. Protože Pán Bůh kdyby se měl starat o
věci, které řídí, takhle by dopadlo. Nemá ani ruce ani nohy, a tím, že všechno
řídí, tak to vypadá, jako by to neřídil. Jako by byl nic. Tím, že všechno
miluje, tak to vypadá jako by nic nemiloval, protože kdo miluje vraha i toho,
kdo je vražděn, tak prosím vás, co to je? Jaká je to láska? To je nelidská
láska. Jako by ta láska neexistovala. Jakto že tady toho vraha nezastavíš, jakto
že mu nezastavíš ruku, když on vraždí? Co je to za lásku?``
}

\textit{
,,Já bych to řekl asi nějakým přirovnáním: Bůh je bytost, promiňte, že takhle
budu mluvit, která... nebo jsoucnost, která je schopna nejvyšší možné lásky. Nikdo
není schopen takové lásky jako on, takže my snad můžeme tu hlásku přirovnat
k~lidem, kteří se mají vzájemně rádi. Já to udělám, ovšem samo sebou, že to bude nebe
a dudy ne?``
}

\textit{
,,Jak líčili ti proroci a ti zasvěcenci toho Boha? Jako krutého mstitele, který
mstí hříchy do třetího a čtvrtého pokolení Já nevím, jak to pokolení k~tomu
vůbec přijde. A krutý mstítel,
soudce a tak dále. Všechno si to můžete v~Starém zákoně přečíst. Ono stačí tahleta
charakteristika. Co to je za vtip, takhle líčit milosrdného Boha? To je úžasně
moudré.
Protože jestliže člověk nevidí, neví o tom, že když je v~této fázi, že je na úrovni
přírodní, kde platí oko za oko, zub za zub, kde silnější má právo a neprohřešuje
se tím,  zabíjet toho slabšího nebo toho nemocného, tak nevychází z~údivu a
myslí si,
že je Bohem opuštěn. Ne, ten bůh se jeví tomu králíku jako absolutně
neexistující,
nebo když existující, tak jako nemilosrdný soudce, protože když ho přepadne orel,
tak je odsouzen k~zániku a žádný pán Bůh mu nepomůže, ani prstem nehne. A na to
vás upozorňuju, že kdykoli upadneme do stavu přírodního člověka, to znamená na
tom světě jenom jíme, pijeme, spíme a máme se dobře jako zvíře, tak není pro nás
žadné milosrdenství Boží připraveno, je pro nás připravenn Bůh nemilosrdný,
soudce, který bude trestat naše hříchy do třetího a čtvrtého pokolení.``
}

\section{Trinitologie}

\textit{
,,Bohu nelze rozumět. A to je to, čím bych chtěl začít. Proč mu nelze rozumět?
Proč se s~ním
nelze setkat? Proč ho nelze vidět? To bych chtěl dopovědět proto, že existuje
Bůh,
který je vlastně trojjediný. To vědí Indové, to vědí Evropané, to vědí křesťani,
tedy všichni na světě. Staří národové. To věděli Sumerové. Co to je ta
trojjedinost? My si řekneme otec, duch a syn a tím to
vyřídíme. Nic jsme tím nevysvětlili. Naopak jsme zatemnili věc. Jsou to, řek bych
školsky, rozdělené funkce jednoho a téhož Boha. Funkce, čili úkoly, které na sebe
vzal nebo ustavičně bere jeden a tentýž Bůh Ten totiž má úkol stvořitelský,
kterýž ustavičně tvoří, a tentýž Bůh má úkol spasitelský, že to stvořené k~sobě
táhne. Teď si už jenom představte, jak je těžké porozumět jednání takového Boha,
který zároveň ustavičně něco od sebe z~domova posílá do světa od sebe pryč a zároveň
z~toho světa to táhne k~sobě. Jak rozeznat jednoho od druhého? Jak to oddělit?
Z~toho byli věřící
v~koncentráku úplně vedle. Všichni věřící, se kterými jsem se tam setkal.
Všichno říkali: ,jak se
na tohleto zvěrstvo může dívat pán Bůh? Jak mě může nechat takhle na holičkách?
Já
jsem přeci byl vždycky zbožný člověk a já tady takhle trpím bez pomoci a takový
zvíře gestapácký má na mě všechna práva a Bůh ani prstem nehne pro mě.`` A kdykoliv
jsem s~nimi souhlasil a takhle myslel, vždycky [jsem] dostal nějaký kopanec od
toho gestapáka. Zrovna! Přesně! To je řízení od pána Boha, opravdu, abych věděl, že
pán Bůh s~tím absolutně souhlasí, že takhle se o člověka nestará. Ale vedle
tohoto,
vedle těchto dvou principů Božích -- stvořitele a spasitele, existuje třetí jeho princip a
to jest, oni říkají křesťani ,potěšitele`, ale lépe bylo řečeno osvětitele, že
opravdu ten člověk vlivem toho Božího působení v~něm může být čím dál
osvícenější. Že nemusí slepě jít ani od toho Boha pryč, ani k~tomu Bohu
nazpátek,
nýbrž může plně s~plným vědomím vědět, co dělá. Velice nutné, aby tento třetí
princip tam taky existoval u toho Boha.``
}

\section{Antropologie}

\textit{
,,Aby včelí rod byl zachován, ona nemůže mít lidský rozum, ona nemůže mít mozek.
To je veliká vymoženost, že nemá mozek. Centrálně by nedokázala to, co dokáže
včela decentrálním rozmístěním čidel, které nahrazují mozek, provést. Takže
včela může bezvadným, naprosto neomylným způsobem letět za květem, který je
vzdálen, který necítí a za kterým ona letí s naprostou bezpečností a přistane na
tom květu, který jí ukázala ta včela, která je tam... která ji tam navedla,
ráno. A to je spo- způsob, kterým- do kterýho nikdy nedorosteme. Protože místo
toho máme mozek. A ten nám toto neskýtá. A protože nám skýtá něco jiného. Nám
skýtá touhu přerůst přes svůj druh. Já chci být lepší, dokonalejší, rozumnější,
moudřejší, než byli moji rodiče. A proto mládež první chybu, kterou dělá, že se
staví na zadní nohy a říká: "Moje... moji předkové byli hloupí proti mně." To je
první š- špatný pohled, chybný pohled na rodiče a na předcházející generaci.
Voni mají za sebou tohle, to je ten pochod kupředu už totiž. A proto se jim
tamto s- zdá být zaostalé. To, co oni právě zažívají, není to pravda. Voni si
prošli tou fází, kterou ta mládež teprve prochází, aby dospěli do nějaké úrovně,
za kterou už se beztak nemůže jít. A to voni neví, že voni taky dospějou do té
fáze, za kterou už nebudou moct jít. Neboť člověk má taky omezené možnosti. No a
tohleto prostě u těch nižších tvorů také existuje. Jenže ta první fáze toho
mládí, až na slony a některé vyspělé ty savce, probíhá velice rychle. Například
mládí u takové třebas mouchy, probíhá v půl dny, co... co my musíme dělat
osmnáct let, tak ona za půl dne je s tím hotova a je hotovou mouchou lítající a
umí lítat z toho automatismu, který v ní je. Čili my ten automatismus opravdu,
co- který v sobě máme také a máme ho daleko složitější, máme ho dok-
dokonalejší, nemáš- než má třebas husa nebo včela. Ale my ho držíme v šachu. My
mu nedovolujeme, aby se rozrůstal tam, kam se rozrůstá třebas v oblasti prácem-
včely. To znamená, toto nechci přírodovědecky tady dál rozvádět, to by vás
strašně unavovalo. Ale já zůstanu u člověka. A řšš- teď to už chápete, že já
jsem se musel vzdát třebas přírodovědy ve svých patnácti letech. Jak je to
dobře, že jsem se toho musel vzdát, protože já bych z této strany do tý
přírodovědy byl nikdy nevnikl. Dneska do tý přírodovědy vnikám způsobem, kterým
málokterý přírodovědec do toho vniká. A závidí mně, když se s ním setkám, žes...
mám tento přístup k věcem. No, ale nechci se tím chlubit, to jsem já nezavinil,
to zavinili lékaři. Tak, teď chci říci: Jestliže člověk je jediný z tvorů, který
touží po tom, aby přerostl svého druha, aby byl lepší. Nebo dokonce horší, což
je taky svým způsobem jiný. Aby byl jiný, než byl ten vedlejší tvor. To je neb-
aby nebyl stádový. Tak je to známka toho, že dorůstá do pravé individuality,
která od všech relativností je oproštěna. To je ta věčná individualita, ze které
všechno jsme vzali a do které se zase nazpátek vracíme. Od Boha přicházíme a k
Bohu se vracíme.``
}
