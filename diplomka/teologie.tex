\chapter{Teologie Karla Makoně}
\label{kap:teologie}

V~kapitole~\ref{kap:temata} jsem se na Makoňovy nahrávky díval jako na zavřenou
krabici neznámého obsahu a pokoušel jsem se do ní z~různých úhlů vrhnout světlo,
aby se ukázalo, co se v~ní nachází. V~této kapitole se chci nadále pokoušet
odpovědět na otázku, co Karel Makoň říká, ovšem z~jiného východiska. Tentokrát
mi nejde o to pokrýt věrně celý obsah pomyslné krabice, nýbrž z~ní vytáhnout
konkrétní vzorky: Vzít si sadu otázek a zjistit, jak na ně Makoň odpovídá.

Otázky sestavím podle tradičních bodů systematicko\-teologické nauky:
\begin{enumerate}
    \item{%
        Nauka o Bohu
        \begin{enumerate}
            \item{Bůh jediný}
            \item{Boží trojjedinost}
        \end{enumerate}
    }
    \item{%
        Kosmologie
        \begin{enumerate}
            \item{Svět}
            \item{Stvoření}
            \item{Boží řízení světa}
        \end{enumerate}
    }
    \item{Antropologie}
    \item{%
        Christologie
        \begin{enumerate}
            \item{Vlastní christologie}
            \item{Soteriologie}
        \end{enumerate}
    }
    \item{%
        Ekleziologie
        \begin{enumerate}
            \item{Vlastní ekleziologie}
            \item{Sakramentologie}
        \end{enumerate}
    }
    \item{Eschatologie}
\end{enumerate}

\section{Metodologie}

Ke každé otázce se pokusím najít pokud možno kompaktní pasáže v~rozsahu desítek
vět, které předkládají ucelený pohled na danou otázku. Výběr provádím jednak za
použití vlastního tematického indexu a jednak fulltextovým vyhledáváním
v~přepisech. Vlastní paměť jsem použil jen tehdy, když se mi do ní daná pasáž
vryla tak hluboko, že jsem schopen vybavit si, kde byla.

Nepokouším se zde ani o redukci subjektivity ani o reprodukovatelnost ani o
rovnoměrné pokrytí všech částí korpusu. Jediným mým cílem zde je podat odpověď
na otázky systematické teologie z~úst Karla Makoně. Pasáže tedy nemusejí
reprezentovat jeho pohled na věc v~celém jeho vývoji.

\section{O Bohu}
\label{teol:o-bohu}

Makoň vidí Boha nikoliv jako bytost, která v~nějakém historickém aktu stvořila
svět, nýbrž jako toho, od jehož bytí je naše bytí odvozeno. Podobně jako toho,
od jehož lásky je odvozena naše láska, od jehož klidu je odvozen náš klid a tak
dále. Boha přirovnává k~prameni řeky, jíž je náš život a náš svět, nebo
k~elektrárně, která napájí přijímač našeho vědomí. Bez Boha by nic neexistovalo.
Rozhodně se tak vymezuje vůči Deismu, neboť myšlence, že Bůh svět stvořil a pak
nechává být, jednoznačně oponuje.

Při hledání pasáží pojednávajících o Boží podstatě jsem si všiml zajímavého
jevu: Karel Makoň o Bohu hovoří prakticky neustále, ale téměř vždy v~tom smyslu,
že Bůh něco dělá a téměř nikdy, že Bůh něco nebo nějaký je. Tento atribut sdílí
se Starým zákonem.\cite{brueggemann2010old} Tento dojem jsem získal ze svých
záznamů v~tematickém indexu. Abych ho ověřil, vybral jsem ze záznamů věty, v nichž je
Bůh podmětem, a spočetl případy, kdy Bůh něco činí a kdy něco je. Rozdíl nakonec
není tak velký, pokud konstrukce typu \textit{Bůh jako XXX} považujeme za
synonymum konstrukce \textit{Bůh je XXX}. Nalezené přísudky jsou uvedeny
v~tabulce~\ref{tab:topicidx-god-verbs}.

\begin{table}[htpb]
\begin{center}
\begin{tabular}{|l l|l|}
\hline
přísudek činný & & přísudek se sponou \\
\hline
bere (2\texttimes{})	& spí	& je bezpřívlastkový	\\
chce	& trestá	& je desítka	\\
čeká	& tvoří (2\texttimes{})	& je bytost	\\
dává	& umrtvuje	& je činitel (4\texttimes{})	\\
dokáže	& ví	& je činnost	\\
jedná (3\texttimes{})	& vměšuje se	& je klid	\\
krmí	& vstoupí	& je lidský pojem	\\
miluje (4\texttimes{})	& vyvíjí se	& je mstitel	\\
nastoupí	& zabíjí	& je nic	\\
obětuje 	& zbavuje	& je osvětitel	\\
odívá 	& zjeví se	& je otec	\\
odvrhuje 	& zlobí se	& je soudce (2\texttimes{})	\\
poroučí (2\texttimes{})	&	& je skutečnost	\\
potřebuje 	&	& je spasitel	\\
předchází 	&	& je stav	\\
přeje si (3\texttimes{})	&	& je stvořitel	\\
přistupuje 	&	& je zdroj	\\
působí 	&	& je žárlivý	\\
rodí se 	&	& není přísný	\\
sestupuje 	&	& není síla	\\
smilovává se (2\texttimes{})	&	& není stvořitel	\\
\hline
celkem 44	&	& celkem 25	\\
\hline
\end{tabular}
\caption{Přísudky k~podmětu Bůh v~tematickém indexu}
\label{tab:topicidx-god-verbs}
\end{center}
\end{table}

Lingvistické vlastnosti mnou psaného indexu pro mě mohou být jistým vodítkem,
ale zajisté nejsou tím, co by mělo primárně vypovídat o Makoňově díle. K~tomu je
potřeba podívat se přímo na přepisy jeho přednášek. Když z~nich vyfiltruju
všechny výskyty slova \textit{Bůh}, spočtu, jak často se jaká slova vyskytují
hned za ním, a omezím se na slovesa, vidím těchto deset nejčastějších variant:

\begin{itemize}
\item{498\texttimes Bůh je}
\item{111\texttimes Bůh není}
\item{ 76\texttimes Bůh miluje}
\item{ 62\texttimes Bůh stvořil}
\item{ 62\texttimes Bůh má}
\item{ 55\texttimes Bůh chce}
\item{ 46\texttimes Bůh přeje}
\item{ 33\texttimes Bůh byl}
\item{ 30\texttimes Bůh nemůže}
\item{ 28\texttimes Bůh vzal}
\end{itemize}

To by napovídalo zcela opačnému tíhnutí k~výpovědím o tom, co Bůh je a není,
ovšem při bližším prozkoumání se ukazuje, že v mnoha případech použití slova
\textit{,,je``} jde o příslovečné určení \textit{(Bůh je tady)} nebo dokonce o
zájmeno \textit{(Bůh je stvořil)}. Také ve 330 případech za slovem \textit{Bůh}
následuje zvratná částice, která napovídá o činném přísudku. Stejně tak se může
v~mnoha případech jednat o pasivní věty \textit{(Bůh je poznáván)}.

Tuto analýzu by bylo lépe udělat za použití větného rozboru. Jeho pořízení
je však výpočetně náročné a pro automatické přepisy, kterých je většina, není příliš
spolehlivý. Větný rozbor se nicméně dá automaticky pořídit pomocí nástroje UDPipe.\cite{udpipe}\footnote{%
\href{https://lindat.mff.cuni.cz/services/udpipe/}{lindat.mff.cuni.cz/services/udpipe/}} V~každém případě tato rychlá
analýza ukazuje, že vyjádření o Bohu ve smyslu přisuzování mu nějakých
vlastností je v~korpusu relativně mnoho, navzdory prvotnímu dojmu. Přesto nebylo
zcela snadné nalézt charakteristické pasáže, kde by se téma Boží podstaty a Božích
vlastností uceleně pojednávalo. Zde je pár příkladů, které jsem našel:

\begin{enumerate}

\item{%
Nahrávka s~označením \texttt{77-05A-Praha}, od pozice 23:08.

\textit{%
,,Teď bych chtěl říci: Tam hrozilo a hrozí dosud, že když takhle někdo si čte ten
Starý zákon i Nový, tak z Boha udělá bytost. Bůh nikdy bytostí nebyl. Takhle se
nám taky dodneška v~křesťanství Bůh definuje. Je to nejdokonalejší bytost, že
ano, všemohoucí, vševědoucí a tak dále. Kdyby tomu takhle bylo, tak by vůbec
nebylo možno se jinak spojit s~tím Bohem než bytostně. A já, který jsem nechápal
Boha také jinak než bytostně, třebas ne jako starce, to [je] jedno, tak jsem
narazil podle starých Židů a jiných předpisů třebas [indických?] na bytostný
způsob dosahování. Existenční já tomu říkám. Takže moje zkušenosti jsou také
z~tohoto oboru jako bytostné, jsou existenční. Ale jaké v~tom je nebezpečí? Že
totiž ten člověk, který si předělá Boha na bytost, nikdy nepochopí, [že] Bůh tedy je
také stavem. Nebo považuje to za vedlejší, že je stavem. Že je stavem vědomí,
stavem lásky, stavem existence, stavem poznání.``
%My totiž poznáváme takhle lidsky
%tím způsobem, že když to vidíme, cítíme nebo smyslem. Smysly nám to předkládají a
%my to rozumně zpracujeme, tak teprve to považujeme za poznané. To ještě poznané
%vůbec není. Tedy to je jenom porozumění věci, ale my to považujeme za poznané.
%Kdežto kdybychom správně mysleli, jak se myslet má, tak bychom věděli, že abychom
%mohli něco tímto způsobem poznávat, je k~tomu třeba mít
%obecnou schopnost poznávací. Takže když už něco
%konkrétně poznávám, tak to je jenom používání té obecné schopnosti poznávací. A je to
%tedy poměr jako projevu k~neprojevenému. Že my obecný způsob nebo obecnou
%schopnost poznávací nepoznáváme. My víme, že poznáváme, teprve když něco
%konkrétního poznáváme, nebo abstraktního, to už je jedno. Prostě poznáme něco, co
%se nám předkládá k~poznání. ale myslíme si, že kdyby by se nám nic
%nepředkládalo, že by prostě potom nezbylo vůbec nic z~toho, co je v oblasti
%poznání, to je omyl.
}

Kritika kategorizace Boha do jakéhokoliv pojmu, ať už je to \textit{bytost} nebo
cokoliv jiného, je vcelku běžným jevem. Není však tolik běžné položit jako
komplement k~pojmu Boha coby bytosti pojem Boha coby stavu. Makoň patrně vychází
z~pojmu \textit{,,sat-čit-ánanda``}, ke kterému se často odkazuje, viz např.
nahrávku \texttt{83-20-K} od pozice 01:15:23: \textit{,,Jestliže Indové třebas
        říkají nebo definují Boha jako sat čit ananda, jako nejvyšší poznání,
        nejvyšší bytí, jako nejvyšší blaženost, tak všimněte si, jak do toho
        zasáhl Ježíš Kristus. Řekl: já jsem cesta, pravda a život.``}, popř.
        spis Hovory o mystice a cestě k~Bohu vůbec na straně
        26.\cite{KaMaHovory}

}

\item{%
Nahrávka s~označením \texttt{83-28-394-84-01A}, od pozice 00:32.

\textit{%
,,Co se myslí tím, že jdeme k~Bohu jako k~ničemu jako k~,nic`? Že ten Duch je
nic? A my, co považujeme za něco, nic vlastně není a tamto nic, co považujeme
za nic, je fakticky všechno? No, něco na tom je. Já tomu nemohu odporovat, této
myšlence, ale já bych rád to ještě z~hlubšího hlediska vysvětlil. Ježíš Kristus
se o tom vyjádřil tak, že on je úhelným kamenem a že ho zavrhli a že potom
nemohou postavit budovu bez toho úhelného kamene, že se to všechno zřítí a že to
nemá základy a podobně. My jsme totiž postaveni svou existencí na tomhletom
,nic`. To znamená na něčem, co bychom nemohli konkretizovat jako existenci. Pro
nás přeci Bůh neexistuje. To si musíme přiznat. Já aspoň jako beznaboh si to
můžu přiznat. Pro mě do těch sedmnácti let Bůh neexistoval, to jsem byl hotový
beznaboh. A svou silou jsem jím pořád. Jenom mocí Boží chápu to jinak. Ale ze
své schopnosti víry, kterou nemám, bych to nepochopil. Já vás obdivuju. No ale
jestliže tomu takhle je, že ten člověk jde k~něčemu, co neexistuje na žádné
úrovni, nýbrž co ty úrovně jenom řídí, a kdyby to existovalo na úrovni
kterékoliv, tak by to nemohlo řídit, tak je to to pravé nic, ke kterému se jde.
Po každé úrovni si můžete kráčet, ať je to sebevyšší. Třetí nebe ať je to, tak
po ní můžete kráčet. Se tam procházím mezi... může mi tam být dobře. Ale pravé
Tao není cesta, po které se dá kráčet. To je, to je to pravé nic. Tam až
vstoupíte, do toho nic, tak si uvědomíte s~Lao'c-em: ,Tao je při všem. Ale do
ničeho nezasahuje. A chceš-li se přiblížiti k tomu Tao, nerozlišuj mezi
příjemným a nepříjemným. Kdo rozlišuje mezi příjemným a nepříjemným, nemůže se
přiblížiti k~pravé moudrosti.` No, ono se tomu lidově říká: ,Slunce Boží svítí i na
špinavou kaluž,` ne? Nerozlišuje. Nesvítí víc do té
čisté kaluže, než do té špinavé, že ano? A v~tom smyslu je to nic. Že je to
vrcholně milosrdné. Ale tak nerozlišeně milosrdné, že už to z~lidského hlediska
jako milosrdenství nemůžeme poznávat. To je tak totálně milující, že z~hlediska
lidského, z~hlediska člověka, který je zvyklý, že láska někam teče, konkrétně
nějakým korytem k~něčemu, k~něčemu se vzpíná, tato láska to není, takže my to
jako lásku poznávat nemůžeme. My i tu lásku Boží poznáváme, jakože neexistuje,
že to nic. A zrovna tak je to s~tím bytím a s~tím poznáním a se vším. Tak bych
to takhle chtěl rozpitvat, abyste viděli, co je to nic. To je všeobsažné.
Na pozadí všeho existující. Ale na \textbf{pozadí} všeho existující. To
znamená, když rozbijete atom na nejmenší ještě další částky a ty zase,
tak Boha nenajdete. Ten není v~něčem existujícím. Ten není
tím existujícím, lépe řečeno. Ale je na pozadí toho. On to všechno udržuje, on
to všechno udržuje při vědomí, při existenci, podle toho, na jaké úrovni se to
vyskytuje, ne? Tak k~tomuto nic jdeme, takže když potom se dostaneme do toho
nic, tak je to pro nás ohromující zkušenost, protože my víme: ,Je to nic, ale
tím jenom proto, že je to nic, tak je to všecko.` Já se nedivím, že někdo je
takový pantheista, řekne, tak Bůh je všechno. Je to nesprávný postoj, to já
uznávám, to je velice, velice zvrhlý postoj. On není všecko. Ale všecko je
ustavičně z~něho. Ale má-li být z~něčeho ustavičně něco, jenom z~toho má
pocházet, tak ono to nemůže být tím nebo oním, to nemůže být. [...]
% To by se, to by
%potom z toho něčeho potom vycházelo se a Z toho druhého, to druhé už by nemohlo,
%to je, má jinou povahu, z toho vycházet.
To musí být opravdu nic, aby z~toho
všechno mohlo vycházet. Já nevím, [je] to asi nesrozumitelné. Je to srozumitelné? Je
to těžko srozumitelné, no ale je to tak. Protože Pán Bůh kdyby se měl starat o
věci, které řídí, takhle by dopadlo. Nemá ani ruce ani nohy, a tím, že všechno
řídí, tak to vypadá, jako by to neřídil. Jako by byl nic. Tím, že všechno
miluje, tak to vypadá jako by nic nemiloval, protože kdo miluje vraha i toho,
kdo je vražděn, tak prosím vás, co to je? Jaká je to láska? To je nelidská
láska. Jako by ta láska neexistovala. Jakto že tady toho vraha nezastavíš, jakto
že mu nezastavíš ruku, když on vraždí? Co je to za lásku?{}``
}

Krom užívání nestandardního slova ,,beznaboh`` je poutavý pojem Boha jako Ničeho. Krom zmíněné souvislosti s~taoismem,
konkrétně se spisem Tao te ťing\cite{lao1971tao} je zde
patrný vliv mistra Eckharta.\cite{10.2307/24355821} Je zde patrný i postoj ke
vztahu Boha a zla, jak jsme viděli v~kapitole~\ref{kap:temata} v~části o
modlitbě. Za povšimnutí stojí i vyhranění se proti pantheismu.

}

\item{%
Nahrávka s~označením \texttt{kotouc-D01-d}, od pozice 01:23:37.

\textit{%
,,Já bych to řekl asi nějakým přirovnáním: Bůh je bytost, promiňte, že takhle
budu mluvit, která... nebo jsoucnost, která je schopna nejvyšší možné lásky. Nikdo
není schopen takové lásky jako on, takže my snad můžeme tu lásku přirovnat
k~lidem, kteří se mají vzájemně rádi. Já to udělám, ovšem samo sebou, že to bude nebe
a dudy, ne?{}``
}

Příklad toho, že Makoň používá modely, které sám považuje za nesprávné, když to
umožní vyložit zamýšlenou stať. Všechna přirovnání o Bohu považuje za nesprávná
a uchyluje se k~nim jako k~nutnému zlu v~konkrétní fázi snahy o pochopení
pravdy.

}

\item{%
Nahrávka s~označením \texttt{81-06B}, od pozice 03:42.

\textit{%
,,Jak líčili ti proroci a ti zasvěcenci toho Boha? Jako krutého mstitele, který
mstí hříchy do třetího a čtvrtého pokolení. Já nevím, jak to pokolení k~tomu
vůbec přijde. A krutý mstitel,
soudce a tak dále. Všechno si to můžete v~Starém zákoně přečíst. Ono stačí tahleta
charakteristika. Co to je za vtip, takhle líčit milosrdného Boha? To je úžasně
moudré.
Protože jestliže člověk nevidí, neví o tom, že když je v~této fázi, že je na úrovni
přírodní, kde platí oko za oko, zub za zub, kde silnější má právo a neprohřešuje
se tím, zabíjet toho slabšího nebo toho nemocného, tak nevychází z~údivu a
myslí si,
že je Bohem opuštěn. Ne, ten Bůh se jeví tomu králíku jako absolutně
neexistující,
nebo když existující, tak jako nemilosrdný soudce, protože když ho přepadne orel,
tak je odsouzen k~zániku a žádný pán Bůh mu nepomůže, ani prstem nehne. A na to
vás upozorňuju, že kdykoli upadneme do stavu přírodního člověka, to znamená na
tom světě jenom jíme, pijeme, spíme a máme se dobře jako zvíře, tak není pro nás
žadné milosrdenství Boží připraveno, je pro nás připraven Bůh nemilosrdný,
soudce, který bude trestat naše hříchy do třetího a čtvrtého pokolení.``
}

Zde se objevuje důležitý a pro Makoně charakteristický koncept různých úrovní
vědomí, které ovlivňují realitu, v~níž se jedinec nachází. Podle úrovně vědomí
platí nejen různé zákonitosti, ale i se jinak projevuje Bůh. Tím také vysvětluje
zdánlivý rozpor mezi Boží povahou ve Starém a Novém zákoně. Ve výpovědi se dá
rozpoznat i téma teodiceje.

}

\end{enumerate}

\subsubsection*{Interpretace}

Jestliže Bůh je ten, z~jehož bytí je naše bytí odvozeno, pak je nesmyslné
uvažovat o Boží existenci v~témže smyslu jako u jsoucen ve světě. Harmonuje to
s~Tillichovou formulací, že není nic, co by bylo Bůh.\cite{tillich1975systematic}
Bůh sám sebe představuje jako ten, který jest (Gn 3, 14). Bytí je Boží vlastnost. Proto my
tím, že jsme, jsme obrazy Božími. Zároveň existence proudu bytí, vědomí a lásky,
který jde jednostranně od Boha k~nám, a který Makoň ztotožňuje s~tvořitelskou
činností Boží, nám umožňuje jít proti jeho proudu, a tak Boha poznávat. Bůh je
tak zároveň zcela nepostižitelný, neobjektivizovatelný, ale přitom není jen
chimérou, ke které bychom se mohli vztahovat jen odevzdanou vírou. Naopak,
striktní ověřování je právě na cestě k~Bohu veledůležitým prvkem, obzvlášť pro
lidi zasažené odkazem osvícenství a vědeckým uvažováním.

Bůh se ukazuje každému jinak podle toho, na jaké úrovni zákonitostí se jedinec
nalézá. To také dokonale vysvětluje zdánlivý rozpor mezi Bohem starozákonním --
přísným a trestajícím, a Bohem novozákonním -- milujícím Otcem. Od doby
konceptualizace starozákonního Boha do doby Ježíšovy podstoupil židovský národ
vývoj a Bůh se mu tak mohl začít jevit odlišným způsobem, ačkoliv sám v~sobě
zůstává neměnný, jak to postuluje Platón.\footnote{Ústava, 381b}

Veškeré dějové výroky o Bohu, jako že se Bůh hněvá nebo že něco činí, hovoří o
tom, jak se Bůh člověku jeví v~jeho mysli, která operuje v~dějích. Slova samotná
štěpí realitu na postupně se odvíjející proces, proto slovní svědectví o Bohu už
nutně podávají dějový obraz o něčem, v~čem je vše věčně obsaženo. Bůh tedy
biblické svědectví přesahuje, ale lepší slovní svědectví než biblické nemusí
nutně existovat, a Makoň trvá na Ježíšově výroku, že nepomine jediná čárka
Zákona (Lk 16, 17).

\subsection{Trinitologie}

Karel Makoň uznává objektivitu Boží trojjedinosti a mnohokrát se ve výkladu
odvolává na vztah Boha stvořitele a Boha spasitele. Boha osvětitele zmiňuje
řidčeji. Píšu je záměrně s~malými
písmeny, protože Makoň je pojímá jako mody Božího působení, takže reálně tvoří,
reálně působí spásu a reálně udělují poznání, takže se tak nejen jmenují, ale to označení mají ze své
podstaty. Osobní rozměr jim explicitně neupírá, aspoň ne konzistentně, ale
rozhodně je na osoby neredukuje. Z~Boží trojjedinosti vyvozuje praktické důsledky.
Říká-li, že naše existence je odvozena od existence Boží, pak v~trinitologických
pojednáních odvozuje i naše poznání od poznání Ducha svatého: On je jediný
poznávající v~nás.

\begin{enumerate}

\item{%
Nahrávka s~označením \texttt{80-02A}, od pozice 29:15.

\textit{%
,,Bohu nelze rozumět. A to je to, čím bych chtěl začít. Proč mu nelze rozumět?
Proč se s~ním
nelze setkat? Proč ho nelze vidět? To bych chtěl dopovědět proto, že existuje
Bůh,
který je vlastně trojjediný. To vědí Indové, to vědí Evropané, to vědí křesťani,
tedy všichni na světě. Staří národové. To věděli Sumerové. Co to je ta
trojjedinost? My si řekneme Otec, Duch a Syn a tím to
vyřídíme. Nic jsme tím nevysvětlili. Naopak jsme zatemnili věc. Jsou to, řek bych
školsky, rozdělené funkce jednoho a téhož Boha. Funkce, čili úkoly, které na sebe
vzal nebo ustavičně bere jeden a tentýž Bůh. Ten totiž má úkol stvořitelský,
kterýž ustavičně tvoří, a tentýž Bůh má úkol spasitelský, že to stvořené k~sobě
táhne. Teď si už jenom představte, jak je těžké porozumět jednání takového Boha,
který zároveň ustavičně něco od sebe z~domova posílá do světa od sebe pryč a zároveň
z~toho světa to táhne k~sobě. Jak rozeznat jednoho od druhého? Jak to oddělit?
Z~toho byli věřící
v~koncentráku úplně vedle. Všichni věřící, se kterými jsem se tam setkal.
Všichni říkali: ,Jak se
na tohleto zvěrstvo může dívat pán Bůh? Jak mě může nechat takhle na holičkách?
Já
jsem přeci byl vždycky zbožný člověk a já tady takhle trpím bez pomoci a takový
zvíře gestapácký má na mě všechna práva a Bůh ani prstem nehne pro mě.` A kdykoliv
jsem s~nimi souhlasil a takhle myslel, vždycky [jsem] dostal nějaký kopanec od
toho gestapáka. Zrovna! Přesně! To je řízení od pána Boha, opravdu, abych věděl, že
pán Bůh s~tím absolutně souhlasí, že takhle se o člověka nestará. Ale vedle
tohoto,
vedle těchto dvou principů Božích -- stvořitele a spasitele, existuje třetí jeho princip a
to jest, oni říkají křesťani ,potěšitele`, ale lépe bylo řečeno osvětitele, že
opravdu ten člověk vlivem toho Božího působení v~něm může být čím dál
osvícenější. Že nemusí slepě jít ani od toho Boha pryč, ani k~tomu Bohu
nazpátek,
nýbrž může plně s~plným vědomím vědět, co dělá. Velice nutné, aby tento třetí
princip tam taky existoval u toho Boha.``
}

Boží trojjedinost je zde podána nad rámec křesťanského pojmu jako něco, co lze
poznávat a bylo historicky poznáváno i bez osoby Ježíše Krista. Makoň potvrzuje
objektivitu Boží trojjedinosti. Nepojímá ji však jako trojici osob, ale trojici
        funkcí. Mohl by tak být zařazen jako modalista, ovšem oproti
        Sabelliovi\cite{lyman2013sabellius} má
pochopitelně výklad značně odlišný, proto je otázka, zda odsouzení antického
        modalismu se vztahuje i na Makoňovo pojetí. Pro podrobnější rozbor modalismu v~antice i
v~současnosti viz např. Jowers (2003).\cite{jowers2003reproach}

Tato pasáž krom Boží trojjedinosti silně promlouvá i k~tématu Božího řízení
        světa.
}

\item{%
Nahrávka s~označením \texttt{82-22}, od pozice 55:16.

\textit{%
,,Otec, Syn a Duch svatý. My si totiž nedovedeme vůbec představit, jak je
to možný, že na jedné straně pán Bůh něco tvoří a ustavičně to tvoří -- kdyby to
ustavičně netvořil, tak my nejsme vůbec na světě, jo? Musí pořád do toho svoji
bytostnou, podstatnou část vkládat, aby mohl existovat tady kámen, čas a prostor
jako celek a taky tedy člověk. Čili to je proud, který míří od něho do
časoprostoru. Kdybyste se, jako normální člověk je, spojili jenom s~tímto
proudem,
to znamená žili si jenom pro sebe, tedy z~toho, co od něho dostáváte, aniž si
uvědomíte, že z~toho žijete, tak promrháte všecko, ne? Protože byste se nechytli
toho druhého proudu, kterým je spasitel a to je tentýž Bůh, ne nějaký druhý.
Jenomže zároveň s~touto silou odstředivou působí síla dostředivá. Vzájemně se drží
v~šachu, ano? Protože když máte odstředivou sílu deset a dostředivou deset,
deset bez desíti je nula a to je Bůh. A jsou tam obě dvě tyto síly, rozumíte mi
dobře? I ta desítka kladná i ta desítka záporná a přesto je to nula, že jo? To je
to Nic přirozenosti Boží. Nesmírně veliká moc tvůrčí, nesmírně veliká moc
spasitelská. A co je to, to třetí? Ten Bůh utěšitel nebo osvětitel, Duch svatý?
To je zase něco, a o to se hádali katolíci s~pravoslavnými, jestli Duch svatý působí
jenom přes Ježíše nebo taky... jestli taky přes Ježíše a nejenom z Otce.
Pravoslavní říkají, že jenom z~Otce pochází, a katolíci, že pochází také ze
Syna,
vychází taky ze Syna. A roku tisíc našeho letopočtu se rozešli a vznikla
Pravoslavná církev z~toho, to byly ještě důvody nějaký politický, ale to nevadí.
Zkrátka pro takovou maličkost se rozešli a byl to zbytečný
rozchod. Dneska by se to už mělo uspořádat, protože ona je to síla, která
působí skrze obojí. Je to jenom další funkce Boží, která se nedá vysvětlit plus
deset a mínus deset, nýbrž by se musela vysvětlit úplně jiným obrazem. Já ho
tady nebudu předvádět, protože to by byl obraz na čtvrt hodiny, ale je to síla
poznání,
se kterou si vůbec křesťané zatím nevěděli rady, a to z~toho důvodu, že je třeba
napřed tu kladnou desítku s~tou zápornou desítkou vyrovnat a teprve potom může
nastoupit ten Duch svatý.``
}

Velice neobvyklý, ba kontroverzní výklad, kde Bůh stvořitel a Bůh spasitel
působí protikladně jako síly jdoucí jedna proti druhé a doslova ,,se vzájemně
drží v~šachu``. Je zde však patrné povědomí toho, že tento obraz je opět jen
nepřesným modelem a nestačí ani na výklad role Ducha svatého.

}

\item{%
Nahrávka s~označením \texttt{79-01}, od pozice 01:06:10.

% 79-01 01:06:10
\textit{%
,,Tady nemějte starost, že by na tomto předělu vám nekynula žádná pomoc. Ta pomoc
je vyššího druhu a není závislá už od vedení událostmi nebo od vedení stvořením,
stvořeným nebo naším já, nýbrž je závislá jenom čistě na tom Bohu, který je jiným
Bohem, než který tvoří, a o kterém se vyjádřil Eckhart, že nemá nic společného
s~Bohem, který je stvořitelem, ano? Je to tentýž Bůh, ale je to jiná jeho tvář, která
nemá nic společného s~tím Bohem, který je stvořitelem.``
}

Pasáž, která rozvíjí ty předchozí. Oproti nemilosrdnému Bohu, který
,,nehne prstem, aby pomohl``, je zde ujištění o Boží pomoci. Rozdíl je
opět dán jinou vývojovou fází, jinou úrovní jedince. Jestliže v~předchozí pasáži
se hovořilo o protikladnosti Božích osob, zde spolu ,,nemají nic společného``.
Těžko říci, která varianta více otřásá zavedeným vnímáním, každopádně se zdá, že
to je záměrem.
}

\item{
Nahrávka s~označením \texttt{76-03A-Kaly-5}, od pozice 23:22.

% 76-03A-Kaly-5 23:22
\textit{%
,,My totiž pořád slavíme trojici Boží: Bůh Otec, Bůh Syn a Bůh Duch svatý, ne?
Ale to synovství Otcovo nemůže být ve stavu určité míry oddělenosti, jak se to
jevilo po ty tři roky, kdy on poslouchal, ale byl menší než Otec. Potom až se
znovu stane Synem, až se mu vrátí synovství na jiné úrovni, tak už bude roven
tomu Otci. On bude mít právo poslat Ducha svatého. A takhle bude vládnout celým
Božstvím, celým obsahem jeho moci. A ta moc největší spočívá v~tom vedení Duchem
svatým všeho. Ten Duch svatý ve všem vládne, ve všem je pravou živou podstatou.
Ten Duch svatý totiž je podstatou všech těch samovolných sil. Ten se ocitá na
jejich podkladě. Ale zatímco ty síly samy o sobě jsou trpné, On je přitom
aktivní, víte? To je ta stránka aktivní těch samovolných sil.``
}

Poslední trinitologická pasáž přináší také významné christologické sdělení o
tom, že Ježíš požíval synovství na různých stupních.

}

\end{enumerate}

K~Makoňovým trinitologickým výpovědím chci jen dodat, že pozoruji zvláštní jev
týkající se kvantity statí o Duchu svatém. Citované pasáže budí dojem, že Duchu
svatému je věnováno mnoho, ne-li nejvíce pozornosti. Tak jest ale pouze tam, kde
se diskutuje Trojice jako taková. Jinak Makoň hovoří spíše o Otci a Synu jako o
Stvořiteli a Spasiteli, nechávaje roli Ducha svatého bez komentáře.

\subsubsection*{Interpretace}

Jediný Bůh tvoří, tedy žene vše od sebe do světa, do dualismu, do dobra a zla.
Zároveň týž jediný Bůh působí spásu, tedy táhne vše ze světa k~sobě. Tento pojem
má důsledek pro chápání vztahu mezi Boží vůlí a událostmi ve světě. Není potřeba
připisovat zlo ďáblovi, nýbrž ve světě protichůdným působením téhož Boha. Co se
jeví jako zlo, může být jen pobídkou k~odstupu od stvořitelského proudu a
nastoupení do spasitelského. V~praxi to znamená přijetí událostí, ať jsou
jakékoliv, ale vždy jde o vnitřní přijetí. Makoň nevede lidi k~tomu, aby se
nechali využívat a vláčet. Zda máme vnějšně proti událostem bojovat, to nám
umožňuje rozlišit Duch svatý nejprve světlem rozumu a později čím dál
dokonalejšími formami vedení.

Duch svatý je jediný zdroj poznání, je tedy vždy přítomen, kdekoliv člověk
pochopí sebetriviálnější fakt. Makoň se směje
církevním synodám, kde houževnatě prosí o Ducha
svatého.\rref{prevzate-kotouc-88-07-Tereza}{30:57}{1857} Říká,
že nějaká slovní prosba je naprosto zbytečná, musí se vystihnout podmínky pro
jeho přijetí. A hlavní podmínkou je vyprázdnění se od sebe samého.
%Na to je výborné dělat něco s takovým nasazením, že u toho člověk na sebe zapomene.

\section{Kosmologie}

Pro Makoně typický pohled na věci z~hlediska úrovní prostupuje i do kosmologie.
Spojuje vlastní zkušenost, proudy mysticismu 20. století, snad se společným
základem jako má antroposofie,\cite{Bezdek2009thesis} a indickou nauku o
čakrách,\cite{atwell2013osi} snad s~příměsí kabbalismu, jak to syntetizuje i
Kentonová.\cite{kenton2015kabbalistic} Svět se podle Makoně dá vnímat ze sedmi
úrovní, odpovídajících sedmi čakrám. Tomu podle něho odpovídá i sedm komnat v~Hradu
nitra\cite{teresa1588castillo} sv. Terezie z~Avily.

\begin{enumerate}

\item{
Nahrávka s~označením \texttt{kotouc-E01-c}, od pozice 00:00.

\textit{%
,,Mája, to je pojem nejméně chápaný u nás na západě, protože my jej chápeme jako
klam, a to je asi tak chybné, [...] Jestliže tady existují nějaké sféry
existenční, jako třebas hmotná, elementální, astrální, rajská, andělská,
projevený Bůh, jestliže takovéto sféry existují, tak všechny vyvěrají z~absolutna.
Z~absolutního božství. A to absolutno, kdybychom to chtěli nějak definovat...
Ne, definovat těžko, ale si představit, tak bychom ho definovali jako nádrž
všeho, co z~něho vyvěrá, všeho co z~něho pořád vyvěrá. Čili tam je v~absolutním
stavu to, co my vidíme v~relativním, v~poměrném stavu, protože to vyvřelo. A
teď ale... jestliže jsme v~proudu, který vyvěrá jinou dírou nebo jiným otvorem,
jiným směrem než jiný proud, tak my jsme schopni vidět skutečnost jenom jako
součást toho proudu. To znamená: Jsme-li v~hmotné oblasti, jsme schopni [...] v~jeho
díle, který zanechal ve hmotě. Ano? Velikost jeho v~jeho tvorech nebo v~jeho
stvoření, jinak ne. A přičemž mám na mysli i takový široký pojem jako světlo
rozumu, to je taky stvořené světlo, ano? I tak ho můžeme chápat do určité míry.
Kdybych žil na úrovni astrální, tak bych ho chápal z~této astrální úrovně.
Elementární a tak dále, všude bych ho chápal jiným způsobem. Takže já když se
dneska ocitnu třeba ve snu v~jiném těle, takzvaném astrálním, jak se tomu odborně
říká, tak já se s~ním ztotožním, a ztotožním se s~celým tím viděním snovým a
považu to za skutečnost, co tam zažívám. Když se probudím, považuju za
skutečnost to, co zažívám zde. Kdybych se probudil do jiného stavu, třebas
elementárního, považoval bych za skutečnost, co tam zažívám. A zrovna tak
kdybych se probudil do
rajského stavu a tak dále. A je to tedy nepravda, co já zažívám? Je to
klam, co tam zažívám? Ne, je to částečka nebo jeden proud skutečnosti.
Takhle se musí mája vysvětlovat. Jeden proud skutečnosti. A
protože je to jenom jedna malá část skutečnosti, proto mohl Ježíš Kristus o této
máje říci: ,Nechať mrtví pochovávají mrtvé.` To znamená: Mohl považovat jednu
část za hotovou smrt vedle toho celku, se kterým on se stýkal.``
}

}

\end{enumerate}

\subsection{Stvoření}

U stvoření stejně jako u spásy klade Makoň vytrvale největší důraz na jejich
ustavičnost. Ustavičné tvoření Makoň vykládá z~vlastního pojetí, neodkazuje se
na svoje předchůdce. Zda je neznal, mi není známo, ale lze se toho dohadovat.
Každopádně \textit{creatio continua} je pojem živě diskutovaný s~doloženým
historickým rámcem jak dnes\cite{congdon2010creatio}\cite{Salim2022Creatio} tak
v~Makoňově době.\cite{marcus1957typen} Nijak to však neznamená, že by Makoňův
pohled byl shodný s~ostatními proponenty \textit{creationis continuae}.

\begin{enumerate}

\item{
Nahrávka s~označením \texttt{82-02}, od pozice 42:07.

% 82-02 42:07
\textit{%
,,Můžeme se opřít o vědecké bádání: archeologické bádání, antropologické bádání,
o všechny možný druhy věd se můžeme opřít a víme tohleto: Že na tom světě se
vyvíjeli napřed jednodušší tvorové, ale vždycky současně zvířecí říše
s~rostlinnou. Tato symbióza mezi zvířetem a rostlinou existuje od začátku toho
života na tomto světě, víte? Takže nedegradujme rostlinu na úroveň pod zvíře. To
je směr jdoucí jinam, než kam jde zvířecí život. Tyto dva musí být vedle sebe se
specializovaným úkolem, aby to vůbec všechno mohlo fungovat. Bez zvířat by
nebyla rostlina a bez rostliny by nebylo zvíře, tak to asi zjednodušeně je. Já
to dělám všechno velice zjednodušeně, školsky. A bez těch dvou by nebyl zase
člověk, To je taková nadstavba nad tímhletím. Ovšem člověk je na straně těch
zvířat, nikoli na straně rostlin. A tak teďka především bych byl rád, kdybyste
věděli, že do této vývojové teorie nemá vůbec co mluvit povídání Geneze ze
Starého zákona, že za šest dní Bůh něco stvořil a sedmý den odpočíval. Vůbec o
tom se tam nemluví. I když je tam jasně řečeno: ,Napřed bylo to, potom bylo
tamto.` To vůbec nemá souvislost s~dnešním pochopením vědy a nic to nemluví
proti dnešnímu pochopení vědy. Rádo se to staví jedno proti druhému, ale tam
je to líčení stavu a tady je to líčení situace, že? Dnešní věda si zkoumá
situaci, vývoj situační. Kdežto tam se mluví o stavové situaci, stavovém stavu.
Já bych tamto řekl takhle: Jestliže by v~člověku neexistovalo na té úrovni, na
které on je, šest dní tvůrčích Božích a sedmý den odpočinku Božího, tak on jako
člověk by nemohl ani prstem hnout směrem ze své úrovně. On by nemohl zdvihnout
hlavu k~Bohu.``
}

Makoň obvykle vykládá biblický stvořitelský mýtus (Gn 1) jako záležitost týkající se
vnitřního vývoje člověka. Nesetkal jsem se s~tím, že by věc vztahoval ke
vzniku vesmíru jako takového. Zdůrazňuje však dokonalou pravdivost
biblického sdělení, je-li správně chápáno, a to nejen u stvořitelského
mýtu.

}

% kotouc-D01-d 01:00:15
%\textit{%
%,,My si představujeme, že Ježíš Kristus to jednou odnes na tom kříži a že nás
%tím spasil, a takhle to je klamný dojem. To jsou všechno symboly toho,
%co se věčně děje. Zrovnatak dejme tomu stvoření světa je
%symbol toho, co se věčně děje, ustavičně probíhá v~tom vesmíru. První den stvoření
%světa, druhý, třetí, čtvrtý, pátý, šestý i to odpočívání. Všechno tam souběžně
%probíhá.
%Ono se to nedá znázornit nijak jinak než jako děj, ale ono to vlastně je
%permanentní situace, ustavičně se opakující. Takže ten Ježíš je ustavičně
%například křižován, ale také ustavičně vstává z~mrtvých. Teď je otázka, jestil my
%se s~ním dáme ukřižovat jenom a jestli my s~ním vstaneme z~mrtvých a nebo
%ne.``
%}

\item{
Nahrávka s~označením \texttt{kotouc-T01-metoda-d}, od pozice 01:24:55.

% kotouc-T01-metoda-d 01:24:55
\textit{%
,,Ježíš v~té době svého mládí trošičku viděl víc do té činnosti svého Otce než
my. Bylo to způsobeno tím, že prošel narozením v~Betlémě, což nebylo obyčejné
narození. To je symbol už velice vznešeného znovuzrození, velice vznešeného
znovuzrození a my jsme tím neprošli, tak proto takhle nevidíme do dílny Boží. A
kdybych tedy, pokud si to dovedu vůbec představit, měl říci, jakým způsobem asi
Ježíš Kristus viděl do tý dílny Boží, a proto viděl, že musí být v~tom, co je jeho
Otce, tak bych to řekl asi takhle: On už dobře znal, že to, co je ve Starém zákoně
pověděno o šesti dnech práce Hospodina a sedmém dnu odpočinku, že to je symbol
toho, co se ustavičně děje. Co ustavičně probíhá. Že probíhá ustavičně činnost Boží
a ustavičně sedmý den odpočinku. Kdo tohleto zažil, tak ví, že stvoření potom má
zcela jinou povahu, než jak my podle symboliky, kterou si nedovedeme zatím
vysvětlit, mu
přičítáme. Že totiž to není nějaká akce, po ní zase následuje další akce, ta tvůrčí
činnost Boží, jako je to tam líčeno, nýbrž že je to soustavná činnost na různých
úrovních, jak je to tam líčeno, neboť nedokážeme to jinak vyjádřit než symbolem
času a prostoru a pak to musím líčit jedno za druhým, když to jednou dáme do času
a do prostoru. Ale když to obrátíme, když z~toho symbolu přejdeme do té
bezčasovosti,
ve které se to děje, ze které to prýští, tak tam se zároveň rodí svět a zároveň se
ničí, zároveň je Bůh v~klidu. A kdyby tomu tak nebylo, tak my nikdy nemůžeme být
činní, protože on je jediný činitel. Jak bychom my mohli být činni, kdyby on
činitelem nebyl? Naše činnost je odvozena od činnosti Boží. Uboze odvozena od
činnosti Boží.
On je jediný činitel. Ale bychom nikdy nemohli být v~klidu, kdyby on v~klidu
nebyl.
Nikdy ne. A protože on je neustále v~klidu, a náš klid je odvozen z~jeho klidu,
tak také můžeme v~klidu být.``
}

Zde je doložen Makoňův ustavičný důraz na ustavičnost. Za povšimnutí též stojí,
že stvoření světa za šest a jeden den lze zažít. Je zde i zajímavá
christologická odbočka. Díky Bohu usmiřování vědeckého a biblického pojetí
stvoření světa není momentálně v~naší společnosti výrazně třeba, ale leckde
ještě ano.\cite{carlin2017cultures}

}

\end{enumerate}

\subsubsection*{Interpretace}

Kosmologický koncept různých úrovní reality, jež se rozkrývají podle otevření
čakramů, má hluboký etický dopad pro současnou společnost, v~níž vládne princip
přizpůsobování se mase. Demokratické principy se zneužívají -- v~rámci svobody
slova může podnikatel zbohatlý díky perverzním dotacím ovládnout televize,
rádia, noviny, časopisy,
billboardy\footnote{\href{https://www.forum24.cz/dalsi-svedectvi-z-mf-dnes-babis-vola-novinarum-a-rika-co-maji-delat-plesl-upravuje-texty/}{forum24.cz/dalsi-svedectvi-z-mf-dnes-babis-vola-novinarum-a-rika-co-maji-delat-plesl-upravuje-texty}}
a umnou manipulací z~dílny externí
agentury\footnote{\href{https://www.lidovky.cz/domov/kdo-je-babisem-polibeny-genius-exnovinar-a-hvezda-marketingu-marek-prchal.A171022\_143055\_ln\_domov\_sk}{lidovky.cz/domov/.A171022\_143055\_ln\_domov\_sk}}
si koupit hlasy voličů. V~demokracii mají vládnout sami lidé tak, aby je nikdo
nemohl zneužívat ve svůj sobecký prospěch, ale nemají jak rozsoudit, co je
správné a co ne. A jak by to mohli umět rozsoudit, když k nikomu lepšímu
nevzhlížejí, kdo by je vzdělal a kultivoval? Jestli se na něčem prakticky
všichni shodnou, pak na tom, že ti, kdo jsou u moci, jsou lotři, zloději a
lháři.\footnote{\href{http://cepin.cz/cze/clanek.php?ID=1206}{cepin.cz/cze/clanek.php?ID=1206}}
Jak bychom k~nim mohli vzhlížet? Informací máme tolik, že je nemožné se v~nich
orientovat\footnote{\href{https://www.wired.com/2017/04/forcing-ads-captive-audience-attention-theft-crime/}{wired.com/2017/04/forcing-ads-captive-audience-attention-theft-crime/}} a tak si každý sleduje to, co ho utvrzuje v~jeho pocitech a
názorech, a nejlákavější z~nich jsou ty nejnižší. Média se vůbec neskrývají
s~tím, že jejich cílem je maximální sledovanost, a tak nenabízejí to, co člověka
pozvedne a co ho donutí se rozvíjet, ale co ho pobaví.\cite{achtenhagen2014challenges} Systematicky jde obsah
sdělovacích prostředků na úroveň té nejnižší masy a tím strhává i ty, kdo mají
potenciál k~růstu, místo aby pozvedávaly všechny na vyšší úroveň. Největší
úspěch mají ti, kdo křičí nejhlasitěji ty nejhrubší urážky, a tak vyvolávají
v~lidech plných zlosti a strachu pocit, že mají pravdu, a pocit zástupného
uspokojení z~vybití. Jdeme pořád níž a níž a intenzivněji a intenzivněji do
ještě hrubších obrazů sexu a násilí a obojího najednou.

Makoňův obraz světa, který existuje na hierarchicky uspořádaných úrovních
vědomí, a především pojem lidí, kteří do různých úrovní vstupují, nabourává
stále hlouběji zakořeňující individualismus a rovnostářství. Rovnostářství ne
v~tom smyslu, že jsme si všichni před Bohem a v~lidských právech rovni, toto
rovnostářství Makoňův pojem světa v~úrovních nenabourává, nýbrž rovnostářství
v~tom smyslu, že můj názor má stejnou váhu jako názor kohokoliv jiného.
Z~Makoňova pojetí světa plyne, že lidé s~vyšší úrovní vědomí, nebo jak někdy
říká, s~vyšším stupněm zasvěcení, nám mají být vzorem a autoritou.

Je to hierarchie postavená na kvalitě, kterou nelze sjednotit s~žádným systémem
vnějších, objektivních měřítek. Dalo by se usuzovat, že klerikalismus nebo i
nábožensky prosazovaná poslušnost světské vrchnosti (srov. Ř 13, 1) měly svoji oporu v~pravidle,
že kdo se dostane do kněžského stavu nebo kdo má patřičnou šlechtickou výchovu,
obdrží i jistý stupeň zasvěcení, ovšem toto je pouhý dohad. Každopádně nelze
jedno s~druhým zaměňovat.

Zůstává zde i finální autorita Boží, neboť čím je člověk na vyšší úrovni, tím
více je nutně sjednocen s~vůlí Boží a extrémním příkladem toho byl Ježíš
Kristus.

\subsection{Řízení světa}

K~tématu Božího řízení světa se pojí anekdota z~Makoňova rozhovoru s~jeho
posluchači. Jeden z~nich na něco reagoval poznámkou, že ,,ani vlas z~hlavy
nevypadne bez vůle Boží.`` Makoň ho pak zvýšeným hlasem několikrát vyzval, aby si
z~hlavy vytrhl vlas. Když to dotyčný po chvíli váhání rozpačitě udělal, zeptal
se ho Makoň: ,,Co ty víš o vůli Boží?{}``

Tato anekdota, o jejíž pravdivosti nemám důvod pochybovat, neboť jsem ji slyšel
vyprávět od očitého svědka, nastiňuje Makoňův důraz na to, že otázku Božího
řízení světa nelze pojímat naivně nebo mechanicky. Jeho snaha o to, aby tuto věc
posluchači nepochopili nesprávně, je zde patrnější než u většiny ostatních
témat.

\begin{enumerate}

\item{
Nahrávka s~označením \texttt{83-05}, od pozice 17:00.

% 83-05 17:00
\textit{%
,,Vidíme, že je tady nějaká hierarchie. Nějaký řádově vyšší a nižší způsob
života. Jenomže není nám dáno, abychom hleděli za hranice tohoto časoprostoru,
abychom věděli, jestli tento řád pokračuje dál. Jenže byli lidé a jsou lidé,
kteří rozšířili svoje vědomí a vědí, že to, co se nám tady jeví jako nějaký řád
hodnot, jako nějaká podřízenost, nadřízenost a tak dále i v~myšlenkovém světě,
jako například je znázorněno v~Otčenáši, tam je znázorněna ta podřízenost tím
seřazením těch modliteb. Tak jestliže jsou tady lidé, kteří o tomto vědí a
jestliže jim můžeme věřit nebo se můžeme dokonce přesvědčit, že mají pravdu, tak
pak bychom neměli pochyb o tom, že Bůh má smysl. Že v~tomto řádu musí být nějaký
vrchol a musí být nějaký spodek. Jestliže toto všechno tady na světě je, při
pouhém pohledu na tento svět, tak můžeme klidně předpokládat, že toto platí za
hranicemi této zákonitosti. Že existujou jiné ohrady, širší, zákonitost vyšší,
svobodnější a za jiných okolností probíhající a tak to jde dál, až možná tam na
vršku je něco nad tím zákonem. A to je v~pořádku, protože já když lidsky na to
jenom tak jdu, tak si řeknu: ,Je-li tady nějaký zákon, tak to přeci předpokládá
nějakého zákonodárce. Takže smysl Boha, to je smysl zákonodárce.` Ovšem dívat se
na Boha jenom jako na zákonodárce, to je chabý pohled. Já se zatím na to omezím,
protože nechci... o Bohu by se dala přednáška udělat, přednáším o Bohu asi
padesát dva let a nemá to konce, takže já se omezím na tenhleten postoj a řeknu,
že tedy smysl Boží je, že něco musí být řízeno tak, musí být řízeno tak, aniž
víme proč, musí být řízeno tak, aby to nebylo vidět, že je to řízeno. Protože
kdyby všechno bylo řízeno tak, aby bylo vidět, že je to řízeno, tak by se to
nedalo vůbec řídit. To je ten hlavní důvod. Já ty ostatní důvody, těch je
stovky, nebudu jmenovat, ale víte, že mnoho věcí probíhá ve vaší duši, které
vědomě neřídíte, ve vašem těle zrovna tak. Já budu mluvit jako k~lékařům. To je
automatismus ve vás živý: srdce, pohyb srdce, že ano, a nevíte, nestaráte se o
to, ne? Jenom že to přiživujete nějak a solidním způsobem žijete, tak to ono
funguje. Má to smysl o tom vědět? Nemá. A proto je nesmyslné vědět něco o Bohu,
který takhle samovolně řídí. Ten nepřemýšlí o tom řízení, vůbec ne. Samovolně
řídí shora, a to je naše pravlast, [kde] se všechno řídí tou mocí této samovolnosti.
To je obrovská moc. To je tak velká moc, viď? Vy jako lékaři to nejlíp pochopíte
ze všech lidí. Jaká je to obrovská moc toho automatismu, já mu říkám
[automatismus] jedna, který v~nás řídí všechen průběh vnitřních sekrecí a
klepání srdce a pohybu střev a to všechno dohromady a harmonizuje to dohromady,
ne? A my jsme k~tomu přišli jako slepí k houslím, ne? Jak je to dobře, že my
jsme tomu takhle přišli. Ne že bychom o tom neměli nic vědět, ale že to tady je.
A tak je taky dobře, že je tady tenhleten Bůh, že to takhle samovolně všechno jde
až nahoru, tam my vůbec nedozírnem. Ale ono se to dá rozšířit a smysl právě toho
lidského života, který jde za hranice toho... toho normálního vidění, spočívá
v~tom, že člověk rozšiřuje svoje vědomí do těch oblastí, kde už vlivem své morální
vyspělosti je schopen zasáhnout jako pán Bůh. Řek bych bez protivenství vůči
Bohu, bez zásahu do vůle Boží. To znamená se zřeknutím se vlastní vůle a se
ctěním nějaké vůle vyšší. Takže smysl toho celého je umět se podřídit vůli, která
je moudřejší.``
}

Pojetí Boha jako zákonodárce se dá v~určitém smyslu chápat jako protiklad
        k~ustavičnému tvoření: Zákonodárce zákony jednou vydá a dále už do dění
        nevstupuje. Makoň s~pojmem zákonitostí -- Bohem daných zákonitostí --
        pracuje bohatě, jak jsme viděli v~kapitole~\ref{kap:temata}. Dá se tedy
        říci, že je proti pojetí Božího řízení v~tom smyslu, že by Bůh do dění
        zasahoval nějakým explicitním vstupem. Pracuje však se schématem, kdy
        člověk může v~určitém okamžiku naplnit podmínky pro přechod z~jedné
        úrovně zákonitostí do druhé a tím zapříčinit radikální změnu vývoje
        události, což se zcela pochopitelně může jevit jako Boží zásah a to
        zpravidla není na škodu věci, ačkoliv by to Makoň asi považoval za
        nesprávný výklad.

Také chci upozornit na myšlenku, že člověk se Bohu rovná tehdy, když se Bohu
zcela podřídí: Jednat jako Bůh znamená podřídit se vůli Boží. To také dává
člověku možnost stát se prostředníkem Božího řízení světa v~dokonalejším smyslu
než bez tohoto svobodného podřízení se.
}

\item{
Nahrávka s~označením \texttt{kotouc-C01-c}, od pozice 00:00.

\textit{%
,,To si myslíte, že není vůbec možné. Že se od té doby zázraky nedějí, ale
je to z~toho důvodu, že mu nedovolujete, aby vstoupil do vašeho života.
Například tady že vstoupil tady doktor Elger zrovna v~okamžiku, než jsem to tam
chtěl začít vykládat, on považuje za režii. Já zase vím, že je to právo, které jsem
dal Ježíši, aby vstoupil do jeho života. Žádná režie. Já mám právo také dávat,
pokud se mi lidé dají, on se včera dal, právo, aby vstoupil on do jejich života,
víte? Protože nesmíte se cítit oddělenou bytostí, ne? A když mně ukáže, jako se
to stalo v~koncentráku poprvně: ,Těch pět mně přivedeš,` ne? tak to právo trvá,
ne? Žádná režie. To je právo. A ty máš taky právo -- každý má právo -- do života Ježíše
pozvat. To nebyla režie, to byla součást toho, že já jsem pozval do tvého
života Ježíše, rozumíš? Mně nejde o slovíčka. Mně by nemuselo vadit, že to
považujete za režii, ale já nechci ubírat tomu pánu Bohu na jeho cti. Nechci
ubírat na tom pochopení jeho úmyslu. On nechce váš zivot režírovat, nýbrž on
chce,
abyste mu patřili. On tebe teďka pozval k~tomu, aby ses naučil mu lépe patřit,
rozumíš? Jestli v~tom vidíš režii, já v~tom vidím pozvání.``
}

Zdá se, že pojem \textit{,,vyšší režie``} života bylo jakýmsi průběžně přítomným
konceptem těžko říci zda u Makoně nebo více u jeho posluchačů. Jeho původ mi
není znám, ale nabízí se společný vliv s~Přemyslem Pittrem.\cite{matous2014premysl}
Ve spisech se ve větší míře objevuje poprvé v~Umění žít\cite{KaMaUZ} z~roku 1969.

Za hlavní zde považuji, že člověk má možnost dovolit Bohu, aby mu vstoupil do
života, což se pak projeví tím, že Bůh do života začne zasahovat
nadpřirozeným řízením událostí.

}

\end{enumerate}

\subsubsection*{Interpretace}

Bůh řídí svět zcela dokonale do nejmenšího detailu, ale nikoliv tak, že by
vnucoval každé jednotlivosti, jak se má udát. Toto řízení je z~hlediska
omezeného lidského chápání nepostižitelné, ale dá se mu porozumět tak, že ho
člověk aplikuje na svůj vlastní život. Jako Bůh není řízen světem, nýbrž on řídí
svět, nemáme ani my být vláčeni zevními vlivy. Tohoto řízení však nemáme
dosáhnout tím, že bychom zevně kontrolovali každou událost, natožpak bližního.
Máme se sjednotit s~Boží vůlí, která v~posledku směřuje k~jedinému cíli
vědomého, trvalého spojení s~Bohem a v~jednotlivých situacích se projevuje
různě. Tím přinutíme všechny zevní okolnosti včetně jednání ostatních lidí, aby
nám v~dosažení cíle pomáhaly. Podobně jako zdravé tělo samovolně slouží záměrům
svého vědomí, tak slouží vesmír Bohu i člověku, který je s~Bohem za jedno. V~tom
spočívá řízení světa.

%Jak tedy Kristus kraluje ve světě, kde je zlo?

%V evangeliu jsme slyšeli, že Kristovo království není z tohoto světa. Tak tedy teď jsme ve světě, který je ve zlu postaven a tady se máme řídit Božími přikázáními a pak po smrti dojdeme do Božího království? Čili svět je pod vládou Satana, ne Krista, a Kristus kraluje až tam na druhém břehu? S tímhle si církevní nauka vystačila celý středověk. Byla by to ale škoda, si s tím vystačit i teď. Protože upřít celý svůj život na důvěru v něco, co nejde nijak ověřit, to prostě nevydrží nápor pochyb, které v životě přicházejí. A navíc nám to tvoří negativní přístup ke světu, k Božímu dílu a k tomu hlavnímu, čím se nám Bůh projevuje. Svět je duální, je tu dobro i zlo. A vidět svět jako zlý a ne jako dobrý, to je scestí.

%Já bych se rád soustředil spíš na to, jak to udělat, aby Kristus ve světě skutečně kraloval.


\section{Antropologie}

Viděli jsme v~podsekci~\ref{kap:keywords}, že \textit{,,člověk``} je podle
automatické metody jedním z~nejcharakterističtějších výrazů pro celkovou
Makoňovu nauku. Celé jeho snažení je motivováno tím, že člověk je schopen spojit
se vědomě s~Bohem a tomu jedinému má smysl zasvětit život. Vyhraněná
antropologie je tedy v~základu celého jeho působení. To ale neznamená, že by ji
celou dobu rozebíral: Naopak, člověku jako fenoménu se příliš často nevěnuje,
nýbrž antropologie zůstává jako východisko. I tak lze v~Makoňově mluveném
korpusu nalézt dostatek výživných antropologických úryvků.

\begin{enumerate}

\item{
Nahrávka s~označením \texttt{85-05A}, od pozice 05:14.

% 85-05A 05:14
\textit{%
,,Aby včelí rod byl zachován, ona nemůže mít lidský rozum, ona nemůže mít mozek.
To je veliká vymoženost, že nemá mozek. Centrálně by nedokázala to, co dokáže
včela decentrálním rozmístěním čidel, které nahrazují mozek, provést. Takže
včela může bezvadným, naprosto neomylným způsobem letět za květem, který je
vzdálen, který necítí a za kterým ona letí s~naprostou bezpečností a přistane na
tom květu, který jí ukázala ta včela, která ji tam navedla
ráno. A to je způsob, do kterýho nikdy nedorosteme. Protože místo
toho máme mozek. A ten nám toto neskýtá. A protože nám skýtá něco jiného. Nám
skýtá touhu přerůst přes svůj druh. Já chci být lepší, dokonalejší, rozumnější,
moudřejší, než byli moji rodiče. A proto mládež první chybu, kterou dělá, že se
staví na zadní nohy a říká: ,Moji předkové byli hloupí proti mně.` To je
první špatný pohled, chybný pohled na rodiče a na předcházející generaci.
Oni mají za sebou tohle, to je ten pochod kupředu už totiž. A proto se jim
tamto zdá být zaostalé. To, co oni právě zažívají, není to pravda. Oni si
prošli tou fází, kterou ta mládež teprve prochází, aby dospěli do nějaké úrovně,
za kterou už se beztak nemůže jít. A to oni neví, že oni taky dospějou do té
fáze, za kterou už nebudou moct jít. Neboť člověk má taky omezené možnosti. No a
tohleto prostě u těch nižších tvorů také existuje. Jenže ta první fáze toho
mládí, až na slony a některé vyspělé savce, probíhá velice rychle. Například
mládí u takové třebas mouchy probíhá v~půl dny. Co my musíme dělat
osmnáct let, tak ona za půl dne je s~tím hotova a je hotovou mouchou lítající a
umí lítat z~toho automatismu, který v~ní je. Čili my ten automatismus opravdu,
který v~sobě máme také a máme ho daleko složitější, máme ho
dokonalejší, než má třebas husa nebo včela. Ale my ho držíme v~šachu. My
mu nedovolujeme, aby se rozrůstal tam, kam se rozrůstá třebas v~oblasti
včely.
[...]
%To znamená, toto nechci přírodovědecky tady dál rozvádět, to by vás
%strašně unavovalo. Ale já zůstanu u člověka.
%A teď to už chápete, že já
%jsem se musel vzdát třebas přírodovědy ve svých patnácti letech. Jak je to
%dobře, že jsem se toho musel vzdát, protože já bych z~této strany do tý
%přírodovědy byl nikdy nevnikl. Dneska do tý přírodovědy vnikám způsobem, kterým
%málokterý přírodovědec do toho vniká. A závidí mně, když se s~ním setkám, že
%mám tento přístup k~věcem. No, ale nechci se tím chlubit, to jsem já nezavinil,
%to zavinili lékaři. Tak, teď chci říci:
Jestliže člověk je jediný z~tvorů, který
touží po tom, aby přerostl svého druha, aby byl lepší... nebo dokonce horší, což
je taky svým způsobem jiný, aby byl jiný, než byl ten vedlejší tvor, to je
aby nebyl stádový... tak je to známka toho, že dorůstá do pravé individuality,
která od všech relativností je oproštěna. To je ta věčná individualita, ze které
všechno jsme vzali a do které se zase nazpátek vracíme. Od Boha přicházíme a
k~Bohu se vracíme.``
}

Když hovoří o člověku, dává ho Makoň často do poměru ke zvířeti. Přisuzuje
člověku oproti zvířatům výjimečnost, ovšem musíme vzít v~potaz, že Makoň měl ke
zvířatům nesmírně blízko, zažíval sám sebe jako jednoho z~nich, a přechod do
lidské společnosti pro něj byl bolestným.\cite{KaMaUZ} Sounáležitost se zvířaty
pro něj ani potom nepominula: \textit{,,Já jsem dneska ještě víc zvířetem než vy. A já
se za to nestydím, já jsem už husou, prasetem, teletem, vším možným už jsem
byl.``} (Nahrávka \texttt{87-03}, 35:47.) Proto jestliže připisuje člověku
výsostné postavení, pak to není z~přezíravosti vůči zvířatům, nýbrž z~nutnosti
kapitulovat před tíhou zjevení.
Hlavním poselstvím zde však je, že člověk je výjimečný tím, že má touhu a tedy i
možnost stát se něčím víc než tím, čím je.

}

\item{
Nahrávka s~označením \texttt{80-07A}, od pozice 37:46.

% 80-07A 36:16
\textit{%
,,Tam [v~nebi] budete konvergovat k~Bohu. A teprve tady se občas snažíte, abyste se mu
přiblížili. Ale tam se mu budete věčně přibližovat. To je úděl bohužel člověka.
Budete věčně poznamenáni jeho svořitelským úkolem. A dobře nám tak. Jinak bychom
nemohli v~tom jeho stvořitelském úkolu fungovat na věky. A ono se od nás chce,
abychom navěky v~něm fungovali. On nás nepotřebuje jenom, abychom byli
spasitelé, on nás potřebuje, abychom byli spolustvořitelé. Spolustvořitelé.
Abychom v~jeho stvoření pokračovali. Jako jsme plodili tady děti, tím jsme se
stávali spolustvořitelé, tak potom chce, abychom zplodili duchovní bytosti.
Budeme dělat totéž. Budeme plodit, pořád budeme plodit, ale na jiné úrovni. Toto
plození na jiné úrovni, ale soustavné, už bez mezer, bude naše nebe.``
}

Že ve stavu nebe je člověk v~činnosti, zdůrazňuje Makoň opakovaně. Zde mi jde o
výrok, že člověk se má stát spoluspasitelem a spolustvořitelem, a také, že
plození bytostí je něco, co k~člověku patří nejen v~těle, ale i ve stavu nebe.

}

\item{
Nahrávka s~označením \texttt{kotouc-F01-b}, od pozice 13:00.

% kotouc-F01-b    13:00
\textit{%
,,Kámen je udržován v~existenci tím, že ustavičně svou existenci do něho Bůh
předává, ale to je kámen. U rostlin je to zase už lepší, tam se toho předává
víc, aby
mohla růst dokonce, a u člověka se předává největší míra těch vlastností Božích.
Tam se předává značná míra existenční, [proto] má velice složitý organismus proti
kameni, souhra buněk a to všecko a organický život je vůbec složitější než
anorganický, ale předává
se do něho také značná míra uvědomovací síly. On si uvědomuje tento svět do
takové míry, že z~něho si může analogicky vydedukovat život věčný. Může mít pojetí
o životu věčným. Dokud toto pojetí nemá, dokud se nedovede vmyslit, že
existuje věčný život, tak nemůže do něho vejít. A to zvíře si nedovede udělat
ještě. A člověk si může. Nemůžete zatoužit po něčem, o čem nic nevíte, co si
nedovedete
nějak představit, že to existuje, že ano? A člověk tuto představu může pojmout a
právě proto může za touto představou jít. On si vytvoří napřed představu a v~tom je
ztělesněno tolik síly od Boha, že pomocí představy která třeba ještě neodpovídá
zdaleka skutečnosti, jaká existuje za tou představou, se odebere do té skutečnosti
vyšší. Tak proto je život lidský připraven pro tyto vlastnosti, které má, k~tomu
přechodu do Věčnosti. A žádný jiný život není připraven.``
}

Z~této pasáže by mohlo plynout, že není jiné cesty ke spáse než skrze představu
o ní. Domnívám se, že toto Makoň vyjádřit nechtěl, a jinde také tvrdí
opak.\rref{83-08}{00:50}{50} Jde však o
to, že možnost představit si věčný život s~sebou nese možnost ho realizovat. A
ta je vlastní jen člověku.

}

\item{
Nahrávka s~označením \texttt{76-02-Kaly}, od pozice 22:37.

\textit{%
,,Ten Ježíš tady není, neexistuje pro mou radost, nýbrž proto že beze mě by
nevyrostl do Jordánu. Lidský život má takovou nesmírnou cenu, že bez něho by se ve
vesmíru Bůh nevyvinul ve spasitele. To je závažný slovo. My máme tvůrčí úkol nebo ještě víc
než tvůrčí úkol, prostě transformační úkol. My jsme transformační stanice něčeho
časného na věčné a bez této transformační stanice ani Bůh nedokáže z~vesmíru
udělat zase sám sebe. Jedině pomocí těchto transformátorů to dokáže a proto nám
posílá spasitele a proto od nás chce, abychom ho následovali a tak dále.``
}

}

Některé Makoňovy výpovědi o člověku a jeho podstatě mohou vyvolávat otázku, jaký
to má mít důsledek. Co z~toho plyne, že je člověk například transformační
stanicí mezi časným a věčným? Stejně jako vlastně u celého zbytku Makoňových
hovorů zde můžeme pozorovat spojující ideu, že člověk by se měl orientovat na
Věčnost a vědomé spojení s~ní.

\end{enumerate}

\subsubsection*{Interpretace}

Všichni jsme nic a klameme sami sebe v domnění, že jsme něco. Bůh je všechno.
,,Já jsem,`` to o sobě pravdivě říká. My jsme oproti Bohu opravdu nic. Nejenže
on by byl o tolik větší a my tak titěrní, že bychom byli jako nic, ale skutečně
nic, ačkoliv také jsme. Krásný příměr dává matematika. Přímka má nekonečnou
délku a bod nulovou. Bod existuje, ne že ne, ale je skutečné nic, je nula.
Stejně tak těleso v prostoru má svůj objem, a celý prostor má objem nekonečný.
My jsme boží obrazy a dvojrozměrný obraz existuje, to bezpochyby, ale objem má
nulový, je to opravdové nic. Tak i my existujeme, ale jsme opravdové nic oproti
Bohu, který je vším. A není to jenom negativní, snižující obraz o nás. Je také
pravda, že jsme v jádru božští, že máme božskou podstatu. Bůh nás tvoří a my
jsme s ním jedno. A ani Bůh nic není. Nelze na něco ukázat a říct: ,,To je
Bůh.`` To, že jsme něco -- tělo, mysl, sociální role, je klam. Myslíme si, že
jsme něco, a přitom nejsme nic, protože jsme jako Bůh. Z~něho všechno pochází,
ale není nic, co by bylo Bůh. Tak ani není nic, co by bylo ,,já``. My jsme obraz
Boží.


%Bible nabízí pro každou situaci, pro každou vývojovou fázi, pro každé východisko vedení. Zastávám sice názor, že pro vkročení do království nebeského je svým lidstvím kvalifikován každý člověk bez rozdílu pokročilosti, a to v každém okamžiku bez rozdílu posvátnosti, ale to nevylučuje důležitost systematické přípravy. Bůh je připraven nás do své náruče přijmout okamžitě a bezpodmínečně, překážka je vždycky stoprocentně na naší straně. Dlouhodobá, systematická práce na překonání těchto překážek je jedna z věcí, které udělat můžeme, a tedy i jedna z věcí, které udělat musíme, protože Bůh nás chce nekompromisně celé. A Bible je nejskvělejším průvodcem v takové systematické přípravě.

Bible nabízí obraz o člověku. Ne nahodilou snůšku obrazů, ale ucelený obraz.
Jeden z pohledů na Bibli je ten, že každá její část nás provází na různé úrovni vývoje. Stvořitelský mýtus objasňuje, jak přicházíme na svět. Abrahamovým zaslíbením se vydělujeme od ostatních, kteří žijí jen pro dočasné cíle (klanějí se modlám), a zasvěcujeme svůj život cíli věčnému. Historické a prorocké knihy nás provázejí zápasem o to, dostát tomuto cíli. Evangelia vykládají zrození božského života v nás a to, jak se v nás postupně ujímá vlády.

Z tohoto pohledu by Geneze byla inkubátorem a školkou, zbytek Tóry základní
školou a tak dále, až ke slovům evangelisty Jana coby doktorátu na Harvardu.
Jestli si na sebe chceme jakkoliv vztahovat např. slova z~Janovy epištoly:
\textit{,,Tyto věci psal jsem vám věřícím ve jménu Syna Božího, abyste věděli,
že máte věčný život, a abyste věřili ve jméno Syna Božího,``} (1J 5, 13) musíme
být lidmi \textit{s~doktorátem}. Jestliže nejsme, pak si myslet, že se nám zde
slibuje věčný život, je obyčená iluze. Ostatně zde Ježíš i vyprošuje zachování
od zlého (1J 5, 18) -- a kdo z nás může říct, že nad ním zlo nemá žádnou moc?

Pokusím se o příklad těchto úrovní a jejich rozpoznání. Máme za sebou genezi do
Abrahama? Jsme úplnými lidmi? Máme zvládnuté svoje světské záležitosti?
Živobytí, sexualitu, sousedské vztahy? Nic nás z toho nevyrušuje? Nezneklidňuje?
Jsme ,,pány země``, čili pozemské části svých životů? Umíme se ubránit? Říct ,,ne``,
když cítíme ,,ne``? Jestli něco z toho zvládnutého nemáme, zpátky do školky,
dorůst, splatit resty vůči sobě a okolí. Jestli máme, můžeme dál.

Máme uzavřenou smlouvu s Hospodinem, abychom byli jeho lidmi a on naším jediným
Bohem? Podřizujeme všechno tomu, co víme, že po nás Bůh chce? Neklaníme se
modlám?
Nesloužíme mamonu? Vyděláváme peníze poctivě a čestně? Pořizujeme si věci proto,
abychom jimi mohli účelně sloužit Bohu? Nebo taky proto, že to dělají ostatní,
nebo proto, že to na chvíli zaplní prázdné místo v duši?
Nesloužíme smilstvu? Milujeme se proto, že cítíme lásku? Neděláme si z partnera
nástroj uspokojení? Nebo nespíme s někým proto, že to od nás očekává?
Nesloužíme obžerství? Jíme tolik a takové věci, abychom mohli žít v harmonii se
Zemí a bytostmi na ní?
Pokud v něčem z toho selháváme, zpátky do základní školy. Na základě svědectví
ze Starého zákona můžeme v problémech, které nás stíhají, vidět láskyplné, ale
přísné usměrňování z Boží strany. Pokud to všechno naplňujeme, můžeme dál.

Narodil se v nás syn Boží? Hovoří v nás naše nesmrtelná podstata? Přemáhá v nás
zázračně všechno nemocné i zlé?
Jestli ne, zpátky na gymnázium. Milujme Hospodina, Boha svého celou svou silou,
celou svou myslí, celou svou duší a celým svým srdcem; a milujme bližního svého
jako sebe samé. Nedělejme si starosti, co bude zítra, ale buďme teď a tady, kde
je i náš nebeský Otec.

Tato interpretace, kde je potřeba splnit podmínky nižšího stupně, aby se člověku
otevřela cesta na stupeň vyšší, však není celou pravdou. Makoň sám zdůrazňuje,
že není důležité, na jaké úrovni se člověk nachází, ale jak je ochoten tuto
úroveň opustit.\rref{kotouc-plzen-E1-b}{03:27}{207}

\section{Christologie}

Klasickým christologickým otázkám jako lidské a božské přirozenosti nebo lidské
a božské vůle se Makoň prakticky nevěnuje. Pokud se jich dotkne, pak buď ve
zmínce, že o něčem takovém se kdysi v~křesťanství hloupě a naivně vedly spory,
nebo okrajově v~rámci jiného výkladu. U Ježíšova božství Makoň rozlišuje mezi
člověkem Ježíšem a preexistentním Bohem Kristem. Primárně mu nejde o to
vyložit, jak se v~historické události Ježíšova života tito dva skloubili, nýbrž
o vyložení vnitřní pravdy, která se ustavičně děje. Redukovat Ježíšovo
vystoupení na historickou, jednorázovou událost, považuje za fatální chybu.
Argumentuje tím, že v~duchovním světě není polopřímek: Cokoliv začalo, musí
skončit, a co je věčné, je věčné jak do minulosti tak do
budoucnosti.\rref{kotouc-plzen-E1-b}{16:05}{965} Z~toho
také plyne, že neexistuje žádné věčné zavržení v~pekle, neboť časné hříchy
nemohou mít věčný dopad.\rref{87-07}{05:07}{307} Naopak naše věčná, božská podstata nás předurčuje
k~tomu, abychom zažívali věčnost u Boha.

Ježíš Kristus podle Makoně dokonale demonstroval Boží spasitelský úkol a zároveň
dal dokonalý vzor pro lidský život. Tím se také Ježíš vymyká všem ostatním
lidem: Každý člověk může přesně a do nejmenších podrobností následovat Ježíšova
příkladu, jen musí Ježíšův příklad aplikovat do kulis svého života. Přirovnává
to k~matematickému vzorci. Ježíš udává obecný vzorec s~proměnnými a naše životy
jsou pak aplikací vzorce s~dosazenými hodnotami.\rref{prevzate-kotouc-79-06B}{27:10}{1630}

Za Ježíšův život považuje Makoň evangelní svědectví ze všech čtyř evangelií
dohromady. Vidí evangelia v~harmonii a bez rozporu a dokonce jako podrobná do
nejmejšího detailu.\rref{kotouc-I01-b}{47:06}{2826} Často uvádí, že Ježíš
vykřikuje na kříži: ,,Bože můj, Bože můj, proč's mě opustil?{}`` A že jedním
dechem dodává: ,,Odevzdávám ti svého ducha.``\rref{80-07B}{16:00}{960} Např. Ehrmann rozpoznává právě
v~těchto dvou verzích pašií jasný neslučitelný rozpor.\cite{ehrman2000new} Pro
Makoně to je naopak detailní rozpracování zážitku při mystické smrti.

Makoň Ježíše považuje za zatím posledního z~řady avatarů, Božích
vtělení.\rref{91-12A}{04:00}{240} Krom
Ježíše jmenuje Krišnu. Oproti svým předchůdcům přinesl Ježíš možnost dosáhnout spásy
během jediné inkarnace, a to pro každého, kdo ho celobytostně následuje. To
považuje za obrovský pokrok, neboť v~joginské tradici, jak říká, mistr vede jen
ty žáky, kteří mistrovi odpovídají svým založením.\rref{85-07B}{44:58}{2698} Navíc je tam pro spásu
potřeba mnoho vtělení. Makoň připouští, že až lidstvo opět dozraje, může přijít
další avatar, který Ježíše překoná.%TODO: ref

% 83-23A-K 39:42
%\textit{%
%,,Vy jste v~manželství s~tímto světem, o tom ani chvíli mě nenecháváme na
%pochybách. vás že tomu taky ale jiné manželství ten navázali když jste mu dokonale věrni a proto jakékoliv rady to je házení hrachu na stěnu. ježíš kristus navázal manželství jiné než mi to Já myslím, že když se narodil na tento svět, že mu nebylo jasno kým je. ale měl předpoklady pro to já vám řeknu které ne tím že byl avatarem aby brzo navázal nové řekl bych spojení nového uvozovkách spojení protože to manželství s bohem naše starší než ten náš původ od Otce je starší než jakýkoliv původ od čehokoliv jiného.``
%}

\begin{enumerate}

\item{
Nahrávka s~označením \texttt{84-27B-p}, od pozice 20:00.

% 84-27B-p 20:00
\textit{%
,,Ježíšovo učení jestliže se chápe jenom podle toho, co řekl, tak se nikdy
pochopit správně nemůže. Ono se taky nepochopí ani s~tím, co se dělo.
Ale to, co se dělo vedle toho, co říkal, je stejně důležité nebo ještě důležitější
možná než to, co říká. Já myslím, že to, co říkal je komentář k~tomu, co udělal,
ano? A
tak si všimněme, co udělal do třicátýho roku, co udělal od třicátého roku do
třiatřicátého, a co udělal potom. Vysvětleme si to jenom těmito třemi
periodami. Do toho svého třicátého roku nebyl učitelem, ne? Oslňoval svou
moudrostí,
ale nebyl učitelem. Byl poddán své rodině. Kdyby byl ustrnul na této úrovni, kdyby byl
nešel do Jordánu, nedal se pokřtít, nedal se svádět Satanem -- svým já -- nevzepřel
se mu, protože on potřeboval nám ukázat, že musíme se odpoutat od svého já takovým
způsobem, že nám nemá co do toho našeho chtění co mluvit, jo? Ne, co si přeje
ten Satan, to je to liské já, které si něco přeje. Vzhledem k~situaci, ve které byl
Ježíš Kristus, měl obrovskou moc k~dispozici, si to jáčko mohlo zase více přát
než u nás obyčejných lidí si přeje. A ono si přálo, aby proměnil chleby nebo kameny
v~chleby, aby seskočil z~nějaké věže a tak dále, aby se mu klaněl a podobně. A on
říkal... jo a ten Satan mu taky říká: ,Ale ono je to psáno ve svatém Písmu. Když ty
jako to neuděláš, tak ti nikdo neuvěří, že jsi opravdovým spasitelem.` A on to přesto
neudělal z~toho principu, že jedině Bohu je se klanět. Takže teď bych k~tomu chtěl
dodat: Jestliže třebas Ježíš Kristus učí svoje učedníky, prochází světem tehdejším
a dělá nějaké zázraky, tak to je fáze, o které se vyjádřil, že je menší než
Otec, za
prvé a jenom o té fázi se tak vyjádřil, za druhé, že nemůže jim zvěstovat to,
co by ještě nepochopili, protože to jim může dát jenom Duch svatý -- pochopení. Čili
je v~tom vidět takový proces vzestupu ve veškerém jeho konání. Co udělal pro své
učedníky po zmrtvýchvstání, po seslání Ducha svatého, nemohl pro ně udělat v~tom
období učitelství, kdy byli jenom jeho učedníky, ano?{}``
}

Jakožto vzor pro člověka prošel Ježíš cestu od nejnižší úrovně po nanebevzetí.
        Ježíš tedy u Makoně prochází vývojovými vrstvami a není pravda, že by po
        celou dobu svého života byl roven Bohu.

}

\item{
Nahrávka s~označením \texttt{77-05B-Praha}, od pozice 16:40.

% 77-05B-Praha 16:40
\textit{%
,,Takže tím jsem řekl, že Ježíš je projev Boha. Projev Boha živého a že to
není Bůh. Projev Boha živého, ovšem daleko vznešenější, než jsme my, ale menší
než Otec. A proto jestliže má přejít na pravici jeho, znamená [to, že] má mu být
roven, jako on o sobě říká, že se to stane, tak on se musí vzdát toho synovství.
Ne že by o to nestál, ale že je to
málo. Že je to jenom prostředek, Ježíš je prostředkem k~tomu, abych se dostal
k~Otci, ne? On nikdy
nikomu neříkal, že vede k~sobě, nýbrž vede do Království Božího, k~Otci, do domu Otcova a
tak dále. Že on je prostředkem, ničím více než prostředkem. No, čili Kristus je
od věky Bohem.
Ježíš je dočasně daným prostředkem vývojově dokázaným, to znamená: Je tam
ukázáno,
jakými proměnami ten prostředek, který nám dává k~dispozici, musí projít, aby se
člověk pomocí téhož prostředku dostal až na to nebe, byl tam  vzat, jak se říká, a ono je
to tak, že opravdu ten prostředek, který ze začátku vypadá jako malý dítě,
nemluvně, se
nám později jeví jako někdo úplně jiný. Za chvilinku je tesařem a za chvilinku je
učitelem a opět za ještě menší chvíli umírá na kříži.``
}
}

\item{
Nahrávka s~označením \texttt{83-23A-K}, od pozice 39:42.

% 83-23A-K 39:42 -- manželství se světem
\textit{%
,,Ježíš Kristus navázal manželství jiné než my. Já myslím, že když se narodil na
tento svět, že mu nebylo jasno, kým je. Měl předpoklady pro to, já vám řeknu
které -- ne tím, že byl avatarem -- aby brzo navázal nové, řekl bych spojení.
Nového v~uvozovkách spojení, protože to manželství s~Bohem naše je starší než...
ten náš původ od Otce je starší než jakýkoliv původ od čehokoliv jiného, co
mezitím bylo původem, také ale už odvozeným.``
}

Jedna z~ojedinělých pasáží, kde Makoň hypotetizuje o vnitřním životě Ježíšově.

}

\end{enumerate}

\subsubsection*{Interpretace}

Ježíš je nahlížen především jako vzor pro člověka -- jediný pravý mistr a jediný
člověk, kterého je správné následovat. Praví-li se, že Ježíš navázal spojení,
nebo že je nejdříve menším než Otec a později mu roven, znamená to, že i on
postupuje po vývojových úrovních. I v~tom je nám vzorem. To, že byl vtěleným
Bohem od počátku se nerozporuje. Spíše se pracuje s~tím, že vtělený Bůh vědomě a
dobrovolně podstupuje vývoj. Popírá se tak prakticky středověká představa
Ježíška jako homuncula, od narození plně vědomého a rozvinutého.\cite{zuffi2003gospel}
Není od narození roven Bohu, nýbrž prochází stejnou cestou jakou máme projít my.
Rozdíl je v~tom, že on tou cestou prochází dokonale a kvůli nám. Bůh se vtělením
do Ježíše omezil do lidství stejně jako to činí u nás, ale v~každé fázi Ježíšova
života Bůh skrze lidství prosvítal ideálním způsobem patřičným k~dané vývojové
fázi. Pro člověka to znamená možnost skutečně Krista následovat, protože nebyl
kategoricky něčím jiným než my, ale zároveň to Kristu ponechává jeho výlučnost a
božství, neboť my jdeme každý s~vlastní zátěží, kdežto on šel, aby nám
zprostředkoval možnost dojít také.
Ježíšovo božství spočívá v~tom, že byl zcela odevzdán Bohu. 

\subsection{Soteriologie}

U Makoně se soteriologie proplétá s~trinitologií a lze je jen těžko oddělit.
Makoň se tvrdě vymezuje vůči přímočaré interpretaci výroku, že Ježíš nás spasil
svojí smrtí na kříži. Opakovaně a s~velkým důrazem tvrdí, že v~žádném případě
nelze považovat svoji spásu za hotovou, a ještě razantněji odmítá, že by stačilo
věřit a ctnostně žít, nýbrž tvrdí, že se k~Ježíšově oběti musíme
přidat.\rref{81-08}{42:05}{2525} Přidává k~tomu, že každý člověk, který se dostane do
Království Božího, spasí stejným způsobem jen s~menším dopadem celé lidstvo,
stejně jako to učinil Ježíš.\rref{kotouc-R01-b-1974}{29:47}{1787} Používá zde
přirovnání, že stvoření je jako dlouhý řetěz a každý článek, který přejde za
hranici tohoto světa, s~sebou popotáhne celý zbytek řetězu.\rref{88-09B}{26:32}{1592}
Jedině v~tomto smyslu má podle něho být vykládáno, že nás Ježíš svojí obětí
spasil.

\begin{enumerate}

\item{
Nahrávka s~označením \texttt{80-12}, od pozice 00:40.

\textit{%
,,Jestliže Ježíš Kristus řekl: ,Já jsem cesta, pravda a život,` řekl, jakou
funkci Boží on zastává jako spasitel.``
}

Že Ježíšovo ztotožnění s~cestou, pravdou a životem
(nebo s~Cestou, Pravdou a Životem)
popisuje obsah spasitelského úkolu, je závažné tvrzení, kvůli kterému úryvek
uvádím. Co znamená, že Ježíš je cesta, pravda a život, Makoň vykládá na mnoha
místech, pokaždé z~jiné perspektivy.% Pojetí tohoto Ježíšova ztotožnění jako
%        deklarace spasitelské úlohy by zasluhovalo širší rozvedení, jež zde
%        nejsem s~to poskytnout.

}

\item{
Nahrávka s~označením \texttt{80-14}, od pozice 00:00.

\textit{%
,,Současně [se stvořením] probíhá odtvořování. To znamená, že Bůh také všechno miluje. Všechno
stvořené miluje a k~sobě táhne. A věci dospěly tak daleko, že bylo třeba jim to
názorně ukázat. Že [je] to názorná ukázka toho spasitelského úkolu Božího, který
od začátku stvoření tu jest. Čili neberte to utrpení Ježíše Krista za věc, která
se jenom stala, nýbrž za něco, co ustavičně jest, a to je rozdíl. Když my jsme
byli svědky jenom toho, přímo nebo nepřímo, co se stalo, to je málo. Teprve když
člověk vnikne do nesmrtelné podstaty Boží, poznává, co jest, a ví tedy, že také
jest ta láska Boží, která tvoří a odtvořuje a samozřejmě osvěcuje. [...] A potom
samosebou se celé to utrpení Ježíše Krista dostává do nového světla. Ne že by
byl on nepotřeboval, proto že jenom to ukazoval, trpět méně. Kdepak? On
soustředěně na sebe bral v~té chvíli utrpení všeho stvořeného a soustředěněji
trpěl, než jak trpí člověk, který trpí jenom za sebe. On za sebe vůbec netrpěl.
On to nedělal, ani toto nedělal pro sebe! I tady platí jeho slova: ,Kdybych něco
dělal pro sebe, nevěřte mně.` On ani toto utrpení nedělal pro sebe. Všechno pro
nás. Bylo třeba... prostě věci dospěly tak daleko, že musel nám to ukázat. Ale
nic nového se nestalo. To všechno jest a trvá. To tehdy ukázal. Čili ještě
jednou jinými slovy: To, co Ježíš ukazuje, to je lidové podání pro lidi určené,
lidové podání zákona, věčného zákona, který funguje ve veškerém stvořeném. A ten
zákon zní: Kdyby Bůh nemiloval stvořené, vůbec by ho byl nestvořil. Protože ho
miluje, tak ho zároveň k~sobě táhne. ,A tady vám ukazujeme,` říká pán Bůh, ,jak
to děláme, a kdokoliv z~vás se chce připojit vlivem tohoto zákona, který
existuje, k~tomu nestvořenému, tak to musí dělat tak jako to udělal Ježíš
Kristus. Nesmí z~toho nic vynechat. Musí také na ten kříž, musí také vstát
z~mrtvých a to musí udělat během toho pozemském života, tedy v rámci toho
stvořeného. Nikoliv někdy za zády stvořeného. Tady to musí udělat.`{}``
}

Krom již výše zmíněných akcentů na ustavičnost a nutnost lidského připojení se
        k~oběti, se setkáváme s~konceptem, který je pro Makoně taktéž
        charakteristický a objevuje se opakovaně, že totiž není spásy po smrti.
        Makoň zdůrazňuje, že fungujícího lidského těla jako transformátoru je
        nezbytně zapotřebí pro průnik na vyšší úroveň. Poslední a velmi
        příhodnou příležitostí pro vstup do Království Božího je okamžik smrti.
        Požadavek ctnostného života a tvrzení, že za něj bude člověk odměněn
        vstupem do nebe po smrti, považuje Makoň za
        zločin.\footnote{\texttt{83-20-K} 43:29; Blahoslavenství str. 4}

}

\item{
Nahrávka s~označením \texttt{83-11A}, od pozice 43:14.

% 83-11A 43:14
\textit{%
,,Ve mši svaté se modlíme: ,Vzal na sebe Ježíš veškeré naše hříchy, abychom byli
spaseni,` a to říkám ve zkratce. Jak je možný, aby jeden jediný tvor na sebe
vzal všechny hříchy všech lidí tak, aby je spasil? No tak na to právě není
rozumová odpověď, ale budete tomu přesto rozumět. Já se budu snažit z~obalu to
vysvětlit. Z~obalu, to znamená z~periferie. A to tímto způsobem: Jestliže třeba
dejme tomu už po těch sedmnácti letech, jak jsem získal vidění Boha... Vidění
v~uvozovkách, protože jsem žádnýho Boha neviděl, ale vědění o tom, že moje
podstata je nesmrtelná a že jsem jí povinován především -- vším, celým svým
životem jsem poddán této nesmrtelné podstatě... Tak od tohoto okamžiku jsem měl
také sílu k~tomu, abych za touto podstatou šel, abych všechno ostatní opomíjel a
za tímhletím šel. A jestliže se člověk připojí k~Ježíši Kristu, k~jeho oběti na
kříži, to znamená když se nespolíhá, že by vlastní silou mohl to dosáhnout, to
spojení s~Bohem, nýbrž přes Ježíše Krista, tak se mu dostane síla k~tomu, aby to
uskutečnil. To jsme viděli u svaté Terezie, která se napřed vlastní silou
snažila dosáhnout vědomého spojení s~Bohem. A ono jí to nešlo. Až Ježíš Kristus
ji upozornil, že se musí připojit k~jeho oběti. To znamená, že se musí připojit
k~jeho svatosti. Že se musí připojit k~vůli Boží. Svatost spočívá v~připojení se
k~vůli Boží. A vůle Boží je, abychom byli s~ním nakonec úplně spojeni, že? A jak
ona se s~touto vůlí spojila, tak pochopitelně musela být spasena a musela mít
trvalé spojení s~Bohem. A potom když někdo, kdo je trvale spojen s~Bohem, vás
přivezeme k~sobě, tak jste spaseni. Ovšem ne tak, jak to bere církev. Vy se
musíte k~tomu aktu spasení přidat. Dobrovolně, ne? A nejlépe po nějaké přípravce.
Protože když se tam bez té přípravky dostanete, tak se dostanete maximálně do
ráje a z~toho zase spadnete. To není nic. To svatá Terezie měla velikou jogickou
přípravu, abych tak řekl, indickým stylem řečeno. A proto se mohla trvale stát
nositelkou poznání Božího a tak dále. Tak je možno, aby jeden člověk nebo jeden
člověk, ovšem spojený s Bohem, spasil... byl spasitelem celého světa, ovšem jen
těch z~toho světa, kdo se k~němu přidají. Dobrovolně k~němu přidají. Bez jejich
vůle, [které] oni se tou chvílí vzdávají, to není možné. Ta vůle stojí mezi nimi
a Bohem a [je] nepřekonatelná i pro Boha. Když tedy Číňani dávno před Kristem
Indům vyčítali, že když chce někdo spasit sám sebe, že to není pravá spása, že
si něco nalhávají, protože porušují základní pravidlo, do kterého by se už byli
měli svým vývojem dostat, že se nelze oddělit od celku, nelze, že když se člověk
nevyrovná aspoň s~tím, že nepovytáhne svoje okolí, tak nemá sám právo. Nemá
[zkrátka?] tam vstoupit. On musí organicky navázat na své okolí a šplhat se
s~ním, jo? Asi tak. A to ostatní, o to se starat nemusí, protože je to [ve]
vzájemné spojitosti všecko, nemusí se starat, aby spasil celý svět.``
}
}

\end{enumerate}

\subsubsection*{Interpretace}

Výrokem \textit{,,Já jsem cesta, pravda a život.``} Ježíš označuje svoji funkci
jako spasitele. Cesta je to, po čem se kráčí. Jasně zde jde o cestu do
království Božího. Lze vyvodit, že není možné se omezit na metodu -- každá
je záležitostí tohoto světa, zatímco Ježíš (obzvlášť janovský) svět přesahuje.
Jde vlastně o variaci na taoistické ,,bez metody to nejde a jen s~metodou také
ne.`` Je-li Ježíš pravdou v~kontextu svojí spasitelské role, nelze nepomyslet na
výrok, že \textit{,,pravda vás vysvobodí.``} (J 8, 32) Tím, že tou pravdou je Ježíš, se ujasňuje,
že nestačí nějaká dílčí pravda, nějaký pravdivý výrok nebo prosté nelhaní. Jde
zde o pravdu, kterou člověk nemůže vlastnit, jen jí patřit, jak to nejzářněji
demonstroval mistr Jan Hus. Jestliže Ježíš je život a jde zde o jeho roli
spasitele, můžeme to vidět jako odkaz na výrok \textit{,,nech ať mrtví pochovávají
mrtvé.``} (Mt 8, 22) Teprve přijetím Boží spásy člověk procitne do skutečného života a život
do té doby vidí jako učiněnou smrt.

Spásu jako protiklad tvoření nelze interpretovat jako ničení. Jde o překonání
světa, ne o destrukci. Stejně tak jestliže spása coby protiklad stvoření je
motivována láskou Boží ke stvořenému, neznamená to, že by tvůrčí proces byl
prost lásky. Spíše jde o to, svořením projít a nezůstat v~jeho limitech, neboť
pak by člověk zanikl. Makoň skutečně expresivně zdůrazňuje, že kdo se během
života v~těle nespojí s~Bohem, \textit{chcípne jako
pes}.\rref{86-05A-Brno-9.2.1986-3}{04:44}{284} Myslí tím, že jeho vědomí
zanikne. Tomuto předejít a pomoci člověku k~tomu, aby své vědomí naštěpoval na
věčné vědomí a tak ho zachoval, je podstatou spasení.

Aby byl člověk Ježíšovou obětí spasen, nestačí pouhá proklamace víry. Je nutné
se k~Ježíšově oběti přidat. To znamená připojit se k~Boží vůli. Jak to konkrétně
provést, to je obsahem celého Makoňova díla.

Vtělení do lidství považuje Makoň za největší Ježíšovu oběť.\rref{81-03A}{32:05}{1925} V~rámci lidského
života však utrpení je jednorázovým a ojedinělým prvkem. To, že se Ježíšův
život podává jako utrpení, je perverzní a zcela zavádějící. Ježíšovo utrpení
trvalo nanejvýš čtyři dny, což je oproti třiatřiceti letům jeho života zcela
nepatrná doba. Ježíš opravdu není typ trpitele, což dokládá i tím, že křísil,
léčil, dokonce doplnil víno na svatbě. Je z~toho patrno, že následovat Ježíše
není trpět, ale radovat se ze života a pomáhat k~tomu i druhým. Trpět se má
v~minimální nutné míře.\rref{kotouc-E01-c}{58:56}{3536}

\section{Ekleziologie}

O církvi se Makoň vyjadřuje spíše sporadicky. Je však zřejmé, že římskokatolickou
církev a její tradici považuje za nejlepší dostupné prostředky na cestě k~Bohu,
jednak svojí univerzálností (může jich užít kdokoliv\footnote{Ne snad v~tom
smyslu, že by Římskokatolická církev měla tak otevřenou a tolerantní politiku
v~přístupu k~lidem, nýbrž v~tom smyslu, že její tradice a svátostný aparát
jsou samy od sebe vhodné pro každý druh lidí.}) a jednak svojí
propracovaností.\footnote{Mystika I, str. 90} Dalo by se říci, že svoji snahu o podání návodu na vstup do
Království Božího pojímá jako korekci římskokatolické tradice a výkladu
katolických svátostí. Tyto totiž pokládá za nejdokonalejší dostupný základ, na
němž může velice dobře stavět. Ostatní církve, a jmenuje zejména evangelickou,
pokládá za též legitimní, ale spíše odchylující se od správného přístupu, než
přibližující se k~němu.\rref{86-41B}{25:36}{1536}

Jeho postoj k~církvi by se dal přirovnat k~Ježíšovu postoji k~zákoníkům
z~Matouše 23: ,,Berte
si od nich svátosti, respektujte jejich úřad, ale neberte si z~nich příklad.``
Makoňovu někdy velmi ostrou, až zdrcující kritiku církve je tedy potřeba chápat
ve světle toho, že církvi vyčítá povrchové chyby na něčem neocenitelně dobrém.

Byl zásadně proti tomu, aby se jeho příznivci od katolické církve odlučovali, a
už vůbec aby zakládali nějaké nové hnutí. Vzniku nějakého \textit{,,makoňismu``}
se přímo hrozil a zapřísahával svoje posluchače, aby je něco takového nikdy ani
nenapadlo realizovat.\footnote{Z~ústního svědectví pamětníků}

Když však byl dotázán, proč je tolik různých církví, jak se mi dostalo
z~vyprávění, odpověděl, že je jich naopak málo. Každý člověk by prý měl mít
vlastní.

\begin{enumerate}

\item{
Nahrávka s~označením \texttt{82-06}, od pozice 01:32:25.

% 82-06 01:32:25
\textit{%
,,Jestli si třebas dejme tomu, co jste ještě vy jako církev neexistovali, to si
myslím, to je proti nějakým evangelíkům, ale když třebas se hádali v~sedmém až
devátém století koncily, jestli jsou andělé mužskýho nebo ženského rodu. A když
se hádaly potom dvě století, jestli žena má duši a jestli to není věc. Prosím
vás, co to bylo za teology? Co to bylo za úroveň duchovní? A koneckonců když
nikdy neměl právo mluvit ten, kdo byl duchovně na výši, nýbrž ten, kdo měl
v~ruce moc, [...] tak tímhletím jednáním církev na sobě prozradila,
že zavrhla Ježíše Krista. A zavrhla ho fakticky roku tři sta dvacet čtyři, když
dovolila vraždit a dovolila vlastnit. Těm tam končí křesťanství. A pracně to ti
prosťáčci partyzánsky dávali a kroutili dohromady. Takový svatý František
nevlastnil, že ano? A následkem toho si mohl vůbec troufat tomu Bohu se
přiblížit.``
}

Je patrno, že Makoň považuje za důležité, aby v~církvi měli slovo lidé duchovně
vyspělí a poznávající Boha. Nejde mu tedy primárně o apoštolskou posloupnost,
vysluhování svátostí či zastávání kněžského úřadu. Jde mu o duchovní vedení
        k~reálné spáse podobně, jako to dělají jogínští mistři, ovšem ideálně
        tak, aby se využily výhody křesťanství: 1) Dosažení spásy v~rámci jednoho
        vtělení (netřeba modelu reinkarnace) a 2) masové aplikovatelnosti. Nutno
        však podotknout, že za jediného Mistra pokládá Krista v~harmonii
        s~Klémentem Alexandrijským\cite{klement2019vychovatel} i řadou dalších.

}

\item{
Nahrávka s~označením \texttt{83-13}, od pozice 27:31.

% 83-13    27:31
\textit{%
,,Petr to [Ježíšovo božství] tu v~ten moment poznával a tady nám Ježíš Kristus ukazuje, že celé
náboženství má svůj pravý základ v~tomto zjevení, které není z~rozumu. To je
z~něčeho vyššího. Proto mu říká: ,Ty jsi skála a na té skále postavím církev
svou.` Pro tyhlety závany zjevení nad rozumem ho právem ustanovil nástupcem. I
když ten Jan ho měl raději než třebas ten Petr, byl mu věrnější, to bylo vidět
pod tím křížem, když všichni utekli a on tam za ním šel, nebyl nástupcem.
Protože on potřeboval pro ten stav lidstva, který bude následovat po jeho smrti,
aby v~čele té církve stál někdo, kdo je vrcholně chybující -- to je to zlo, které
nesmí být vymýceno -- a přitom přístupný zjevení. A to byl ten Petr: vrcholně
chybující. Nikdo se tolik (kromě Jidáše, to byl ale zrádce... toto zrádce
nebyl...) neproviňuje jako svatý Petr, že zapřel Ježíše třikrát, než jednou
kohout zakokrhal. To je vzor toho, ukázky toho, koho si to ten Ježíš Kristus
vyvolil. A my dneska se pozastavujeme třebas nad karamboly, který se staly
v~církvi během staletí a tisíciletí. I ta nevědomost, která tam vládla a třeba i
vládne, to by nás nemělo zastrašovat. To je nějakým způsobem v~plánu. Podívej,
tak například nezanevře na špatné následovníky dobrých příkladů, nýbrž bude si
vědom toho, že jsme křehké nádoby zhora až dolů a že je to v~záměru Božím,
abychom ten svůj koukol si z~milosti Boží podrželi do skonání věků. Protože
bychom jinak nevyužili správně dualismus.``
}

Chyby, kterých se církev v~historii dopustila, jsou zahrnuty do Božího plánu a
ustanovení chybujícího sv. Petra do role vůdce církve je toho symbolem.

}

\end{enumerate}

\subsubsection*{Interpretace}

Smyslem lidského života je vědomé spojení s~věčností. Je to zároveň nejtěžší
úkol, který na sebe člověk může vzít. Ačkoliv je to úkolem každého člověka, daří
se to jen nepatrné hrstce. Pomoc člověku na cestě ke spojení s~Bohem je tedy to
nejepší a nejvyšší, co lze ve světě činit. Ježíš jako první a jediný ukázal svým
příkladem cestu, která k~tomuto nejvyššímu cíli vede během jediné inkarnace a je
otevřená pro kohokoliv, kdo je ochoten se na ni celobytostně dát. Úkolem církve
proto je efektivně Ježíšovu cestu zprostředkovávat lidem a reálně je tak vést do
vědomého spojení s~věčností. Veškeré nároky na autoritu odvolávající se na
Ježíšem daný úřad nebo na zjevenou pravdu pak v~tomto pojetí ztrácejí moc, pokud
církev neplní svůj úkol vést lidi do vědomého prožívání věčného života. Církev,
která se redukuje na formální provádění obřadů a světskou pomoc je tak
podrobována tvrdé kritice. Stejně tak ovšem i s~církví středověkou, která
světskou pomoc podle Makoně zanedbávala,\rref{78-Kaly-CD03-tr05}{47.09}{47.09} což odporovalo příkladu Ježíšovu, který
léčil a křísil. Církev tedy má nezastupitelné místo a každému je radno být jejím
členem, ovšem nikoliv jen formálně, nýbrž opět v~rámci cesty ke spojení s~Bohem.

\subsection{Sakramentologie}

Ani svátosti nejsou pro Makoně typickým a častým tématem. Citované pasáže jsou
prakticky jediné, které jsem našel, věnující se vyloženě jim jako tématu. O
jednotlivých svátostech, především o křtu, večeři Páně a zpovědi mluví Makoň velmi
rozsáhle. O večeři Páně napsal samostatnou knihu Oběť mše svaté,\cite{KaMaMse} křtu věnoval
rozsáhlé pojednání v~Umění následovat Krista\cite{KaMaUNK} a o zpovědi píše
v~Postile.\cite{KaMaPost} Ostatním svátostem
se věnuje nepoměrně méně. O manželství hovoří několikrát, ale bez důrazu na
svátostný rozměr, spíše prakticky, ve svém stylu ,,jak využít manželství jako
prostředek na cestě k~Bohu``. O útěše nemocných převážně pod označením
,,poslední pomazání`` je několik letmých zmínek přítomno. Kněžskému svěcení a biřmování se
prakticky nevěnuje.

\begin{enumerate}

\item{
Nahrávka s~označením \texttt{83-09}, od pozice 45:09.

% 83-09    45:09
\textit{%
,,Samozřejmě tam ještě něco při té modlitbě, které my říkáme [u] mše svaté,
existuje něco, co my normálními smysly nemůžeme chápat. Ale co existuje
objektivně bez ohledu na nás. Já jsem jednou šel s~panem Válkem [slepcem] okolo kostela
v~Gottwaldově. A on nesledoval vůbec, kudy jdeme. Protože já jsem ho vedl za
ruku, on se mě totiž držel pod paží, takže vůbec přestal sledovat, kde jsme. Ani
si neuvědomoval, že jsme před kostelem a on najednou říkal: ,Tadyhle vpravo se
něco děje, kde to jsme, Karle?{}` Já jsem říkal: ,Před kostelem.` A on říkal: ,A
počkáme chvíli. Mě to tak tady tahne doprava.` A ono tam zrovna bylo
pozdvihování a proměňování, ano? A slyšeli jsme za chvilinku ty zvonky, které
tehdy zněly přitom a on to proměňování cítil na vzdálenost třiceti metrů asi tak
přibližně. Tak to je něco objektivního, ta transsubstanciace se tam děje bez
ohledu na to, že ten kněz to dělá mechanickým způsobem. Prostě to jim bylo
vštípeno tím dědictvím po svatém Petru a tak dále.``
}

Ve své podstatě velmi netypický výrok pro Karla Makoně. U modlitby kupříkladu i
        u jiných jevů trvá na tom, že mechanické provádění je nežádoucí, a že
        záleží jedině na duchu, ve kterém se provádí. Spoléhání na formální úřad
        svěřený Kristem je nešvar. Obecně souhlasí s~naukou, že i nehodný kněz
        vysluhuje platnou svátost, ovšem objektivita transsubstanciace na
        základě apoštolské posloupnosti je pro mě překvapujícím a heterogenním
        prvkem v~Makoňově nauce.

}

\item{
Nahrávka s~označením \texttt{91-29B}, od pozice 38:10.

% 91-29B#ts=2290.23 38:10
\textit{%
,,Většina třebas katolíků, já budu mluvit jenom za ně, že mezi ně patřím, že jo?
svým narozením a svým křtem, jsou pověrčiví: Že si myslí, že konání obřadů všech
-- chodění na mši svatou, ke svátostem, takzvaný svátostný život [...] a tak
jestliže bych já považoval například ty svátosti za dostatečný prostředek
k~tomu, abych uskutečnil to, co se dá uskutečnit jenom vírou, tak bych se
dopouštěl nesmírné chyby. Protože bych zastupoval víru pověrou. Co je v~tom
pověrčivého, v těch svátostech? Že totiž svátost, i ta nejlepší, je jenom
obřadem a v~tom není poznání. I když může být v~tom nějaké požehnání, to může
být. Ale poznání samo o sobě v~tom není. Takže kdo plní všechny možné obřady, je
to v~pořádku, ale pokud se nedostavuje poznání spojení, málokdy se to dostaví,
to téměř nikdy ne, tak to byla pověra, kterou jsme neuspěli.``
}

Zde opět typičtější Makoňův přístup. Výpověď, že svátost má vést k~poznání,
        považuji za nesamozřejmou.

}

\item{
Nahrávka s~označením \texttt{87-14A}, od pozice 34:07.

% http://radio.makon.cz/zaznam/87-14A#ts=2047.82 34:07
\textit{%
,,Pro mě když já mluvím třeba s~katolickým knězem a říka mně: ,Tak co, ty jsi
proti těm svátostem?{}` A já jsem říkal: ,Vůbec ne, je to ale příliš formální.
Duch se z toho ztratil, vy musíte toho ducha do toho vložit, jinak po vás
všichni budou plivat a budou vás vraždit a budou vás shazovat z~toho oltáře.`
[...] % A víte, že už se přihlásili ke mně knězi, říkali: my se divíme, že ještě
%u toho oltářes měl být, když jsme četli tvoje spisy. Třebas e... to Sladké jho,
%to jim v hlavě zatopilo. Ale až do konce ho četli. Taky já nemám než první dva
%díly, a druhý... další tři nemám, protože to je všechno u kněží, tam uvízlo to,
%koluje mezi kněžstvem katolickým, protože oni se chytají za nos a vědí, že tam
%je třeba do toho dodat ducha.
Není to špatné, ale není v~tom duch, duch z toho vyprchal. Žádnou formou já se
nemohu například stát pokřtěným, k~tomu musím dorůst. Já mohu zahájit tím
formálním křtem, ale já musím to dodělat. Já jsem ten křest dodělal teprve
v~sedmnácti letech. A tak to musí jít se vším, co je v~té katolické, se všemi
svátostmi já musím dojít k~sobě a tam se s~nimi musím vypořádat.``
}

Další doklad důrazu na vnitřní podstatu svátostí a nikoliv na formální stránku.
Pod křtem Makoň rozumí objektivní změnu stavu člověka, která není vázána na
církevní akt.

}

\end{enumerate}

\subsubsection*{Interpretace}

Vnější a vnitřní život a Písmo svaté jsou provázané vzájemným zrcadlením. To, co
se děje navenek je odrazem toho, co se děje v~nitru, a to je odrazem toho, co se
píše v~Písmu. Svátosti v~tomto pojetí, které prolíná celým Makoňovým dílem,
zaujímají zvláštní postavení, neboť v~nich se rozdíl mezi zobrazeným a
zobrazujícím stírá, respektive posouvá. Například při křtu nemluvíme o tom, že
jsme se nechali polít vodou, ale že jsme přijali křest. V~principu jsou svátosti
stejné činnosti jako jiné, jenže u nich si jsme vědomi toho, že pozbývají
smyslu, jestliže nejsou doprovázeny vnitřním obsahem. Pokud přistoupíme k~večeři
Páně a nespojíme se s~Kristem, pak jsme jen snědli oplatku a popřípadě si lokli
vína, ale k~účasti na večeři Páně ve skutečnosti nedošlo. Takto bychom měli
přistupovat k~celému životu a takto bychom měli číst Písmo svaté: Vše aby mělo
rozměr vnitřní realizace. Svátosti jsou tak vlastně můstkem k~přechodu do
prožívání života jako svátosti.

Mluví-li navíc Makoň o tom, že např. do křtu se musí dorůst a nelze se stát
pokřtěným žádnou formou, posouvá to svátostné úkony do pozice pouhých příslibů,
formálních aktů, které se s~reálným uskutečněním vůbec nemusí překrývat.

\section{Eschatologie}

V~případě eschatologie by se s~trochou nadsázky dalo Makoňovo pojetí shrnout do
jedné věty: ,,Konec světa je individuální událostí, která musí nastat před vstupem
do Království Božího.`` Podobně jako odmítá historický výklad biblického stvoření
světa, odmítá i historický výklad apokalypsy. Sám tvrdí, že konec světa zažil
v~koncentračním táboře,\rref{86-37B}{31:02}{1862} a u některých jiných osobností naznačuje, zda na základě
svého poznání koncem světa prošli či nikoliv.\footnote{\rid{80-12}{46:01}{2761},
\rid{87-14A}{41:10}{2470}}

\begin{enumerate}

\item{
Nahrávka s~označením \texttt{90-05A}, od pozice 12:41.

% 90-05A 12:41
\textit{%
,,Vrátíš se ještě někdy? To tam na tom nebi zůstaneš? A on říkal: ,Podívejte se,
já se vrátím až na konci světa a to se musím vrátit, protože když končí svět,
jako dneska končí svět pro mě, tak jednou bude končit taky pro vás, a to je čas,
ve kterém se kráčí do nebe, takže vám slibuju: Na konci světa přijdu.` A oni
řekli: ,Kdy to bude?{}` No a on řekl: ,Podívejte se, to záleží čistě na vás, kdy
bude ten konec světa.` ,No tomu nerozumíme. Přece když je konec světa, to je
něco jako zkáza tohoto světa.` ,Ne, to si to špatně představujete. Konec světa,
takový konec světa nikdy nebude. Ale ten konec světa nastane v~každém člověku,
který přerostl tu zemi a dostal se do povědomí jiného, ve kterém ví o tom nebi.
A kdokoliv z~vás se dostane do tohoto vědomí toho nebe, pro toho bude konec
světa.` ,No tak to je divné, to my nepoznáme.` ,Ale podívejte se: Napřed přijdu
já, potom přijde ten, kdo vás tam do toho nebe dovede, takže až přijdu já a
zeptáte se mě, tak já vám jasně řeknu: \guillemotright{}To jsem já, který
předchází toho, který po mně přijde a do toho nebe vás odvede a způsobí ten váš
osobní konec světa, že ve vás bude konec světa\guillemotleft{}`. Tak oni řekli:
,Tak to tedy poznáme.`{}``
}

Tento úryvek je vyňat z~dlouhé pasáže, kde Makoň spatra svými slovy pohádkově
        vykládá biblický příběh od Eliáše po Josefův návrat do Nazareta. Zde
        Makoň hovoří o Eliášovi a částečně taky za něho.
Individuální a nikoliv všeobecný rozměr konce světa zde explikuje velmi jasně a
,,po lopatě``.

}

% 90-16B-414! 14:05
%\texit{%
%,,...a teď jsem vám zažil konec světa. Takže já vím, co je to konec světa, že je to
%individuální záležitost člověka. A že marně ta církev čeká na slavné vzkříšení a
%na konec světa. A už to ,konec světa` vynechává ale slavný příchod -- to už tam
%není ten konec světa. ,Slavný příchod Kristův,
%toho se dočkáme,` že? A ,ať nám dá pán Bůh, abychom se toho dočkali,` tak se
%modlili několikrát,
%že jo? Vícekrát. No je hrůza. Kdyby to říkali středověce: ,Bude konec světa.`
%Ale už nemůžou říkat. Dva tisíce pořád čekají konec světa a pořád konec světa
%není,
%tak to už je to přestalo bavit, ne? Tak už to vynechávají. Jak pomalinku odcházejí od
%těch bludů. On si člověk nesmí nikdy zoufat nad jejich stavech pitomosti.``
%}

\item{
Nahrávka s~označením \texttt{82-25}, od pozice 21:15.

% 82-25
\textit{%
,,Když se tam vykládá o posledním soudu Ježíše Krista, tak se tam říká: ,Syn
člověka přijde s~veškerou mocí svou s~anděly,` že? ,A ve velké slávě to začne
soudit.` A soudit podle tohoto hlediska: Ti, kteří byli dobrý, ti pudou
nalevo... napravo a ti, který pudou do věčnýho zavržení, že jo? na pravo...
nalevo, tak. No to je konečně jedno, ty strany, ale jedny zavrhne a druhé
přijme. A které přijme? Ty, kteří jemu dali pít, jemu dali jíst, jeho navštívili
v~nemoci nebo v~žaláři, že ano? Tak by se řeklo a katolická církev chybně
usuzovala: ,No, když jsi dobročinný, tak je to v~nejlepším pořádku. To je to, to
je projev lásky.` To není v~nejlepším pořádku. Když oni si totiž představovali,
že ten soud, který se tam líčí, že to je fyzický konečný soud nad celým světem,
že nastane fyzický konec světa, tak se mýlili. To je líčen individuální soud
v~nitru člověka a tak taky všechny ty vlastnosti, které se tam líčí jako
spasitelné, třeba: ,dali jste mi jíst`, ,dali jste mi pít` a tak dále, jsou
vlastnosti, které má vyvinout jeden jediný člověk. To znamená: Tam se
reprezentativně jenom stanoví, že za všech i nemožných okolností mám milovat bez
ohledu na sebe. A jestliže člověk toto dělá... On tam v~tomto případě ukazuje
cestu lásky. On neukazuje cestu poznání, kterou je možnost se dostat také tam.
Cestu lásky. Tito lidé, kteří šli cestou lásky tam [???] nepoznávali. Ti...
poznání u nich kulhalo. Oni říkali: ,Kdy jsme tě viděli? Kdy jsme ti dali jíst?
Kdy jsme ti dali napít?{}` Říkal: ,Cokoliv jste udělali pro nejmenšího z~bratřích
mých nepatrných, mně jste učinili,` že ano? To znamená: Není třeba mít poznání.
Ono to bez poznání jde láskou také. Dokonce to jde bez takových peripetií, bez
těch těžkostí, jak jsme věděli u svatýho Jana. Ale je důležité vědět, že se to
týká jednoho jediného člověka a že poslední soud je nevyzpytatelná záležitost,
jak o ní se vyjádřil Ježíš Kristus, a já mám dojem, že ho podezřívá jak
katolická církev tak všechny ostatní, že on něčemu jako nerozuměl nebo že to
zalhal nebo že to tam bylo nějak přimontováno, vpašováno nebo tak nějak. Protože
on říkal: ,Amen, pravím vám,` tam [se] dušoval, ,amen`, to je dušování. ,Už jsou
tady někteří mezi vámi, kteří ten poslední soud zažijou.` No a to už panečku
pěkně víme, že nějaká doba od té doby uplynula a takový nějaký fyzický soud
nenastal. Čili z~toho je jasně vidět, že chtěl, jestli se nemýlil a jestli
nelhal, že myslel vnitřní soud, jo? Protože říkal všechny příznaky: Shromáždí se
všechny, že jo? Ano, musí se shromáždit všechny lidské [...]
vlastnosti v~jedné a téže bytosti, jinak se nikam nezhrabete. To
je podmínka, celobytostně tam jít, ano? A druhá podmínka oproti těm soudům,
který tady líčíme u Moodyho a u mě: Poslední soud se liší od těhletěch tím
způsobem, že je při něm zatracen hřích, zatracena špatnost. To znamená: Já už
nebudu potřebovat, řečeno vědečtěji, dualismu k~dalšímu vývoji. Já už nebudu
potřebovat po tomto posledním soudu například tělo, to už nebudu potřebovat, to
už tělo dohrálo roli, už jsem z~něho vydobyl, co se dalo, ano? Že jestli ho budu
mít, tak jenom pro nějaký úkol, který tady mám, ale ne pro sebe už, ano? To
znamená: Tam se vychází z~dualismu. Jak se tam správně říká: ,Do věčného
zavržení přijdou ti špatní.` Tak je to a čárku tomu dává svatý Jan, který tomu
moc dobře rozuměl, ve své epištole. Říká: ,Znovuzrození, to je to pravé
znovuzrození, to už je znovuzrození z~Ducha.` To pravé znovuzrození spočívá
v~tom, že člověk už dál nehřeší.``
}

Z~toho, že si Makoň dvakrát po sobě splete levou a pravou a pak jejich význam
bagatelizuje, nelze usuzovat, že by takový byl jeho přístup k~věci. Strany
nemusely být důležité pro tento konkrétní výklad, a navíc hovořil prakticky
vždycky spatra a mnoho hodin (často i mnoho dní) v~kuse, takže únava se někdy
nutně projevila. Význam levé a pravé strany vyzdvihuje\footnote{Úlohy, str. 59} například při výkladu
Matoušovy perikopy o odvržení pravého oka a pravé ruky, které člověka
pohoršují.\footnote{Umění následovat Krista, str. 149}
Celkově lpí na tom, že z~Bible nelze ani písmenko vynechat. Všechno v~ní má
neopominutelný význam.

Je vidět, že Makoň bere slova připisovaná Ježíšovi v~evangeliích za skutečně
Ježíšem pronesená. Nezpochybňuje tradiční autorství evangelií Ježíšovými přímými
učedníky. Historickou kritikou je prakticky netknut, dokonce jsem zaznamenal
postoj, že její přístup je zcestný, ovšem není mi známo, na jaké bádání přesně
reagoval.\rref{82-07}{40:17}{2417} Je znát, že soudobá teologie, se kterou se setkával, byla na žalostné
úrovni.

Zmínka Moodyho patrně naráží na knihu Život po životě,\cite{moody1975life}
kterou sám přeložil a okomentoval.\footnote{\href{http://makon.cz/pdf/MOODY.pdf}{makon.cz/pdf/MOODY.pdf}}
Především zde ale podkládá biblickým výkladem pojetí posledního soudu a konce
světa jako ryze individuální záležitosti, která nastává před fyzickou smrtí.

}

\end{enumerate}

\subsubsection*{Interpretace}

Jako Genesis nehovoří o historickém vzniku světa, nehovoří Zjevení Janovo o jeho
historickém konci. Naším cílem je Království Boží. Poslední soud je jedním ze
stupňů na cestě do něho. Makoňova eschatologie proto není naukou o posledních
věcech, ale o překonání časového prožívání a připojení se k~věčnosti. Ježíšova
smrt na kříži jakožto vzor pro naši mystickou smrt je tedy eschaton samo o sobě.
Je to konec časovosti, konec světa, který ovšem každý musí zakusit individuálně.
Nejde však o pouhé zažívání bezčasovosti v~usebrání, jak někteří moderní autoři
naznačují,\footnote{\href{https://rhythmictheologyproject.com/2016/10/08/flow/}{rhythmictheologyproject.com/2016/10/08/flow/}}
nýbrž jde o nevratný přechod do stavu naprosté nezávislosti na všem pomíjejícím.

Člověk, který prošel koncem světa, může dál žít v~těle. Není již schopen hříchu,
neboť k~Boží vůli se přimkl tak bytostně, že se jí cele řídí, čímž se uvádí
v~absolutní svobodu.

Vkládá-li Makoň Eliášovi do úst, že \textit{takový (rozuměj globální, fyzický)
konec světa nikdy nebude}, může mu jít o to skutečně tvrdit, jak se věci
v~budoucnu vyvinou či spíše nevyvinou, nebo o to radikálně se vymezit vůči
povrchní, historické interpretaci eschatologických výroků v~Bibli. Je možné, že
zamýšlel obojí, ale větší důraz vidím na druhou možnost. Makoňova slova mi
nebrání věřit v~konec světa podle Písma i v~globálním měřítku.

\section{Shrnutí}

Makoňovo pojetí náboženství je ve své podstatě nesmírně systematické. Sám se
vyjádřil, že to, co svými výklady dělá, je jen zaujetí rozumu, aby se zapojil do
cesty k~Bohu, a člověk tak mohl jít
celobytostně.\rref{kotouc-B01-70-08-29-Zlin-moderni\_tendence\_v\_oblasti\_poznani-b}{29:07}{1747} S~tím také harmonuje, že svoje
výklady nepovažuje ani za finální pravdu,\rref{81-12}{34:12}{2052} ani za nezbytné pro
lidskou spásu.\footnote{Plyne triviálně z~toho, že uznává za spasené mnohé svoje
předchůdce.} Pro
úspěšné hromadné uplatnění křesťanství v~moderní době však aktualizaci za nutnou
považuje a proto jí věnuje tolik úsilí.

Zmiňuje opakovaně, že kacířství spočívá v~tom, když člověk vytrhuje části Písma
z~kontextu a vydává je za celek.\rref{kotouc-A01-c}{01:16:59}{4619} Tím, že jsem vytrhl úryvky jeho hovorů
z~kontextu a udělal z~nich celek výkladu systematicko\-teologických otázek, jsem
se takového kacířství vůči jeho nauce mohl dopustit. Mým cílem však není
z~Makoňových slov učinit dogmatický celek, nýbrž dát nahlédnout na záběr a
charakter toho, co zvěstoval.
