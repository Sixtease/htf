\chapter{Teologie Karla Makoně}
\label{kap:teologie}

V~kapitole~\ref{kap:temata} jsem se na Makoňovy nahrávky díval jako na zavřenou
krabici neznámého obsahu a pokoušel jsem se do ní z~různých úhlů vrhnout světlo,
aby se ukázalo, co se v~ní nachází. V~této kapitole se chci nadále pokoušet
odpovědět na otázku, co Karel Makoň říká, ovšem z~jiného východiska. Tentokrát
mi nejde o to, pokrýt věrně celý obsah pomyslné krabice, nýbrž si vzít sadu
konkrétních otázek a zjistit, jak na ně Makoň odpovídá.

Za tyto otázky vezmu tradiční body systematicko-teologické nauky:
\begin{enumerate}
    \item{%
        Nauka o Bohu
        \begin{enumerate}
            \item{O Bohu jediném (boží podstata a vlastnosti)}
            \item{Boží trojjedinost}
        \end{enumerate}
    }
    \item{%
        Kosmologie, nauka o stvoření, o Bohu stvořiteli
        \begin{enumerate}
            \item{Bůh stvořitel}
            \item{Boží prozřetelnost: svět je řízen}
        \end{enumerate}
    }
    \item{%
        Antropologie: K~čemu je člověk, jaký má smysl a jak ho porušil
    }
    \item{%
        Christologie
        \begin{enumerate}
            \item{Christologie: o osobě, podstatě, přirozenostech}
            \item{%
                Soteriologie: o Kristově díle; o ospravedlnění v Kristu; o posvěcení
            }
        \end{enumerate}
    }
    \item{%
        Ekleziologie: o církvi, jaká církev je pravá, viditelná / neviditelná církev
        \begin{enumerate}
            \item{%
                Sakramentologie: nauka o svátostech; otázka a Božího slova a svátostí
            }
        \end{enumerate}
    }
    \item{Eschatologie}
\end{enumerate}

\section{O Bohu}

\textit{%
,,Teď bych chtěl říci: Tam hrozilo a hrozí dosud, že když takhle někdo si čte ten
Starý zákon i Nový, tak z Boha udělá bytost. Bůh nikdy bytostí nebyl. Takhle se
nám taky dodneška v~křesťanství Bůh definuje. Je to nejdokonalejší bytost, že
ano, všemohoucí, vševědoucí a tak dále. Kdyby tomu takhle bylo, tak by vůbec
nebylo možno se jinak spojit s~tím Bohem než bytostně. A já, který jsem nechápal
Boha také jinak než bytostně, třebas ne jako starce, to jedno, tak jsem
narazil podle starých Židů a jiných předpisů třebas indických na bytostný
způsob dosahování. Existenční já tomu říkám. Takže moje zkušenosti jsou také
z~tohoto oboru jako bytostné, jsou existenční. Ale jaké v~tom je nebezpečí? Že
totiž ten člověk, který si předělá Boha na bytost, nikdy nepochopí, Bůh tedy je
také stavem. Nebo považuje to za vedlejší, že je stavem. Že je stavem vědomí,
stavem lásky, stavem existence, stavem poznání.``
%My totiž poznáváme takhle lidsky
%tím způsobem, že když to vidíme, cítíme nebo smyslem. Smysly nám to předkládají a
%my to rozumně zpracujeme, tak teprve to považujeme za poznané. To ještě poznané
%vůbec není. Tedy to je jenom porozumění věci, ale my to považujeme za poznané.
%Kdežto kdybychom správně mysleli, jak se myslet má, tak bychom věděli, že abychom
%mohli něco tímto způsobem poznávat, je k~tomu třeba mít
%obecnou schopnost poznávací. Takže když už něco
%konkrétně poznávám, tak to je jenom používání té obecné schopnosti poznávací. A je to
%tedy poměr jako projevu k~neprojevenému. Že my obecný způsob nebo obecnou
%schopnost poznávací nepoznáváme. My víme, že poznáváme, teprve když něco
%konkrétního poznáváme, nebo abstraktního, to už je jedno. Prostě poznáme něco, co
%se nám předkládá k~poznání. ale myslíme si, že kdyby by se nám nic
%nepředkládalo, že by prostě potom nezbylo vůbec nic z~toho, co je v oblasti
%poznání, to je omyl.
}

\textit{%
,,Co se myslí tím, že jdeme k~Bohu jako k ničemu jako k~,,nic``, že ten Duch je
nic? A my, co považujeme za něco, nic vlastně není a tamto nic, co považujeme
za nic, je fakticky všechno? No, něco na tom je. Já tomu nemohu odporovat, této
myšlence, ale já bych rád to ještě z~hlubšího hlediska vysvětlil. Ježíš Kristus
se o tom vyjádřil tak, že on je úhelným kamenem a že ho zavrhli a že potom
nemohou postavit budovu bez toho úhelného kamene, že se to všechno zřítí a že to
nemá základy a podobně. My jsme totiž postaveni svou existencí na tomhletom
,,nic``. To znamená na něčem, co bychom nemohli konkretizovat jako existenci. Pro
nás přeci Bůh neexistuje. To si musíme přiznat. Já aspoň jako beznaboh si to
můžu přiznat. Pro mě do těch sedmnácti let Bůh neexistoval, to jsem byl hotový
beznaboh. A svou silou jsem jím pořád. Jenom mocí Boží chápu to jinak. Ale ze
své schopnosti víry, kterou nemám, bych to nepochopil. Já vás obdivuju. No ale
jestliže tomu takhle je, že ten člověk jde k~něčemu, co neexistuje na žádné
úrovni, nýbrž co ty úrovně jenom řídí, a kdyby to existovalo na úrovni
kterékoliv, tak by to nemohlo řídit. Tak je to to pravé nic, ke kterému se jde.
Po každé úrovni si můžete kráčet, ať je to sebevyšší. Třetí nebe ať je to, tak
po ní můžete kráčet. Se tam procházím mezi... může mi tam být dobře. Ale pravé
Tao není cesta, po které se dá kráčet. To je, to je to pravé nic. Tam až
vstoupíte, do toho nic, tak si uvědomíte s~Lao'c-em, ,,Tao je při všem. Ale do
ničeho nezasahuje. A chceš-li se přiblížiti k tomu Tao, nerozlišuj mezi
příjemným a nepříjemným. Kdo rozlišuje mezi příjemným a nepříjemným, nemůže se
přiblížiti k~pravé moudrosti.`` No, ono se tomu lidově říká: ,,Slunce Boží svítí i na
špinavou kaluž,`` ne? Nerozlišuje. Nesvítí víc do té
čisté kaluže, než do té špinavé, že ano? A v~tom smyslu je to nic. Že je to
vrcholně milosrdné. Ale tak nerozlišeně milosrdné, že už to z~lidského hlediska
jako milosrdenství nemůžeme poznávat. To je tak totálně milující, že z~hlediska
lidského, z~hlediska člověka, který je zvyklý, že láska někam teče, konkrétně
nějakým korytem k~něčemu, k~něčemu se vzpíná, tato láska to není, takže my to
jako lásku poznávat nemůžeme. My i tu lásku Boží poznáváme, jakože neexistuje,
že to nic. A zrovna tak je to s~tím bytím a s~tím poznáním a se vším. Tak bych
to takhle chtěl rozpitvat, abyste viděli, co je to nic. To je všeobsažné.
Na pozadí všeho existující. Ale na \textbf{pozadí} všeho existující. To
znamená, když rozbijete atom na nejmenší ještě další částky a ty zase,
tak Boha nenajdete. Ten není v~něčem existujícím. Ten není
tím existujícím, lépe řečeno. Ale je na pozadí toho. On to všechno udržuje, on
to všechno udržuje při vědomí, při existenci, podle toho, na jaké úrovni se to
vyskytuje, ne? Tak k~tomuto nic jdeme, takže když potom se dostaneme do toho
nic, tak je to pro nás ohromující zkušenost, protože my víme: ,,Je to nic, ale
tím jenom proto, že je to nic, tak je to všecko.`` Já se nedivím, že někdo je
takový pantheista, řekne, tak Bůh je všechno. Je to nesprávný postoj, to já
uznávám, to je velice, velice zvrhlý postoj. On není všecko. Ale všecko je
ustavičně z~něho. Ale má-li být z~něčeho ustavičně něco, jenom z~toho má
pocházet, tak ono to nemůže být tím nebo oním, to nemůže být.% To by se, to by
%potom z toho něčeho potom vycházelo se a Z toho druhého, to druhé už by nemohlo,
%to je, má jinou povahu, z toho vycházet.
To musí být opravdu nic, aby z~toho
všechno mohlo vycházet. Já nevím, to asi nesrozumitelné. Je to srozumitelné? Je
to těžko srozumitelné, no ale je to tak. Protože Pán Bůh kdyby se měl starat o
věci, které řídí, takhle by dopadlo. Nemá ani ruce ani nohy, a tím, že všechno
řídí, tak to vypadá, jako by to neřídil. Jako by byl nic. Tím, že všechno
miluje, tak to vypadá jako by nic nemiloval, protože kdo miluje vraha i toho,
kdo je vražděn, tak prosím vás, co to je? Jaká je to láska? To je nelidská
láska. Jako by ta láska neexistovala. Jakto že tady toho vraha nezastavíš, jakto
že mu nezastavíš ruku, když on vraždí? Co je to za lásku?``
}

\textit{%
,,Já bych to řekl asi nějakým přirovnáním: Bůh je bytost, promiňte, že takhle
budu mluvit, která... nebo jsoucnost, která je schopna nejvyšší možné lásky. Nikdo
není schopen takové lásky jako on, takže my snad můžeme tu hlásku přirovnat
k~lidem, kteří se mají vzájemně rádi. Já to udělám, ovšem samo sebou, že to bude nebe
a dudy ne?``
}

\textit{%
,,Jak líčili ti proroci a ti zasvěcenci toho Boha? Jako krutého mstitele, který
mstí hříchy do třetího a čtvrtého pokolení Já nevím, jak to pokolení k~tomu
vůbec přijde. A krutý mstítel,
soudce a tak dále. Všechno si to můžete v~Starém zákoně přečíst. Ono stačí tahleta
charakteristika. Co to je za vtip, takhle líčit milosrdného Boha? To je úžasně
moudré.
Protože jestliže člověk nevidí, neví o tom, že když je v~této fázi, že je na úrovni
přírodní, kde platí oko za oko, zub za zub, kde silnější má právo a neprohřešuje
se tím,  zabíjet toho slabšího nebo toho nemocného, tak nevychází z~údivu a
myslí si,
že je Bohem opuštěn. Ne, ten bůh se jeví tomu králíku jako absolutně
neexistující,
nebo když existující, tak jako nemilosrdný soudce, protože když ho přepadne orel,
tak je odsouzen k~zániku a žádný pán Bůh mu nepomůže, ani prstem nehne. A na to
vás upozorňuju, že kdykoli upadneme do stavu přírodního člověka, to znamená na
tom světě jenom jíme, pijeme, spíme a máme se dobře jako zvíře, tak není pro nás
žadné milosrdenství Boží připraveno, je pro nás připravenn Bůh nemilosrdný,
soudce, který bude trestat naše hříchy do třetího a čtvrtého pokolení.``
}

\section{Trinitologie}

\textit{%
,,Bohu nelze rozumět. A to je to, čím bych chtěl začít. Proč mu nelze rozumět?
Proč se s~ním
nelze setkat? Proč ho nelze vidět? To bych chtěl dopovědět proto, že existuje
Bůh,
který je vlastně trojjediný. To vědí Indové, to vědí Evropané, to vědí křesťani,
tedy všichni na světě. Staří národové. To věděli Sumerové. Co to je ta
trojjedinost? My si řekneme otec, duch a syn a tím to
vyřídíme. Nic jsme tím nevysvětlili. Naopak jsme zatemnili věc. Jsou to, řek bych
školsky, rozdělené funkce jednoho a téhož Boha. Funkce, čili úkoly, které na sebe
vzal nebo ustavičně bere jeden a tentýž Bůh Ten totiž má úkol stvořitelský,
kterýž ustavičně tvoří, a tentýž Bůh má úkol spasitelský, že to stvořené k~sobě
táhne. Teď si už jenom představte, jak je těžké porozumět jednání takového Boha,
který zároveň ustavičně něco od sebe z~domova posílá do světa od sebe pryč a zároveň
z~toho světa to táhne k~sobě. Jak rozeznat jednoho od druhého? Jak to oddělit?
Z~toho byli věřící
v~koncentráku úplně vedle. Všichni věřící, se kterými jsem se tam setkal.
Všichno říkali: ,jak se
na tohleto zvěrstvo může dívat pán Bůh? Jak mě může nechat takhle na holičkách?
Já
jsem přeci byl vždycky zbožný člověk a já tady takhle trpím bez pomoci a takový
zvíře gestapácký má na mě všechna práva a Bůh ani prstem nehne pro mě.`` A kdykoliv
jsem s~nimi souhlasil a takhle myslel, vždycky [jsem] dostal nějaký kopanec od
toho gestapáka. Zrovna! Přesně! To je řízení od pána Boha, opravdu, abych věděl, že
pán Bůh s~tím absolutně souhlasí, že takhle se o člověka nestará. Ale vedle
tohoto,
vedle těchto dvou principů Božích -- stvořitele a spasitele, existuje třetí jeho princip a
to jest, oni říkají křesťani ,potěšitele`, ale lépe bylo řečeno osvětitele, že
opravdu ten člověk vlivem toho Božího působení v~něm může být čím dál
osvícenější. Že nemusí slepě jít ani od toho Boha pryč, ani k~tomu Bohu
nazpátek,
nýbrž může plně s~plným vědomím vědět, co dělá. Velice nutné, aby tento třetí
princip tam taky existoval u toho Boha.``
}

% 82-22 55:16
\textit{%
,,Otec, Syn a Duch svatý. My si totiž nedovedeme vůbec představit, jak je
to možný, že na jedné straně pán Bůh něco tvoří a ustavičně to tvoří -- kdyby to
ustavičně netvořil, tak my nejsme vůbec na světě, jo? Musí pořád do toho svoji
bytostnou, podstatnou část vkládat, aby mohl existovat tady kámen, čas a prostor
jako celek a taky tedy člověk, čili to je proud, který míří od něho do
časoprostoru. Kdybyste se, jako normální člověk je, spojili jenom s~tímto
proudem,
to znamená žili si jenom pro sebe, tedy z~toho, co od něho dostáváte, aniž si
uvědomíte, že z~toho žijete, tak promrháte všecko. Ne protože byste se nechytli
toho druhého proudu, kterým je spasitel a to je tentýž Bůh, ne nějaký druhý,
jenomže zároveň s~touto silou odstředivou působí síla dostředivá. Vzájemně se drží
v~šachu, ano? Protože když máte odstředivou sílu deset a dostředivou deset,
deset bez desíti je nula a to je Bůh. A jsou tam obě dvě tyto síly, rozumíte mi
dobře? I ta desítka kladná i ta desítka záporná a přesto je to nula, že jo? To je
to nic přirozenosti Boží. Nesmírně veliká moc tvůrčí, nesmírně veliká moc
spasitelská. A co je to, to třetí? Ten Bůh utěšitel nebo osvětitel, Duch svatý?
To je zase něco, a o to se hádali katolíci s~pravoslavnými, jestli Duch svatý působí
jenom přes Ježíše nebo taky... jestli taky přes Ježíše a nejenom z Otce.
Pravoslavní říkají, že jenom z~Otce pochází, a katolíci, že pochází také ze
Syna,
vychází taky ze Syna. A roku tisíc našeho letopočtu se rozešli a vznikla
Pravoslavná církev z toho, to byly ještě důvody nějaký politický, ale to nevadí.
Zkrátka pro takovou maličkost se rozešli a byl to zbytečný
rozchod. Dneska by se to už mělo uspořádat, protože ona je to síla, která
působí skrze obojí. Je to jenom další funkce Boží, která se nedá vysvětlit plus
deset a mínus deset, nýbrž by se musela vysvětlit úplně jiným obrazem. Já ho
tady nebudu předvádět, protože to by byl obraz na čtvrt hodiny, ale je to síla
poznání,
se kterou si vůbec křesťané zatím nevěděli rady, a to z~toho důvodu, že je třeba
napřed tu kladnou desítku s~tou zápornou desítkou vyrovnat a teprve potom může
nastoupit ten Duch svatý.``
}

% 79-01 01:06:10
\textit{%
,,Tady nemějte starost, že by na tomto předělu vám nekynula žádná pomoc. Ta pomoc
je vyššího druhu a není závislá už od vedení událostmi nebo od vedení stvořením,
stvořeným nebo naším já, nýbrž je závislá jenom čistě na tom Bohu, který je jiným
Bohem než který tvoří a o kterém se vyjádřil Eckhart, že nemá nic společného
s~Bohem, který je stvořitelem, ano? Je to tentýž Bůh, ale je to jiná jeho tvář, která
nemá nic společného s~tím Bohem, který je stvořitelem.``
}

% 76-03A-Kaly-5 23:22
\textit{%
,,My totiž pořád slavíme trojici Boží: Bůh Otec, Bůh Syn a Bůh Duch svatý, ne?
Ale to synovství Otcovo nemůže být ve stavu určité míry oddělenosti, jak se to
jevilo po ty tři roky, kdy on poslouchal, ale byl menší než Otec. Potom až se
znovu stane synem, až se mu vrátí synovství na jiné úrovni, tak už bude roven
tomu Otci. On bude mít právo poslat Ducha svatého. A takhle bude vládnout celým
Božstvím, celým obsahem jeho moci. A ta moc největší spočívá v~tom vedení Duchem
svatým všeho. Ten Duch svatý ve všem vládne, ve všem je pravou živou podstatou.
Ten Duch svatý totiž je podstatou všech těch samovolných sil. Ten se ocitá na
jejich podkladě. Ale zatímco ty síly samy o sobě jsou trpné, On je přitom
aktivní, víte? To je ta stránka aktivní těch samovolných sil.``
}

\section{Stvoření}

% 82-02 42:07
\textit{%
,,Můžeme se opřít o vědecké bádání: archeologické bádání, antropologické bádání,
o všechny možný druhy věd se můžeme opřít a víme tohleto: Že na tom světě se
vyvíjeli napřed jednodušší tvorové, ale vždycky současně zvířecí říše
s~rostlinnou. Tato symbióza mezi zvířetem a rostlinou existuje od začátku toho
života na tomto světě, víte? Takže nedegradujme rostlinu na úroveň pod zvíře. To
je směr jdoucí jinam, než kam jde zvířecí život. Tyto dva musí být vedle sebe se
specializovaným úkolem, aby to vůbec všechno mohlo fungovat. Bez zvířat by
nebyla rostlina a bez rostliny by nebylo zvíře, tak to asi zjednodušeně je. Já
to dělám všechno velice zjednodušeně, školsky. A bez těch dvou by nebyl zase
člověk, To je taková nadstavba nad tímhletím. Ovšem člověk je na straně těch
zvířat, nikoli na straně rostlin. A tak teďka především bych byl rád, kdybyste
věděli, že do této vývojové teorie nemá vůbec co mluvit povídání Geneze ze
Starého zákona, že za šest dní Bůh něco stvořil a sedmý den odpočíval. Vůbec o
tom se tam nemluví. I když je tam jasně řečeno: ,,Napřed bylo to, potom bylo
tamto.`` To vůbec nemá souvislost s~dnešním pochopením vědy a nic to nemluví
proti dnešnímu pochopení vědy. Rádo se to staví jedno proti druhému, ale je tam
je to líčení stavu a tady je to líčení situace, že?. Dnešní věda si zkoumá
situaci, vývoj situační. Kdežto tam se mluví o stavové situacei, stavovém stavu.
Já bych tamto řekl takhle: Jestliže by v~člověku neexistovalo na té úrovni, na
které on je, šest dní tvůrčích Božích a sedmý den odpočinku Božího, tak on jako
člověk by nemohl ani prstem hnout směrem ze své úrovně. On by nemohl zdvihnout
hlavu k~Bohu.``
}

% kotouc-D01-d 01:00:15
%\textit{%
%,,My si představujeme, že Ježíš Kristus to jednou odnes na tom kříži a že nás
%tím spasil, a takhle to je klamný dojem. To jsou všechno symboly toho,
%co se věčně děje. Zrovnatak dejme tomu stvoření světa je
%symbol toho, co se věčně děje, ustavičně probíhá v~tom vesmíru. První den stvoření
%světa, druhý, třetí, čtvrtý, pátý, šestý i to odpočívání. Všechno tam souběžně
%probíhá.
%Ono se to nedá znázornit nijak jinak než jako děj, ale ono to vlastně je
%permanentní situace, ustavičně se opakující. Takže ten Ježíš je ustavičně
%například křižován, ale také ustavičně vstává z~mrtvých. Teď je otázka, jestil my
%se s~ním dáme ukřižovat jenom a jestli my s~ním vstaneme z~mrtvých a nebo
%ne.``
%}

% kotouc-T01-metoda-d 01:24:55
\textit{%
,,Ježíš v~té době svého mládí trošičku viděl víc do té činnosti svého Otce než
my. Bylo to způsobeno tím, že prošel narozením v~Betlémě, což nebylo obyčejné
narození. To je symbol už velice vznešeného znovuzrození, velice vznešeného
znovuzrození a my jsme tím neprošli, tak proto takhle nevidíme do dílny Boží. A
kdybych tedy, pokud si to dovedu vůbec představit, měl říci, jakým způsobem asi
Ježíš Kristus viděl do ty dílny Boží a proto viděl, že musí být v~tom, co je jeho
Otce, tak bych to řekl asi takhle: On už dobře znal, že to, co je ve Starém zákoně
pověděno o šesti dnech práce Hospodina a sedmém dnu odpočinku, že to je symbol
toho, co se ustavičně děje. Co ustavičně probíhá. Že probíhá ustavičně činnost Boží
a ustavičně sedmý den odpočinku. Kdo tohleto zažil, tak ví, že stvoření potom má
zcela jinou povahu, než jak my podle symboliky, kterou si nedovedeme zatím
vysvětlit, mu
přičítáme. Že totiž to není nějaká akce, po ní zase následuje další akce, ta tvůrčí
činnost Boží, jako je to tam líčeno, nýbrž že je to soustavná činnost na různých
úrovních, jak je to tam líčeno, neboť nedokážeme to jinak vyjádřit než symbolem
času a prostoru a pak to musím líčit jedno za druhým, když to jednou dáme do času
a do prostoru. Ale když to obrátíme, když z~toho symbolu přejdeme do té
bezčasovosti,
ve které se to děje, ze které to prýští, tak tam se zároveň rodí svět a zároveň se
ničí, zároveň je Bůh v~klidu. A kdyby tomu tak nebylo, tak my nikdy nemůžeme být
činní, protože on je jediný činitel. Jak bychom my mohli být činni, kdyby on
činitelem nebyl? Naše činnost je odvozena od činnosti Boží. Uboze odvozena od
činnosti Boží.
On je jediný činitel. Ale bychom nikdy nemohli být v~klidu, kdyby on v~klidu
nebyl.
Nikdy ne. A protože on je neustále v~klidu, a náš klid je odvozen z~jeho klidu,
tak také můžeme v~klidu být.``
}

\section{Řízení světa}

% 83-05 17:00
\textit{%
,,Vidíme, že je tady nějaká hierarchie. Nějaký řádově vyšší a nižší způsob
života. Jenomže není nám dáno, abychom hleděli za hranice tohoto časoprostoru,
abychom věděli, jestli tento řád pokračuje dál. Jenže byli lidé a jsou lidé,
kteří rozšířili svoje vědomí a vědí, že to co se nám tady jeví jako nějaký řád
hodnot, jako nějaká podřízenost nadřízenost a tak dále i v myšlenkovém světě,
jako například je znázorněno v Otčenáši, tam je znázorněna ta podřízenost tím
seřazením těch modliteb, tak jestliže jsou tady lidé, kteří o tom to vědí a
jestliže jim můžeme věřit nebo se můžeme dokonce přesvědčit, že mají pravdu, tak
pak bychom neměli pochyb o tom, že Bůh má smysl. Že v~tomto řádu musí být nějaký
vrchol a musí být nějaký spodek. Jestliže toto všechno tady na světě je, při
pouhém pohledu na tento svět, tak můžeme klidně předpokládat, že toto platí za
hranicemi této zákonitosti. Že existujou jiné ohrady, širší, zákonitost vyšší,
svobodnější a za jiných okolností probíhající a tak to jde dál, až možná tam na
vršku je něco nad tím zákonem. A to je v pořádku, protože já když lidsky na tom
jenom tak jdu, tak si řeknu: ,Je-li tady nějaký zákon, tak to přeci předpokládá
nějakého zákonodárce. Takže smysl Boha, to je smysl zákonodárce.` Ovšem dívat se
na Boha jenom jako na zákonodárce, to je chabý pohled. Já se zatím na to omezím,
protože nechci... o Bohu by se dala přednáška udělat, přednáším o Bohu asi
padesát dva let a nemá to konce, takže já se omezím na tenhleten postoj a řeknu,
že tedy smysl Boží je, že něco musí být řízeno tak, musí být řízeno tak, aniž
víme proč, musí být řízeno tak, aby to nebylo vidět, že je to řízeno. Protože
kdyby všechno bylo řízeno tak, aby bylo vidět, že je to řízeno, tak by se to
nedalo vůbec řídit. To je ten hlavní důvod. Já ty ostatní důvody, těch je
stovky, nebudu jmenovat, ale víte, že mnoho věcí probíhá ve vaší duši, které
vědomě neřídíte, ve vašem těle zrovna tak. Já budu mluvit jako k lékařům. To je
automatismus ve vás živý: srdce, pohyb srdce, že ano, a nevíte, nestaráte se o
to, ne? Jenom že to přiživujete nějak a solidním způsobem žijete, tak to ono
funguje. Má to smysl o tom vědět? Nemá. A proto je nesmyslné vědět něco o Bohu,
který takhle samovolně řídí. Ten nepřemýšlí o tom řízení, vůbec ne. Samovolně
řídí shora, a to je naše pravlast, se všechno řídí tou mocí této samovolnosti.
To je obrovská moc. To je tak velká moc, viď, vy jako lékaři to nejlíp pochopíte
ze všech lidí. jaká je to obrovská moc toho automatismu, já mu říkám
[automatismus] jedna, který v~nás řídí všechen průběh vnitřních sekrecí a
klepání srdce a pohybu střev a to všechno dohromady a harmonizuje to dohromady,
ne? A my jsme k~tomu přišli jako slepí k houslím, ne?. Jak je to dobře, že my
jsme tomu takhle přišli. Ne že bychom o tom neměli nic vědět, ale že to tady je.
Atak je taky dobře, že je tady tenhleten Bůh, že to takhle samovolně všechno jde
až nahoru, tam my vůbec nedozírnem. Ale ono se to dá rozšířit a smysl právě toho
lidského života, který jde za hranice toho toho normálního vidění, spočívá v
tom, že člověk rozšiřuje svoje vědomí do těch oblastí, kde už vlivem své morální
vyspělosti je schopen zasáhnout jako pán Bůh. Řek bych bez protivenství vůči
Bohu, bez zásahu do vůle Boží. To znamená se zřeknutím se vlastní vůle a se
ctění nějaké vůle vyšší. Takže smysl toho celého je umět se podřídit vůli, která
je moudřejší.``
}

\textit{%
,,To si myslíte, že není vůbec možné. Že se od té doby zázraky nedějí, ale
je to z~toho důvodu, že mu nedovolujete, aby vstoupil do vašeho života.
Například tady že vstoupil tady doktor Elger zrovna v~okamžiku, než jsem to tam
chtěl začít vykládat, on považuje za režii. Já zase vím, že je to právo, které jsem
dal Ježíši, aby vstoupil do jeho života. Žádná režie. Já mám právo také dávat,
pokud se mi lidé dají, on se včera dal, právo, aby vstoupil on do jejich života,
víte? Protože nesmíte se cítit oddělenou bytostí, ne? A když mně ukáže, jako se
to stalo v~koncentráku poprvně: ,Těch pět mně přivedeš,` ne? tak to právo trvá,
ne? Žádná režie. To je právo. A ty máš taky právo -- každý má právo -- do života Ježíše
pozvat. To nebyla režie, to byla součást toho, že já jsem pozval do tvého
života Ježíše, rozumíš? Mně nejde o slovíčka. Mně by nemuselo vadit, že to
považujete za režii, ale já nechci ubírat tomu pánu Bohu na jeho cti. Nechci
ubírat na tom pochopení jeho úmyslu. On nechce váš zivot režírovat, nýbrž on
chce,
abyste mu patřili. On tebe teďka pozval k~tomu, aby ses naučil mu lépe patřit,
rozumíš? Jestli v~tom vidíš režii, já v~tom vidím pozvání.``
}

\section{Antropologie}

% 85-05A 05:14
\textit{%
,,Aby včelí rod byl zachován, ona nemůže mít lidský rozum, ona nemůže mít mozek.
To je veliká vymoženost, že nemá mozek. Centrálně by nedokázala to, co dokáže
včela decentrálním rozmístěním čidel, které nahrazují mozek, provést. Takže
včela může bezvadným, naprosto neomylným způsobem letět za květem, který je
vzdálen, který necítí a za kterým ona letí s naprostou bezpečností a přistane na
tom květu, který jí ukázala ta včela, která je tam... která ji tam navedla,
ráno. A to je spo- způsob, kterým- do kterýho nikdy nedorosteme. Protože místo
toho máme mozek. A ten nám toto neskýtá. A protože nám skýtá něco jiného. Nám
skýtá touhu přerůst přes svůj druh. Já chci být lepší, dokonalejší, rozumnější,
moudřejší, než byli moji rodiče. A proto mládež první chybu, kterou dělá, že se
staví na zadní nohy a říká: "Moje... moji předkové byli hloupí proti mně." To je
první š- špatný pohled, chybný pohled na rodiče a na předcházející generaci.
Voni mají za sebou tohle, to je ten pochod kupředu už totiž. A proto se jim
tamto s- zdá být zaostalé. To, co oni právě zažívají, není to pravda. Voni si
prošli tou fází, kterou ta mládež teprve prochází, aby dospěli do nějaké úrovně,
za kterou už se beztak nemůže jít. A to voni neví, že voni taky dospějou do té
fáze, za kterou už nebudou moct jít. Neboť člověk má taky omezené možnosti. No a
tohleto prostě u těch nižších tvorů také existuje. Jenže ta první fáze toho
mládí, až na slony a některé vyspělé ty savce, probíhá velice rychle. Například
mládí u takové třebas mouchy, probíhá v půl dny, co... co my musíme dělat
osmnáct let, tak ona za půl dne je s tím hotova a je hotovou mouchou lítající a
umí lítat z toho automatismu, který v ní je. Čili my ten automatismus opravdu,
co- který v sobě máme také a máme ho daleko složitější, máme ho dok-
dokonalejší, nemáš- než má třebas husa nebo včela. Ale my ho držíme v šachu. My
mu nedovolujeme, aby se rozrůstal tam, kam se rozrůstá třebas v oblasti prácem-
včely. To znamená, toto nechci přírodovědecky tady dál rozvádět, to by vás
strašně unavovalo. Ale já zůstanu u člověka. A řšš- teď to už chápete, že já
jsem se musel vzdát třebas přírodovědy ve svých patnácti letech. Jak je to
dobře, že jsem se toho musel vzdát, protože já bych z této strany do tý
přírodovědy byl nikdy nevnikl. Dneska do tý přírodovědy vnikám způsobem, kterým
málokterý přírodovědec do toho vniká. A závidí mně, když se s ním setkám, žes...
mám tento přístup k věcem. No, ale nechci se tím chlubit, to jsem já nezavinil,
to zavinili lékaři. Tak, teď chci říci: Jestliže člověk je jediný z tvorů, který
touží po tom, aby přerostl svého druha, aby byl lepší. Nebo dokonce horší, což
je taky svým způsobem jiný. Aby byl jiný, než byl ten vedlejší tvor. To je neb-
aby nebyl stádový. Tak je to známka toho, že dorůstá do pravé individuality,
která od všech relativností je oproštěna. To je ta věčná individualita, ze které
všechno jsme vzali a do které se zase nazpátek vracíme. Od Boha přicházíme a k
Bohu se vracíme.``
}

% 80-07A 36:16
\textit{%
,,Tam budete konvergovat k~Bohu. A teprve tady se občas snažíte, abyste se mu
přiblížili. Ale tam se mu budete věčně přibližovat. To je úděl bohužel člověka.
Budete věčně poznamenání jeho svořitel úkolem. A dobře nám tak. Jinak bychom
nemohli v~tom jeho stvořitelském úkolu fungovat na věky. A ono se od nás chce,
abychom navěky v~něm fungovali. On nás nepotřebuje jenom, abychom byli
spasitelé, on nás potřebuje, abychom byli spolustvořitelé. Spolustvořitelé.
Abychom v~jeho stvoření pokračovali. Jako jsme plodili tady děti, tím jsme se
stávali spolustvořitelé, tak potom chce, abychom zplodili duchovní bytosti.
Budeme dělat totéž. Budeme plodit, pořád budeme plodit, ale na jiné úrovni. Toto
plození na jiné úrovni, ale soustavné, už bez mezer, bude naše nebe.``
}

% kotouc-F01-b    13:00
\textit{%
,,Kámen je udržován v~existenci tím, že ustavičně svou existenci do něho Bůh
předává, ale to je kámen. U rostlin je to zase už lepší, tam se toho předává
víc, aby
mohla růst dokonce, a u člověka se předává největší míra těch vlastností Božích.
Tam se předává značná míra existenční, [proto] má velice složitý organismus proti
kameni, souhra buněk a to všecko a organický život je vůbec složitější než
anorganický, ale předává
se do něho také značná míra uvědomovací síly. On si uvědomuje tento svět do
takové míry, že z~něho si může analogicky vydedukovat život věčný. Může mít pojetí
o životu věčným. Dokud toto pojetí nemá, dokud se nedovede vmyslit, že
existuje věčný život, tak nemůže do něho vejít. A to zvíře si nedovede udělat
ještě. A člověk si může. Nemůžete zatoužit po něčem, o čem nic nevíte, co si
nedovedete
nějak představit, že to existuje, že ano? A člověk tuto představu může pojmout a
právě toto může za touto představou jít. On si vytvoří napřed představu a v~tom je
ztělesněno tolik síly od Boha, že pomocí představy která třeba ještě neodpovídá
zdaleka skutečnosti, jaká existuje za tou představou, se odebere do té skutečnosti
vyšší. Tak proto je život lidský připraven pro tyto vlastnosti, které má, katomu
přechodu do věčnosti. A žádný jiný život není připraven.``
}

\textit{%
,,Ten Ježíš tady není, neexistuje pro mou radost, nýbrž proto že beze mě by
nevyrostl do Jordánu. Lidský život má takovou nesmírnou cenu, že bez něho by se ve
vesmíru Bůh nevyvinul ve spasitele. To je závažný slovo. My máme tvůrčí úkol nebo ještě víc
než tvůrčí úkol, prostě transformační úkol. My jsme transformační stanice něčeho
časného na věčné a bez této transformační stanice ani Bůh nedokáže z~vesmíru
udělat zase sám sebe. Jedině pomocí těchto transformátorů to dokáže a proto nám
posílá spasitele a proto od nás chce abychom ho následovali a tak dále.``
}

\section{Christologie}

% 83-23A-K 39:42
%\textit{%
%,,Vy jste v~manželství s~tímto světem, o tom ani chvíli mě nenecháváme na
%pochybách. vás že tomu taky ale jiné manželství ten navázali když jste mu dokonale věrni a proto jakékoliv rady to je házení hrachu na stěnu. ježíš kristus navázal manželství jiné než mi to Já myslím, že když se narodil na tento svět, že mu nebylo jasno kým je. ale měl předpoklady pro to já vám řeknu které ne tím že byl avatarem aby brzo navázal nové řekl bych spojení nového uvozovkách spojení protože to manželství s bohem naše starší než ten náš původ od Otce je starší než jakýkoliv původ od čehokoliv jiného.``
%}

% 84-27B-p 20:00
\textit{%
,,Ježíšovo učení jestliže se chápe jenom podle toho, co řekl, tak se nikdy
pochopit správně nemůže. Ono se taky nepochopí ani s~tím, co se dělo.
Ale to, co se dělo vedle toho, co říkal je stejně důležité nebo ještě důležitější
možná než to, co říká. Já myslím, že to co říkal je komentář k~tomu, co udělal,
ano? A
tak si všimněme, co udělal do třicátýho roku, co udělal od třicátého roku do
třiatřicátého, a co udělal potom. Vysvětleme si to jenom těmito třemi
periodami. Do toho svého třicátého roku nebyl učitelem, ne? Oslňoval svou
moudrostí,
ale nebyl učitelem. Byl poddán své rodině. Kdyby byl ustrnul na této úrovni, kdyby byl
nešel do Jordánu, nedal se pokřtít, nedal se svádět Satanem -- svým já -- nevzepřel
se mu, protože on potřeboval nám ukázat, že musíme se odpoutat od svého já takovým
způsobem, že nám nemá co do toho našeho chtění co mluvit, jo? Ne, co si přeje
ten Satan, to je to liské já, které si něco přeje vzhledem k~situaci, ve které byl
Ježíš Kristus, měl obrovskou moc k~dispozici, si to jáčko mohlo zase více přát
než u nás obyčejných lidí si přeje. A ono si přálo, aby proměnil chleby nebo kameny
v~chleby, aby seskočil z~nějaké věže a tak dále, aby se mu klaněl a podobně. A on
říkal... jo a ten Satan mu taky říká: ,ale ono je to psáno ve svatém písmu. Když ty
jako to neuděláš, tak ti nikdo neuvěří, že jsi opravdovým spasitelem.` A on to přesto
neudělal z~toho principu, že jedině Bohu je se klanět. Takže teď bych k~tomu chtěl
dodat: Jestliže třebas Ježíš Kristus učí svoje učedníky, prochází světem tehdejším
a dělá nějaké zázraky, tak to je fáze, o které se vyjádřil, že je menší než
Otec, za
prvé a jenom o té fázi se tak vyjádřil, za druhé, že nemůže jim zvěstovat to,
co by ještě nepochopili, protože to jim může dát jenom Duch svatý -- pochopení. Čili
je v~tom vidět takový proces vzestupu ve veškerém jeho konání. Co udělal pro své
učedníky po zmrtvýchvstání, po seslání Ducha svatého, nemohl pro ně udělat v~tom
období učitelství, kdy byli jenom jeho učedníky, ano?``
}

% 77-05B-Praha 16:40
\textit{%
,,Takže tím jsem řekl, že Ježíš je projev Boha. Projev Boha živého a že to
není Bůh. Projev Boha živého, ovšem daleko vznešenější, než jsme my, ale menší
než Otec. A proto jestliže má přejít na pravici jeho, znamená má mu být
roven, jako on o sobě říká, že se to stane, tak on se musí vzdát toho synovství.
Ne že by o to nestál, ale že je to
málo. Že je to jenom prostředek, Ježíš je prostředkem k~tomu, abych se dostal
k~Otci, ne? On nikdy
nikomu neříkal, že vede k~sobě, nýbrž vede do Království Božího, k~Otci, do domu Otcova a
tak dále. Že on je prostředkem, ničím více než prostředkem. No, čili Kristus je
od věky Bohem.
Ježíš je dočasně daným prostředkem vývojově dokázaným, to znamená: Je tam
ukázáno,
jakými proměnami ten prostředek, který nám dává k~dispozici, musí projít, aby se
člověk pomocí téhož prostředku dostal až na to nebe, byl tam  vzat, jak se říká, a ono je
to tak, že opravdu ten prostředek, který ze začátku vypadá jako malý dítě,
nemluvně, se
nám později jeví jako někdo úplně jiný. Za chvilinku je tesařem a za chvilinku je
učitelem a opět za ještě menší chvíli umírá na kříži.``
}

% 83-23A-K 39:42 -- manželství se světem
\textit{%
,,Ježíš Kristus navázal manželství jiné než my. Já myslím, že když se narodil na
tento svět, že mu nebylo jasno kým je. Měl předpoklady pro to, já vám řeknu
které -- ne tím, že byl avatarem -- aby brzo navázal nové, řekl bych spojení
nového v~uvozovkách spojení, protože to manželství s~Bohem naše je starší než...
ten náš původ od Otce je starší než jakýkoliv původ od čehokoliv jiného, co
mezitím bylo původem, také ale už odvozeným.``
}

\section{Soteriologie}

\textit{%
,,Jestliže Ježíš Kristus řekl: ,Já jsem cesta, pravda a život,` řekl, jakou
funkci Boží on zastává jako spasitel.``
}

\textit{%
,,Současně probíhá odtvořování. To znamená, že Bůh také všechno miluje. Všechno
stvořené miluje a k~sobě táhne. A věci dospěly tak daleko, že bylo třeba jim to
názorně ukázat. Že [je] to názorná ukázka toho spasitelského úkolu Božího, který
od začátku stvoření tu jest. Čili neberte to utrpení Ježíše Krista za věc, která
se jenom stala, nýbrž za něco, co ustavičně jest, a to je rozdíl. Když my jsme
byli svědky jenom toho, přímo nebo nepřímo, co se stalo, to je málo. Teprve když
člověk vnikne do nesmrtelné podstaty Boží, poznává, co jest, a ví tedy, že také
jest ta láska Boží, která tvoří a odtvořuje a samozřejmě osvěcuje. [...] A potom
samosebou se celé to utrpení Ježíše Krista dostává do nového světla. Ne že by
byl on nepotřeboval, proto že jenom to ukazoval, trpět méně. Kdepak? On
soustředěně na sebe bral v~té chvíli utrpení všeho stvořeného a soustředěněji
trpěl, než jak trpí člověk, který trpí jenom za sebe. On za sebe vůbec netrpěl.
On to nedělal, ani toto nedělal pro sebe! I tady platí jeho slova: ,Kdybych něco
dělal pro sebe, nevěřte mně.` On ani toto utrpení nedělal pro sebe. Všechno pro
nás. Bylo třeba... prostě věci dospěly tak daleko, že musel nám to ukázat. Ale
nic nového se nestalo. To všechno jest a trvá. To tehdy ukázal. Čili ještě
jednou jinými slovy: To, co Ježíš ukazuje, to je lidové podání pro lidi určené,
lidové podání zákona, věčného zákona, který funguje ve veškerém stvořeném. A ten
zákon zní: Kdyby Bůh nemiloval stvořené, vůbec by ho byl nestvořil. Protože ho
miluje, tak ho zároveň k~sobě táhne.  ,A tady vám ukazujeme,` říká pán Bůh, ,jak
to děláme, a kdokoliv z~vás se chce připojit vlivem tohoto zákona, který
existuje, k~tomu nestvořenému, tak to musí dělat tak jako to udělal Ježíš
Kristus. Nesmí z~toho nic vynechat. Musí také na ten kříž, musí také vstát
z~mrtvých a to musí udělat během toho pozemském života, tedy v rámci toho
stvořeného. Nikoliv někdy za zády stvořeného. Tady to musí udělat.``
}

% 83-11A 43:14
\textit{%
,,Ve mši svaté se modlíme: ,Vzal na sebe Ježíš veškeré naše hříchy, abychom byli
spaseni,` a to říkám ve zkratce. Jak je možný, aby jeden jediný tvor na sebe
vzal všechny hříchy všech lidí tak, aby je spasil? No tak na to právě není
rozumová odpověď, ale budete tomu přesto rozumět. Já se budu snažit z~obalu to
vysvětlit. Z~obalu, to znamená z~periferie. A to tímto způsobem: Jestliže třeba
dejme tomu už po těch sedmnácti letech, jak jsem získal vidění Boha... Vidění
v~uvozovkách, protože jsem žádnýho Boha neviděl, ale vědění o tom že moje
podstata je nesmrtelná a že jsem ji povinován především -- vším, celým svým
životem jsem poddán této nesmrtelné podstatě... Tak od tohoto okamžiku jsem měl
také sílu k~tomu, abych za touto podstatou šel, abych všechno ostatní opomíjel a
za tímhletím šel. A jestliže se člověk připojí k~Ježíši Kristu, k~jeho oběti na
kříži, to znamená když se nespolíhá, že by vlastní silou mohl to dosáhnout, to
spojení s~Bohem, nýbrž přes Ježíše Krista, tak se mu dostane síla k~tomu, aby to
uskutečnil. To jsme viděli u svaté Terezie, která se napřed vlastní silou
snažila dosáhnout vědomého spojení s~Bohem. A ono jí to nešlo. Až Ježíš Kristus
ji upozornil, že se musí připojit k~jeho oběti. To znamená, že se musí připojit
k~jeho svatosti. Že se musí připojit k~vůli Boží. Svatost spočívá v~připojení se
k~vůli Boží. A vůle Boží je, abychom byli s~ním nakonec úplně spojeni, že? A jak
ona se s~touto vůlí spojila, tak pochopitelně musela být spasena a musela mít
trvalé spojení s~Bohem. A potom když někdo, kdo je trvale spojen s~Bohem, vás
přivezeme k~sobě, tak jste spaseni. Ovšem ne tak, jak to bere církev. Vy se
musíte k~tomu aktu pasení přidat. Dobrovolně, ne? A nejlépe po nějaké přípravce.
Protože když se tam bez té přípravky dostanete, tak se dostanete maximálně do
ráje a z~toho zase spadnete. To není nic. To svatá Terezie měla velikou jogickou
přípravu, abych tak řekl, indickým stylem řečeno. A proto se mohla trvale stát
nositelkou poznání Božího a tak dále. Tak je možno, aby jeden člověk nebo jeden
člověk, ovšem spojený s Bohem, spasil... byl spasitelem celého světa, ovšem jen
těch z~toho světa, kdo se k~němu přidají. Dobrovolně k~němu přidají. Bez jejich
vůle, [které] oni se tou chvílí vzdávají, to není možné. Ta vůle stojí mezi nimi
a Bohem a [je] nepřekonatelná i pro Boha. Když tedy číňani dávno před Kristem
Indům vyčítali, že když chce někdo spasit sám sebe, že to není pravá spása, že
si něco nalhávají, protože porušují základní pravidlo, do kterého by se už byli
měli svým vývojem dostat, že se nelze oddělit od celku, nelze, že když se člověk
nevyrovná aspoň s~tím, že nepovytáhne svoje okolí, tak nemá sám právo. Nemá
[zkrátka] tam vstoupit. On musí organicky navázat na své okolí a šplhat se
s~ním, jo? Asi tak. A to ostatní, o to se starat nemusí, protože je to [ve]
vzájemné spojitosti všecko nemusí se starat, aby spasil celý svět.``
}

\section{Ekleziologie}

% 82-06 01:32:25
\textit{%
,,Jestli si třebas dejme tomu, co jste ještě vy jako církev neexistovali, to si
myslím, to je proti nějakým evangelíkům, ale když třebas se hádali v sedmém až
devátém století koncily, jestli jsou andělé mužskýho nebo ženského rodu. A když
se hádaly potom dvě století, jestli žena má duši a jestli to není věc. Prosím
vás, co to bylo za teology, co to bylo za úroveň duchovní. A koneckonců když
nikdy neměl právo mluvit ten, kdo byl duchovně na výši, nýbrž ten, kdo měl v
ruce moc, tak je tady zpráva, tak tímhletím jednáním církev na sobě prozradila,
že zavrhla Ježíše Krista. A zavrhla ho fakticky roku tři sta dvacet čtyři, když
dovolila vraždit a dovolila vlastnit. Těm tam končí křesťanství. A pracně to ti
prosťáčci partyzánsky dávali a kroutili dohromady. Takový svatý František
nevlastnil, že ano? A následkem toho si mohl vůbec troufat tomu Bohu se
přiblížit.``
}

% 83-13    27:31
\textit{%
,,Petr to tu v~ten moment poznával a tady nám Ježíš Kristus ukazuje, že celé
náboženství má svůj pravý základ v~tomto zjevení, které není z~rozumu. To je
z~něčeho vyššího. Proto mu říká: ,Ty jsi skála a na té skále postavím církev
svou.` Pro tyhlety závany zjevení nad rozumem ho právem ustanovil nástupcem. I
když ten Jan ho měl raději než třebas ten Petr, byl mu věrnější, to bylo vidět
pod tím křížem, když všichni utekli a on tam za ním šel, nebyl nástupcem.
Protože on potřeboval pro ten stav lidstva který bude následovat po jeho smrti,
aby v~čele té církve stál někdo, kdo je vrcholně chybující. To je to zlo, které
nesmí být vymýceno, a přitom přístupný zjevení. A to byl ten Petr: vrcholně
chybující. Nikdo se tolik (kromě Jidáše, to byl ale zrádce... toto zrádce
nebyl...) neproviňuje jako svatý Petr, že zapřel Ježíše třikrát, než jednou
kohout zakokrhal. To je vzor toho, ukázky toho, koho si to ten Ježíš Kristus
vyvolil. A my dneska se pozastavujeme třebas nad karamboly, který se staly
v~církvi během staletí a tisíciletí. I ta nevědomost, která tam vládla a třeba i
vládne, to by nás nemělo zastrašovat. To je nějakým způsobem v~plánu. Podívej,
tak například nezanevře na špatné následovníky dobrých příkladů, nýbrž bude si
vědom toho, že jsme křehké nádoby zhora až dolů a že je to v~záměru Božím,
abychom ten svůj koukol si z~milosti Boží podrželi do skonání věků. Protože
bychom jinak nevyužili správně dualismus.``
}

\section{Sakramentologie}

% 83-09    45:09
\textit{%
,,Samozřejmě tam ještě něco při té modlitbě, které my říkáme [u] mše svaté,
existuje něco, co my normálními smysly nemůžeme chápat. Ale co existuje
objektivně bez ohledu na nás. Já jsem jednou šel s~panem Válkem okolo kostela
v~Gottwaldově. A on nesledoval vůbec kudy jdeme. Protože já jsem ho vedl za
ruku, on se mě totiž držel pod paží, takže vůbec přestal sledovat, kde jsme. Ani
si neuvědomoval, že jsme před kostelem a on najednou říkal: ,Tadyhle vpravo se
něco děje, kde to jsme, Karle?` Já jsem říkal: ,Před kostelem.` A on říkal: ,A
počkáme chvíli. Mě to tak tady tahne doprava.` A ono tam zrovna bylo
pozdvihování a proměňování, ano? A slyšeli jsme za chvilinku ty zvonky, které
tehdy zněly přitom a on to proměňování cítil na vzdálenost třiceti metrů asi tak
přibližně. Tak to je něco objektivního, ta transsubstanciace se tam děje bez
ohledu na to, že ten kněz to dělá mechanickým způsobem. Prostě to jim bylo
vštípeno tím dědictvím po svatém Petru a tak dále.``
}

% 91-29B#ts=2290.23 38:10
\textit{%
,,Většina třebas katolíků, já budu mluvit jenom za ně, že mezi ně patřím, že jo?
svým narozením a svým křtem, jsou pověrčiví: Že si myslí, že konání obřadů všech
-- chodění na mši svatou, ke svátostem, takzvaný svátostný život [...] a tak
jestliže bych já považoval například ty svátosti za dostatečný prostředek
k~tomu, abych uskutečnil to, co se dá uskutečnit jenom vírou, tak bych se
dopouštěl nesmírné chyby. Protože bych zastupoval víru pověrou. Co je v~tom
pověrčivého, v těch svátostech? Že totiž svátost, i ta nejlepší, je jenom
obřadem a v~tom není poznání. I když může být v~tom nějaké požehnání, to může
být. Ale poznání samo o sobě v~tom není. Takže kdo plní všechny možné obřady, je
to v~pořádku, ale pokud se nedostavuje poznání spojení, málokdy se to dostaví,
to téměř nikdy ne, tak to byla pověra, kterou jsme neuspěli.``
}

% http://radio.makon.cz/zaznam/87-14A#ts=2047.82 34:07
\textit{%
,,Pro mě když já mluvím třeba s~katolickým knězem a říka mně: ,Tak co, ty si
proti těm svátostem?` A já jsem říkal: ,Vůbec ne, je to ale příliš formální.
Duch se z toho ztratil, vy musíte toho ducha do toho vložit, jinak po vás
všichni budou plivat a budou vás vraždit a budou vás shazovat z~toho oltáře.`
[...] % A víte, že už se přihlásili ke mně knězi, říkali: my se divíme, že ještě
%u toho oltářes měl být, když jsme četli tvoje spisy. Třebas e... to Sladké jho,
%to jim v hlavě zatopilo. Ale až do konce ho četli. Taky já nemám než první dva
%díly, a druhý... další tři nemám, protože to je všechno u kněží, tam uvízlo to,
%koluje mezi kněžstvem katolickým, protože oni se chytají za nos a vědí, že tam
%je třeba do toho dodat ducha.
Není to špatné, ale není v~tom duch, duch z toho vyprchal. Žádnou formou já se
nemohu například stát pokřtěným, k~tomu musím dorůst. Já mohu zahájit tím
formálním křtem, ale já musím to dodělat. Já jsem ten křest dodělal teprve
v~sedmnácti letech. A tak to musí jít se vším, co je v~té katolické, se všemi
svátostmi já musím dojít k~sobě a tam se s~nimi musím vypořádat.``
}

\section{Eschatologie}

% 90-05A 12:41
\textit{%
,,``
}
