\chapter{Vybraná témata v~pojetí Karla Makoně}

Při opakovaném poslechu Makoňových hovorů si všímám, že určitá sdělení se
neustále vracejí a ačkoliv je pokaždé trochu jiné, ve stěžejních bodech zůstává
stále totéž. Mohu uvést několik příkladů takových sdělení, se kterými se
opakovaně u Karla Makoně setkávám:

\begin{enumerate}
  \item{
      Cesta k Bohu se dělí na fáze. Podle toho, kde se nacházím, platí pro mě
      jiné zákonitosti.  Mezi fázemi jsou ostré předěly.
  }
  \item{
      Pro spojení s Bohem je zapotřebí vrcholné aktivity i absolutní pasivity.
  }
  \item{
      Modlitba nesmí být mechanická, musí být autentická.
      Prosebná modlitba má smysl jako prostředek na cestě.
      ,,Modlete se neustále`` dokazuje, že nejlepší modlitba není slovní, ba ani neznamená myslet na Boha.
  }
  \item{
    Ježíš bojoval proti osudovosti.
    Nejsme v rukou osudu, ale jsme vedeni událostmi ke spojení s Bohem.
  }
  \item{
    Satan je naše \textit{já}.
    \textit{Já} je potřeba rozvinout a přemoci zapřažením do úkolu.
    Světci byli pronásledováni Satanem, protože nechávali svoje \textit{já} zahálet.
  }
  \item{
    Ctnosti nestačí, naopak člověku brání v postupu, když se s nimi ztotožňuje.
    Ctnostný život nevede do nebe, pokud se ctnosti neopustí.
    Člověk musí být nic.
  }
  \item{
    Ráj je stav, ze kterého jsme padli a není našim cílem.
    Je zastávkou na cestě a oklikou.
    Ráj zažívají zvířata ve spánku.
    Pohlavní pud nebo touha po alkoholu jsou projevem touhy po ráji.
    Nebe je stav, ze kterého nelze vypadnout.
    Ve stavu nebe je člověk více činný než v běžném životě.
  }
  \item{
    Katolická tradice je nedoceněná, mělo by se na ni navázat, ale musí se korigovat.
  }
  \item{
    Do spojení s Věčností lze vstoupit jen celobytostným sebeodevzdáním.
  }
  \item{
    Boží milost je zákonitá, není nevyzpytatelná.
    Podmínky jsou plné rozvinutí hlavní složky života a její bezpodmínečné odevzdání.
  }
  \item{
    Stvoření neproběhlo, ale probíhá.
    Genesis hovoří o vzniku člověka po duchovní i tělesné stránce.
  }
  \item{
    Každý prostředek na cestě má omezenou platnost, včetně víry.
    Cíl musí být vždy nejvyšší, a to vědomé spojení s Bohem.
    Království Boží je stav, kdy v člověku kraluje Bůh.
  }
  \item{
    Krize je volání Boží.
  }
  \item{
    Apoštolové symbolizují lidské vlastnosti.
  }
  \item{
    Křesťanství překonává model reinkarnace tím, že umožňuje vyvázání ze samsáry během jediného vtělení.
    Nepřevtěluje se člověk z těla do těla, ale člověk navazuje a odhazuje karmické balíky.
    Tak lze být převtělením mnoha lidí najednou a jeden člověk se vtěluje do mnoha jiných.
    Indové se silou představy převtělují tradičně.
  }
  \item{
    Vlastnictví je největším hříchem.
    Máme spravovat, ne vlastnit.
    Ježíš učil především nevlastnit.
  }
  \item{
    Všechno utrpení je způsobeno nerovnováhou mezi vnitřním a vnějším životem.
  }
  \item{
    Láska je jediným prostředkem, který nelze překonat.
    Cesta lásky je jediná, která může být úspěšná i bez krizí.
    Láska člověka vyvádí z něho samého.
    Symbolem lásky byl apoštol Jan.
    Láska je otázkou vůle.
  }
  \item{
    Bible je dokonalým, podrobným symbolickým plánem na cestu k Bohu pro každého člověka.
    Ježíšův život je obecným vzorcem (a² + b²) a naše životy jsou aplikací vzorce.
  }
  \item{
    Panna Maria symbolizuje nesmrtelnou duši.
    Maria jediná člověka nikdy neopouští (byla i pod křížem).
  }
  \item{
    Smrt na kříži symbolizuje mystickou smrt, která musí proběhnout před smrtí fyzickou a je poslední podmínkou vstupu do věčnosti.
  }
  \item{
    Smrt je svlečením hmotného těla a oblečením astrálního.
    Mrtví žijí na úrovni druhé čakry, tamtéž jako my když sníme.
  }
  \item{
    Krom verbálního existuje i obsahové sdělování.
    Probíhá \textit{en bloc} a napříč jazyky.
    Podmínkou je zastavení povídavé mysli při zachování vědomí.
  }
  \item{
    Extáze je překonaným fenoménem.
    Budoucnost patří cestě bez extází.
    Extáze je nejvyšší stav, kam lze dojít vlastní silou.
  }
  \item{
    Míra sebezáporu (později odosobnění) je měrou duchovního pokroku.
  }
\end{enumerate}

Dojem o konzistentně se opakujících poselstvích vyvolává otázku, do jaké míry se
témata skutečně opakují a jak se jejich pojetí vyvíjí v~čase. U několika témat
se pokusím tuto otázku zodpovědět.

\section{Metodologie}

Vycházím z~kompletně přepsaných zvukových záznamů Makoňových hovorů, kde zčásti
se jedná o přepisy manuální prakticky bez chyb a zčásti automatické s~mírou
chybovosti úměrné akustické kvalitě záznamu, běžně se však pohybující kolem 90\%
správně přepsaných slov. Nutno také podotknout, že i chybně přepsaná slova mají
mnohdy správně určený kmen a pro účely vyhledávání je tedy můžeme považovat za
správně přepsané. Krom toho využívám manuálně vytvořený tematický index
pokrývající v~dubnu 2022 všechny písmenem označené kotouče a kazety označené
ročníkem do osmdesátého pátého, celkem 312 z~1138. Záznam je ke každému úseku
s~konzistentním tématem.

Postupuji v~následujících krocích:
Projdu záznamy tematického indexu a ke každému přiřadím jedno či více
obecnějších témat, do kterých zázam spadá.
Témata s~více než pěti přiřazenými záznamy procházím a pro každé provedu
následovné:
Určím klíčová slova pro fulltextové vyhledávání, která předpokládám, že povedou
k~úsekům, kde se dané téma pojednává.
Klíčová slova vyhledám v~celém přepisu a setřídím si nahrávky podle počtu
výskytů.
Projdu nahrávky s~velkým počtem výskytů tak, aby pokryly maximální časové
období, preferuji rozestup kolem dvou až tří let mezi dobou pořízení.
Z~každé vyberu krátký úsek (jednu či několik vět), který vyjadřuje mluvčího
postoj k~tématu, je-li to možné.
Nenacházím-li takové úseky, uchyluji se k~nahrávkám s~menším počtem výskytů, a
pokud ani tam nenacházím, téma vyřazuji ze seznamu.
Na vybraných výrocích pak vyhodnocuji, v~čem se liší a co mají naopak
společného.

Tímto postupem se snažím vyvážit časově efektivní využití dostupných materiálů a
přitom omezit zbytečnou subjektivitu. Subjektivita zde vstupuje i tak v~několika
bodech. Samotný tematický index byl tvořen ručně, takže sám o sobě zahrnuje
subjektivní interpretaci. Přiřazování záznamů do obecnějších témat je dalším
subjektivním krokem. Výběr klíčových slov pro vyhledávání je také potenciální
branou subjektivity. Naposledy výběr charakteristických úseků je krokem
s~největší měrou subjektivního zásahu. Aby se subjektivita redukovala, musel by
se postup opakovat několika různými badateli, což je ovšem vysoko nad rámec
možností této diplomové práce. Subjektivitu v~postupu tedy identifikuji a
přijímám jako součást práce, snažím se ji však izolovat a minimalizovat.

Některé manuální kroky by bylo možné nahradit automatickými postupy počítačového
zpracování přirozeného jazyka, ovšem za cenu neúměrného zvýšení náročnosti
provedení.

\section{Téma trpnosti}

%utrpení:
%rovnováha mezi vnitřním a vnějším životem;
%nevědomost;
%nutnost odevzdání

%karma

%dobro a zlo

%zákonitost milosti Boží
