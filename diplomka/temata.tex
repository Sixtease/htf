\chapter{Vybraná témata v~pojetí Karla Makoně}

Při opakovaném poslechu Makoňových hovorů si všímám, že určitá sdělení se
neustále vracejí a ačkoliv je pokaždé trochu jiné, ve stěžejních bodech zůstává
stále totéž. Mohu uvést několik příkladů takových sdělení, se kterými se
opakovaně u Karla Makoně setkávám:

\begin{enumerate}
  \item{
      Cesta k Bohu se dělí na fáze. Podle toho, kde se nacházím, platí pro mě
      jiné zákonitosti.  Mezi fázemi jsou ostré předěly.
  }
  \item{
      Pro spojení s Bohem je zapotřebí vrcholné aktivity i absolutní pasivity.
  }
  \item{
      Modlitba nesmí být mechanická, musí být autentická.
      Prosebná modlitba má smysl jako prostředek na cestě.
      ,,Modlete se neustále`` dokazuje, že nejlepší modlitba není slovní, ba ani neznamená myslet na Boha.
  }
  \item{
    Ježíš bojoval proti osudovosti.
    Nejsme v rukou osudu, ale jsme vedeni událostmi ke spojení s Bohem.
  }
  \item{
    Satan je naše \textit{já}.
    \textit{Já} je potřeba rozvinout a přemoci zapřažením do úkolu.
    Světci byli pronásledováni Satanem, protože nechávali svoje \textit{já} zahálet.
  }
  \item{
    Ctnosti nestačí, naopak člověku brání v postupu, když se s nimi ztotožňuje.
    Ctnostný život nevede do nebe, pokud se ctnosti neopustí.
    Člověk musí být nic.
  }
  \item{
    Ráj je stav, ze kterého jsme padli a není našim cílem.
    Je zastávkou na cestě a oklikou.
    Ráj zažívají zvířata ve spánku.
    Pohlavní pud nebo touha po alkoholu jsou projevem touhy po ráji.
    Nebe je stav, ze kterého nelze vypadnout.
    Ve stavu nebe je člověk více činný než v běžném životě.
  }
  \item{
    Katolická tradice je nedoceněná, mělo by se na ni navázat, ale musí se korigovat.
  }
  \item{
    Do spojení s Věčností lze vstoupit jen celobytostným sebeodevzdáním.
  }
  \item{
    Boží milost je zákonitá, není nevyzpytatelná.
    Podmínky jsou plné rozvinutí hlavní složky života a její bezpodmínečné odevzdání.
  }
  \item{
    Stvoření neproběhlo, ale probíhá.
    Genesis hovoří o vzniku člověka po duchovní i tělesné stránce.
  }
  \item{
    Každý prostředek na cestě má omezenou platnost, včetně víry.
    Cíl musí být vždy nejvyšší, a to vědomé spojení s Bohem.
    Království Boží je stav, kdy v člověku kraluje Bůh.
  }
  \item{
    Krize je volání Boží.
  }
  \item{
    Apoštolové symbolizují lidské vlastnosti.
  }
  \item{
    Křesťanství překonává model reinkarnace tím, že umožňuje vyvázání ze samsáry během jediného vtělení.
    Nepřevtěluje se člověk z těla do těla, ale člověk navazuje a odhazuje karmické balíky.
    Tak lze být převtělením mnoha lidí najednou a jeden člověk se vtěluje do mnoha jiných.
    Indové se silou představy převtělují tradičně.
  }
  \item{
    Vlastnictví je největším hříchem.
    Máme spravovat, ne vlastnit.
    Ježíš učil především nevlastnit.
  }
  \item{
    Všechno utrpení je způsobeno nerovnováhou mezi vnitřním a vnějším životem.
  }
  \item{
    Láska je jediným prostředkem, který nelze překonat.
    Cesta lásky je jediná, která může být úspěšná i bez krizí.
    Láska člověka vyvádí z něho samého.
    Symbolem lásky byl apoštol Jan.
    Láska je otázkou vůle.
  }
  \item{
    Bible je dokonalým, podrobným symbolickým plánem na cestu k Bohu pro každého člověka.
    Ježíšův život je obecným vzorcem (a² + b²) a naše životy jsou aplikací vzorce.
  }
  \item{
    Panna Maria symbolizuje nesmrtelnou duši.
    Maria jediná člověka nikdy neopouští (byla i pod křížem).
  }
  \item{
    Smrt na kříži symbolizuje mystickou smrt, která musí proběhnout před smrtí fyzickou a je poslední podmínkou vstupu do věčnosti.
  }
  \item{
    Smrt je svlečením hmotného těla a oblečením astrálního.
    Mrtví žijí na úrovni druhé čakry, tamtéž jako my když sníme.
  }
  \item{
    Krom verbálního existuje i obsahové sdělování.
    Probíhá \textit{en bloc} a napříč jazyky.
    Podmínkou je zastavení povídavé mysli při zachování vědomí.
  }
  \item{
    Extáze je překonaným fenoménem.
    Budoucnost patří cestě bez extází.
    Extáze je nejvyšší stav, kam lze dojít vlastní silou.
  }
  \item{
    Míra sebezáporu (později odosobnění) je měrou duchovního pokroku.
  }
\end{enumerate}

Dojem o konzistentně se opakujících poselstvích vyvolává otázku, do jaké míry se
témata skutečně opakují a jak se jejich pojetí vyvíjí v~čase. U několika témat
se pokusím tuto otázku zodpovědět.

\section{Metodologie}

Vycházím z~kompletně přepsaných zvukových záznamů Makoňových hovorů, kde zčásti
se jedná o přepisy manuální prakticky bez chyb a zčásti automatické s~mírou
chybovosti úměrné akustické kvalitě záznamu, běžně se však pohybující kolem 90\%
správně přepsaných slov. Nutno také podotknout, že i chybně přepsaná slova mají
mnohdy správně určený kmen a pro účely vyhledávání je tedy můžeme považovat za
správně přepsané. Krom toho využívám manuálně vytvořený tematický index
pokrývající v~dubnu 2022 všechny písmenem označené kotouče a kazety označené
ročníkem do osmdesátého pátého, celkem 312 z~1138. Záznam je ke každému úseku
s~konzistentním tématem.

Postupuji v~následujících krocích:
\begin{itemize}
\item{
Projdu záznamy tematického indexu a ke každému přiřadím jedno či více
obecnějších témat, do kterých záznam spadá.
}
\item{
Témata s~více než pěti přiřazenými záznamy procházím a pro každé provedu
následovné:
\begin{itemize}
\item{
Určím klíčová slova pro fulltextové vyhledávání, která předpokládám, že povedou
k~úsekům, kde se dané téma pojednává.
}
\item{
Klíčová slova vyhledám v~celém přepisu a setřídím si nahrávky podle počtu
výskytů.
}
\item{
Projdu nahrávky s~velkým počtem výskytů tak, aby pokryly maximální časové
období, preferuji rozestup kolem dvou až tří let mezi dobou pořízení.
}
\item{
Z~každé vyberu krátký úsek (jednu či několik vět), který vyjadřuje mluvčího
postoj k~tématu, je-li to možné.
}
\item{
Nenacházím-li takové úseky, uchyluji se k~nahrávkám s~menším počtem výskytů, a
pokud ani tam nenacházím, téma vyřazuji ze seznamu.
}
\item{
Na vybraných úsecích pak vyhodnocuji, v~čem se liší a co mají naopak
společného.
}
\end{itemize}
}
\end{itemize}

Tímto postupem se snažím vyvážit časově efektivní využití dostupných materiálů a
přitom omezit zbytečnou subjektivitu. Subjektivita zde vstupuje i tak v~několika
bodech. Samotný tematický index byl tvořen ručně, takže sám o sobě zahrnuje
subjektivní interpretaci. Přiřazování záznamů do obecnějších témat je dalším
subjektivním krokem. Výběr klíčových slov pro vyhledávání je také potenciální
branou subjektivity. Naposledy výběr charakteristických úseků je krokem
s~největší měrou subjektivního zásahu. Aby se subjektivita redukovala, musel by
se postup opakovat několika různými badateli, což je ovšem vysoko nad rámec
možností této diplomové práce. Subjektivitu v~postupu tedy identifikuji a
přijímám jako součást práce, snažím se ji však izolovat a minimalizovat.

Některé manuální kroky by bylo možné nahradit automatickými postupy počítačového
zpracování přirozeného jazyka, ovšem za cenu neúměrného zvýšení náročnosti
provedení.

\section{Téma trpnosti}

Trpnost čili pasivitu vykládá Makoň jako jeden z~nejdůležitějších prvků, které
chybějí na cestě jeho posluchačů do vědomého spojení s~Bohem. Ze základního
předpokladu, že člověk má za cíl dojít do Království Božího, mu vyplývá trpnost
jako princip, který může chybět v~dosahování jiných cílů, a jeho zapojení tedy
není samozřejmé, pročež ho vyzdvihuje.

Použité vyhledávací vzory: \texttt{aktiv*}, \texttt{trpn*}.

\begin{enumerate}
  \item{
    Nahrávka s~označením \texttt{kotouc-D01-b}. Rok pořízení neznámý, ale
    jistě v~sedmdesátých letech.

    Citace: \textit{%
      ,,My totiž tady v tomto životě cokoliv dobýváme, tak vždycky to dobýváme
      nějakou aktivitou, že totiž si to přejeme, jdeme za tím, dosahujeme to,
      případně nedosahujeme. A jakou to činnost? Jak ji máme klasifikovat,
      takovou činnost? To je vlastně pokračování v tvůrčí činnosti Boží. My tady
      rozmnožujeme statky, které jsou nám od Boha dány, a dále prostě pro ně
      žijeme a dále jdeme do stvořeného, do zmnožného. Jde ta cesta z
      jednoduchosti do množství, do stále větší šíře, ano? To je, to je
      tenhleten, ta aktivita. Tímto způsobem když člověk přistupuje i k
      duchovním úkolům, to znamená k tomu, aby se stal nezrozeným, věčným to
      znamená nezrozeným, věčným, nestvořeným, tak když tímto způsobem
      postupuje, tak nikdy se jím nestane, i když se sebe pilněji snaží, protože
      mu chybí tahleta zpětná vazba, ta pasivita. On zná jenom ten aktivní
      způsob a ten vlastně jde úplně opačným směrem než ta trpnost mystická,
      která je zapotřebí stejnou měrou jako ta aktivita a kterou jedinou se
      můžete dostat nazpátek.``
    }

    Jedná se o fundamentální výrok o trpnosti, kde Karel Makoň kompaktně uvádí,
    proč ji považuje za důležitou a doplňuje o letmé nastínění principů, které
    ji důležitou činí.
  }
  \item{
    Nahrávka s~označením \texttt{kotouc-I01-d}. Rok pořízení rovněž neznámý, ale
    spadající do sedmdesátých let.

    Citace: \textit{%
      ,,Ve skutečnosti každý člověk na všechno hledí aktivně, i na mystickou
      modlitbu, i na modlitbu, která má ho spojit s Bohem a dopadá to potom
      příšerně. Podívejte se na všechny žáky Weinfurtrovy, kdy oni říkali
      nějaká písmenová cvičení, já proti tomu nic nemám, ale mysleli si že
      tímhletím dobydou království Boží. Jo, království Boží násilí trpí,
      násilím je dobýváno, ale ne dobyto. Dobýváno. A oni u toho dobývání
      zůstali, museli zůstat protože nikdo jim neřekl, že modlitba se musí
      stávat postupně stále více trpnou. Trpnější. Pasivní. A jestliže tento
      požadavek není splněn, tak nemůže vést ke spojení, vědomému spojení s
      Bohem.``
    }

    Příklad školy Karla Weinfurtera z~pohledu Karla Makoně je užita pro
    demonstraci nezbytnosti trpného přístupu. Je zde postulována nezbytnost
    trpnosti pro vstup do Království Božího.
  }
  \item{
      Nahrávka s~označením \texttt{kotouc-77-brezen01-b}. Rok pořízení 1977.

      Citace: \textit{%
        ,,Trpná stránka, to je stránka koncentrační. Koncentrace není činnost v
        tom běžném slova smyslu, k jaké já vás tady navádím v těchto
        algoritmech. To není ještě koncentrace vůbec. Todleto, co já- k čemu vás
        já tady navádím to je nanejvýš meditace, jo? A z ní pocházející čin, čin
        prostý anebo čin láskyplný, ano? Ale koncentrace mystická, ne
        soustředění pouhé, konaná za účelem spojovacím, mající tento cíl, to je
        vyhraněně trpná záležitost. To znamená, já musím se jenom tak dlouho
        koncentrovat, smím lépe řečeno se tak dlouho koncentrovat, dokud
        zůstávám trpný, ale ne civějící.``
      }

      I v~tomto úseku ze sedmdesátých let je důraz kladen na nezbytnost trpnosti
      na cestě ke spojení s~Bohem, tentokrát v~rámci mystické koncentrace.
      Objevuje se zde i pojem civění, který se v~souvislosti s~trpností objevuje
      u Makoně často a který značí trpnost tedy pasivitu nežádoucího druhu.
  }
  \item{
      Nahrávka s~označením \texttt{80-11A}. Rok pořízení 1980.

      Citace: \textit{%
        ,,Musím se snažit o to, milovat z celého srdce a tak dále, jak je to tam
        všechno řečeno, ale my jsme si už řekli, že pochopit, kde jsou meze toho
        a kde a jak máme Bohu dát příležitost, aby on nám ukázal svou lásku, to
        že už není, co se dá vyjádřit tím slovem, nýbrž to se dá vyjádřit jedině
        příkladem Ježíšovým. Celým jeho životem.``
      }

      Zde není důraz kladen na potřebnost trpnosti jako takovou, nýbrž na to, že
      najít správné meze mezi činností a trpností není otázkou formulovatelného
      receptu, nýbrž otázkou pochopení příkladu Ježíšova života jako celku.
  }
  \item{
      Nahrávka s~označením \texttt{83-18-K}. Rok pořízení 1983.

      Citace: \textit{%
        ,,Už to narození Ježíše Krista vysvětluje značnou míru trpnosti, ale v
        které oblasti, to je mi jasné: Ne všude. Například jakápak je to
        trpnost, když utečou do Egypta? To je velká činnost, ale trpnost je v
        tom, že poslouchali. Vy musíte přesně vědět, že to vedle sebe má
        existovat a že to má své místo, to i ono.``
      }

      Jako v~předchozím úseku je důraz kladen na nalezení mezí aktivity a
      pasivity a navíc na legitimitu obou pólů.
  }
  \item{
      Nahrávka s~označením \texttt{85-25A-10}. Rok pořízení 1985.

      Citace: \textit{%
        ,,Láska má dvě stránky: Aktivní a trpnou, ano? To jsem říkal všechny
        čtyři dni. Ne že je to jenom trpná stránka, ale že trpná stránka je
        dovršení oné aktivní stránky lásky. A kdyby to nebylo dovršení aktivní
        stránky, nebylo by co dovršovat a bylo by zbytečné ta negace nebo ta
        trpnost v lásce. By byla bezvýsledná, to by nikam nevedlo.``
      }

      V~tomto úseku pozorujeme ještě silnější odklon on pouhého zdůrazňování
      trpnosti. Důraz je naopak kladen na to, že trpnost, která nenavazuje na
      činnost, nemá význam.
  }
  \item{
      Nahrávka s~označením \texttt{prevzate-kotouc-88-03}. Rok pořízení 1988.
      Uvádím dva úseky.

      První citace: \textit{%
        ,,Vznik i té nejmenší metafyzické zkušenosti je založen na trpnosti.``
      }

      Druhá citace: \textit{%
        ,,Trpnost musí být musí mít protiváhu, že ji dovedete uplatit, aktivně
        uplatnit. Ježíš Kristus to doved aktivně uplatnit, že byl poddán trpně
        třebas své rodině a tak dále. On, který měl tu obrovskou moudrst za
        sebou, že, a v sobě, který... Takže on to potom už začal nám ukazovat
        jako míchanou trpnost s~aktivitou. A to je správně pojatý život. To má
        být tak namícháno, že když je zapotřebí být aktivní, tak musím být
        aktivní až do dřeně kosti. Když je třeba být trpný, tak musím být trpný
        zase tam.``
      }

      V~první citaci se jedná o postulát nezbytnosti trpnosti, jak jsme viděli u
      raných výroků. Druhá citace opět zdůrazňuje potřebnost obou pólů.
  }
  \item{
      Nahrávka s~označením \texttt{91-11B}. Rok pořízení 1991. Opět uvádím dva
      úseky.

      První citace: \textit{%
        ,,Kdo neumí být trpný, tak není zasvěcencem, a kdo se bojí trpnosti, tak
        neví, že se bojí něčeho, čeho by se neměl bát. To mu umožňuje stát se
        zasvěcencem a být spojen vědomě s~Bohem.``
      }

      Druhá citace: \textit{%
        ,,Ani Ježíš Kristus neudělal první zázrak, a kdyby byl neudělal první
        zázrak za pomoci Panny Marie, tak neudělal ani další zázraky. Začal to
        dělat jenom proto, že ho Panna Maria požádala. Kde pak jsme s~tou
        méněcennosti ženy? Prosím vás, odpovězte mně na to. A protože tomu tak,
        že ona je cennější a v~té trpnosti, která je jí vrozená, je veliká
        přednost a říkáme jí, té přednosti, metafyzická zkušenost, která se bez
        trpnosti, bez stavu trpnosti neobejde, tak my mužové musíme také tuto
        trpnost zvládnout. My ji musíme umět navodit, stejně jako ji umí snáze
        navodit žena.``
      }

      Stejně jako u předchozí nahrávky první úsek postuluje nezbytnost trpnosti.
      Chybí však důraz na potřebnost činnosti. Místo toho uvádí vztah trpnosti
      s~ženským principem.
  }
\end{enumerate}

Nebyl jsem bohužel s~to vybrat nahrávky, které by byly spolehlivě rozdělené
v~kýžených časových rozestupech před rokem 1980, nicméně odstup v~pořadovém
označení kotoučů napovídá, že snad nebyly pořízeny bezprostředně po sobě.

Z~ukázek jasně plyne, že Karel Makoň napříč celým svým zaznamenaným přednáškovým
působením tvrdí, že trpnost je nezbytným prvkem na cestě ke spojení s~Věčností.
V~některých případech je nezbytnost trpnosti podána v~rámci výroku o nezbytnosti
jak činnosti tak trpnosti. Nabízí se hypotéza, že v~raných osmdesátých letech
kladl Makoň důraz na legitimitu činnosti, ovšem tu bychom museli ověřit na
větším vzorku. S~jistotou však můžeme tvrdit, že o nutnosti trpnosti na duchovní
cestě poučoval Makoň vcelku konzistentně po dobu aspoň dvou dekád.

\section{Téma modlitby}

Použitý vyhledávací vzor: \texttt{modlit*}.

%utrpení:
%rovnováha mezi vnitřním a vnějším životem;
%nevědomost;
%nutnost odevzdání

%karma

%dobro a zlo

%zákonitost milosti Boží
