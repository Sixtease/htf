\chapter{Úvod}
\label{kap:uvod}

\section{Ústava a zákony}

Tématem mojí práce je porovnání úpravy vzniku, změn a zániku náboženské obce
podle ústavy oproti organizačnímu řádu Církve československé husitské. Proč asi
právě toto téma bylo jedním z mála navrhovaných? Jistě abych si kvůli tomu musel
nastudovat příslušné pasáže naší ústavy a jednoho z řádů a provedl tak cvičnou
práci s cirkevněprávním textem. Krom toho mi ale toto zadání ukládá zamyslet se
nad vztahem ústavy a podřízeného zákona.

To druhé považuji za mnohem zásadnější. Cvik v práci s legislativním textem je
totiž dovedností, zatímco uvědomění si vztahu ústavy a ostatních zákonů je
náhledem. Obzvlášť v době, kdy se pod rouškou ochrany veřejného zdraví prosazují
nařízení odporující zákonům a ústavě obzvlášť a výkonné orgány je vymáhají s
nelidskostí sobě vlastní, kdy se pod rouškou z bílé látky ustrašeně chmuří tváře
nás všech, je potřeba mít na paměti, co vlastně naše zákony a naše ústava
dovolují a zaručují a do jaké míry se na to dbá či nikoliv. Kdyby nic jiného,
může to ukázat, jakou hodnotu dobře sestavená ústava představuje, a s jakou
vervou ji musíme ochraňovat a jak moc se nás týká.

Vznik, změna a zánik náboženské obce v Církvi československé husitské není na
rozdíl od segregace očkovaných a neočkovaných nikterak kontroverzním tématem,
proto moje expedice do hlubin této droboučké úpravy je příkladem vztahu
bezproblémového, což je jistě dobrý začátek.

\section{Náboženská obec v~CČSH}

Pojem náboženská obec je protestantským ekvivalentem katolické farnosti.
V~angličtině se obojímu říká \textit{parish}, což je slovo stejného původu jako
\textit{farnost}, a sice z~řeckého {$\pi\alpha\rho$\textit{o}$\iota\kappa\iota\alpha$} přes latinské
\textit{paroecia}, s~původním významem \textit{,,pobývání v~cizí zemi``}.
Samotný výraz tedy implikuje propojení s~územím.
Historická souvislost farnosti s~územím je patrná převážně ze středověku, kdy
namnoze přesně odpovídala panství.
V~době, kdy bylo možno
člověka mocí přiřadit k~jistému území a též ho přinutit k~účasti na
bohoslužebném životě v~konkrétním kostele u konkrétního kněze, to bylo zcela
přirozené.

Z~praxe dnešního církevního života je však zřejmé, že faktické určení náboženské
obce skutečně více odpovídá termínu \textit{obec}. Jde totiž o obecenství,
vlastně množinu lidí, kteří se k~obci hlásí, ať již formálním zápisem, aktivní
účastí na náboženském a praktickém dění či účastí finanční.

Pro vstup do obce se nekladou žádné požadavky na bydliště a nezřídka jsou členy
obce i lidé z~daleka či ze sousedních měst vesnic, které spadají místně pod
jinou farnost, ovšem osobní preference či zvyk jim velí náležet ke svojí
vzdálenější obci. Místní vymezení se tak v~mnoha případech stává jakýmsi
formálním, imaginárním určením, vlastně jistým doporučením pro obyvatele okolí,
do které obce se přidat.

Ústava Církve československé husitské dle mého názoru moudře definuje v~bodě 2
článku 6 jako právnickou osobu tvořenou jejími členy, avšak dodává, že tito mají
zpravidla bydliště v~územním obvodu obce.

Jsem toho názoru, že pojetí náboženské obce jako skupiny lidí, kteří spolu
chtějí sdílet náboženský život navzdory geografické vzdálenosti, je dnes
vitálnější než kdy jindy. Jednak proto, že lidé jsou čím dál mobilnější, jednak
proto, že vzrůstající individualismus má lidi k~tomu, aby pečlivěji vyhledávali
jiné lidi, kteří mají podobné smýšlení, a spolčovat se s~ostatními, které si
sami nevybrali, jim je čím dál více proti mysli. S~tím koreluje i fenomén
vzrůstající anonymity v~městech, ale i v~obcích, kdy lidé, kteří spolu sousedí
třeba přes tenkou panelákovou zeď, se vůbec neznají.

Neříkám, že tento jev je žádoucí a že je vhodné ho podporovat, ale je třeba ho
vzít na vědomí. Ačkoliv by třeba bylo zdravější, aby se k~uctívání Hospodina
lidé scházeli do svojí lokální obce, nelze to vynutit, pročež koncept obce
bez vazby k~bydlišti členů považuji za vhodný zvážení. Nabízí se otázka, je-li
vůbec co zvažovat, vzhledem k~tomu, že takové obce se mohou již teď organicky
tvořit. Domnívám se, že je, neboť mezi samovolnou tvorbou alokálních obcí a
záměrným jejich tvořením je přece jenom rozdíl.

\chapter{Porovnání}

Vznik, územní změny a zánik náboženské obce upravuje článek 17 Ústavy a §6 a §14
-- §17 Organizačního řádu. První pozorování je, že dokumenty na sebe vzájemně
odkazují. Z~hlediska softwarového architekta je toto pro mne kuriózní zjištění.
Cyklické závislosti mám zapsané jako fenomén, kterému se je záhodno vyhnout,
je-li to jen trochu možné, neboť přináší značnou komplexitu do výsledného
významu a zvyšuje riziko chyb.

Celkově ale Ústava odkazuje k~Organizačnímu řádu jen v~tom smyslu, že upozorňuje
na jeho přítomnost a platnost. Konkrétně z~něj necituje. Naopak organizační řád
se hned v~bodě 2 článku 6 odkazuje ke konkrétní pasáži Ústavy a doplňuje ji,
aniž by její znění v~plnosti citoval. Toto mi opět z~pohledu programátora
připadá ošidné. V~Organizačním řádu totiž nenacházím explicitní určení verze
Ústavy, ke které se odkazuje. Když se tedy odvolává na termíny zavedené
v~určitém bodě Ústavy, vyvstává nebezpečí, že čtenář bude mít v~ruce od doby
vzniku OŘ novelizovanou Ústavu, kde např. daný pojem bude mít jiný význam nebo
bude definován v~jiném bodě. Tomuto částečně zabraňuje fakt, že součástí
Organizačního řádu je hlavička udávající datum poslední úpravy samotného OŘ, ale
bez explicitního uvedení verze Ústavy je párování nejisté. Samozřejmě že změna
Ústavy je aktem velmi zřídkavým, ale to na principielním problému nic neubírá,
obzvlášť když jeho řešení je tak snadné jako uvedení vydání Ústavy, na niž se
řád odvolává nebo vložení plného znění citace.

Ve věci založení NO Organizační řád blíže určuje podmínky pro platnost, jako
minimální počet členů výboru, komu a v~jakých lhůtách se podává odvolání atp.
Stejně tak Organizační řád přidává délku funkčního období orgánů náboženské
obce. Je zde, konkrétně v~bodě 3 článku 6 OŘ, použit pro mne poněkud matoucí
výraz ,,první orgány NO``. Co jsou první orgány? Článek 19 Ústavy jmenuje
,,základní orgány NO`` -- je to totéž? Nebo se myslí základní orgány od první do
druhé volby jejich členů?

§14 - §17 OŘ mají pozoruhodnou podobu. Splynutí, rozdělení a zánik NO jsou
pojednány společně v~§17, zatímco §14, §15 a §16 každý tento fakt konstatují,
každý pro jedno ze jmenovaných témat. V~tomto naopak z~hlediska programátora
spatřuji dobrou praxi, neboť se zamezuje repetitivnosti, ke které by došlo,
kdyby se témata pokrývala v~oddělených paragrafech, ale zároveň je pro potřebu
vyhledávání, strukturování a indexace každému z~témat vyhrazen zvláštní
paragraf.

\chapter{Závěr}


