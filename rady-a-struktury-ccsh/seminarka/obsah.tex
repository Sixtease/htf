\chapter{Úvod}
\label{kap:uvod}

\section{Ústava a zákony}

Tématem mojí práce je porovnání úpravy vzniku, změn a zániku náboženské obce
podle ústavy oproti organizačnímu řádu Církve československé husitské. Proč asi
právě toto téma bylo jedním z mála navrhovaných? Jistě abych si kvůli tomu musel
nastudovat příslušné pasáže naší ústavy a jednoho z řádů a provedl tak cvičnou
práci s cirkevněprávním textem. Krom toho mi ale toto zadání ukládá zamyslet se
nad vztahem ústavy a podřízeného zákona.

To druhé považuji za mnohem zásadnější. Cvik v práci s legislativním textem je
totiž dovedností, zatímco uvědomění si vztahu ústavy a ostatních zákonů je
náhledem. Obzvlášť v době, kdy se pod rouškou ochrany veřejného zdraví prosazují
nařízení odporující zákonům a ústavě obzvlášť a výkonné orgány je vymáhají s
nelidskostí sobě vlastní, kdy se pod rouškou z bílé látky ustrašeně chmuří tváře
nás všech, je potřeba mít na paměti, co vlastně naše zákony a naše ústava
dovolují a zaručují a do jaké míry se na to dbá či nikoliv. Kdyby nic jiného,
může to ukázat, jakou hodnotu dobře sestavená ústava představuje, a s jakou
vervou ji musíme ochraňovat a jak moc se náš týká.

Vznik, změna a zánik náboženské obce v Církvi československé husitské není na
rozdíl od segregace očkovaných a neočkovaných nikterak kontroverzním tématem,
proto moje expedice do hlubin této droboučké úpravy je příkladem vztahu
bezproblémového, což je jistě dobrý začátek.

\section{Náboženská obec v~CČSH}

Pojem náboženská obec je protestantským ekvivalentem katolické farnosti.
V~angličtině se obojímu říká \textit{parish}, což je slovo stejného původu jako
\textit{farnost}, a sice z~řeckého {$\pi\alpha\rho$\textit{o}$\iota\kappa\iota\alpha$} přes latinské
\textit{paroecia}, s~původním významem \textit{,,pobývání v~cizí zemi``}.
Historická souvislost farnosti s~územím je patrná převážně ze středověku, kdy
namnoze přesně odpovídala panství.

V~době, kdy bylo možno
člověka mocí přiřadit k~jistému území a též ho přinutit k~účasti na
bohoslužebném životě v~konkrétním kostele u konkrétního kněze, to bylo zcela
přirozené.

Z~praxe dnešního církevního života je však zřejmé, že faktické určení náboženské
obce skutečně více odpovídá termínu \textit{obec}. Jde totiž o obecenství,
vlastně množinu lidí, kteří se k~obci hlásí, ať již formálním zápisem, aktivní
účastí na náboženském a praktickém dění či účastí finanční.

Pro vstup do obce se nekladou žádné požadavky na bydliště a nezřídka jsou členy
obce i lidé z~daleka či ze sousedních měst vesnic, které spadají místně pod
jinou farnost, ovšem osobní preference či zvyk jim velí náležet ke svojí
vzdálenější obci. Místní vymezení se tak v~mnoha případech stává jakýmsi
formálním, imaginárním určením, vlastně jistým doporučením pro obyvatele okolí,
do které obce se přidat.

Ústava Církve československé husitské dle mého názoru moudře definuje v~bodě 2
článku 6 jako právnickou osobu tvořenou jejími členy, avšak dodává, že tito mají
zpravidla bydliště v~územním obvodu obce.

\chapter{Porovnání}



\chapter{Závěr}


