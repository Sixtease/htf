\chapter{Úvod}
\label{kap:uvod}

\section{Vymezení pojmu svátost}
\label{div:vymezeni}

Co je vlastně svátost? Jak tomu v~humanitních oborech bývá zvykem, pojem není
ostře a bezesporně ohraničen. Teologie je navíc obor, který se zabývá věcmi
člověka přesahujícími, ba tento svět přesahujícími. O co více tedy je obtížné
definovat pojmy, jako to umí např. matematika? Přesná definice je možná,
pohybujeme-li se v~rámci dobře definovaného modelu. Hůře je možná, pohybujeme-li
se v~rámci popisu reálného světa. A ještě hůře je možná, jestliže hranice světa
překračujeme.

S~rostoucí obtížností podat definici ale roste i relevance a obsažnost pojmu.
Např. v~hypotetickém modelu ve formě počítačové hry mohu svátost zcela přesně
definovat jako nějaký úkon nebo předmět, který postavě dodá třeba nějaký bonus
nebo zvláštní schopnost. Definice může být precizní, ale týká se jenom hry a
souvislost se životem bude nulová. Navíc se tam pojem může zcela vyprázdnit.
V~rámci světa mohu svátost definovat vágně a napadnutelně např. jako viditelné
znamení Božího působení. Otevírají se zde dveře ke hnidopišství a otázkám, co
tedy ještě za svátost považovat a co nikoliv. Mohu vymýšlet různá viditelná
znamení, která by se za svátost považovala stěží, a definici tak nabourat.
Nicméně při dobré vůli rozumět se může takto definovaným pojmem předat něco
z~jeho vnitřního smyslu. Může pomoci k~pochopení a sladění myslí toho, kdo
definici tvoří či sděluje a toho, komu je předložena. Na třetím stupni, za
hranicemi světa, u Boha má svátost význam sama o sobě a její pochopení a prožití
je zázrak, který nejde mechanicky zprostředkovat. Odtud pojem pramení, jeho
prožití mu dává vzniknout v~lidském duchu a následně v~jazyce. Definovat ho tak,
aby došlo k~pochopení na této úrovni, je možné jen na základě skutečného
propojení dvou duší a užití slov či jiných prostředků přizpůsobených konkrétnímu
jedinci v~konkrétní situaci. Zde se pak můžeme bavit o pojmu zasvěcení spíše než
o běžném sdělování.

Jde-li nám tedy o to, ujasnit si, co míníme svátostí, nemůžeme toho na těchto
stránkách vyčerpávajícím způsobem dosáhnout. V~rámci modelu je to snadné: Pro
Církev československou husitskou je svátostí sedm, a to
\begin{enumerate}
\item{křest,}
\item{biřmování,}
\item{večeře Páně,}
\item{kněžské svěcení,}
\item{zpověď,}
\item{manželství a}
\item{útěcha nemocných.}
\end{enumerate}
Každá z~nich má pak samostatnou definici, a to opět na každé ze jmenovaných
úrovní.

V~rámci tohoto světa pojem svátosti definovali mnozí. K~nejčastěji zmiňovaným
patří sv. Augustýn: \textit{,,Sacramentum est sacrum signum,``} (De Civitate Dei
X, 5) čili ,,svátost je svaté znamení``. Známější verze
\textit{,,sacramentum est invisibilis gratiae visibilis forma,``} čili ,,svátost
je viditelná forma neviditelné milosti,`` je patrně Augustýnovi připisována
neprávem, byť harmonuje s~duchem jeho nauky \citep{king1967origin}.

Dalším velikánem je Tomáš Akvinský, který praví: \textit{,,Sacramentum potest
ali\-quid dici vel quia in se habet aliquam sanctitatem occultam, [...] vel quia
habet aliquem ordinem ad hanc sanctitatem,``} (Summa Theologiae III Q60) čili
,,Svátostí může být zváno něco, co v~sobě má nějakou skrytou svatost nebo co má
nějaký řád (rozuměj vztah) k~takové svatosti.``

Z~moderních autorů zmiňme Karla Bartha, který se svým způsobem drží Augustinova
pojetí: ,,Svátost je doprovodným a potvrzujícím symbolickým úkonem.``
\citep[I, 1]{barth1946kirchliche} Doprovodným se míní ke zvěstování, z~čehož pro
mne plyne pojetí svátosti jako čehosi principielně menšího než zvěstování.

K~doteku třetí, božské úrovně nám může dopomoci samotný dokument, kterým se
zaobírá tato práce. V~závěru úvodu stojí: \textit{,,Kéž tento naukový dokument
v~milosti a světle Ducha svatého je pomocí ke katechezi, vzdělávání ve víře,
zvěstování církve a jejímu liturgickému slavení a svátostné službě.``} Zde totiž
jazyk umožňuje dvojí čtení: svátostnou službu jednak jako vysluhování svátostí
definovaných na úrovni modelu, čili jedné ze sedmi, a jednak jako službu, která
je svým charakterem svátostná. Toto druhé, patrně nezamýšlené čtení může
evokovat libovolnou činnost, která je prováděna ve skutečné službě Duchu
svatému, a to ji činí svátostnou. Můžeme tak aplikovat Tomášův pojem svátosti
jako něčeho, co v~sobě skrytě obsahuje svatost. V~tomto duchu nechť je svátost
chápána a nechť je takto ve svátost proměněna veškerá naše činnost.

\vspace{15mm}
\chapter{Svátost v~Základech víry}
\label{div:zaklady}

\section{Mimo stať věnovanou svátostem}

Krom výše uvedeného užití obratu \textit{,,svátostná služba``} nacházíme zmínku
o svátosti v~citované pasáži \textbf{Čtyř pražských artikulů}. Ty jsou uvedeny jako jeden
z~pramenů našich Základů víry. Hovoří se zde o \textit{,,velebné svátosti těla a
krve Pána našeho Ježíše Krista,``}. Samo použití pozitivního elativu
\textit{velebný} má dalekosáhlé důsledky. Nastavuje to osobní a zúčastněný
postoj vůči svátosti, v~tomto případě večeře Páně. Je to svým způsobem vyznání a
bez něho se nelze svátosti přiblížit jinak než jako sterilnímu pojmu. Je proto
nanejvýš šťastné, že se Základy na tento starý text odvolávají jako na svůj
pramen.

Další užití slova svátost nacházíme v~odpovědi na \textbf{otázku 13}: \textit{,,v čem je
Boží církev po celém světě jednotná?{}``}, a sice že \textit{,,užívá svátosti křtu
a večeře Páně.``} Zdálo by se, že se vlastně zde o svátostech nic nevypovídá,
nýbrž že se vypovídá jen o církvi. Opak je pravdou: Svátosti křtu a večeře Páně
jsou zde postaveny do spojujícího prvku všech světových křesťanských denominací.
Dává jim to další rozměr posvátnosti: Krom inherentního významu tím, co jsou,
přibírají rozměr definujícího prvku celosvětového křesťanstva. ,,Jsem-li účasten
těchto dvou svátostí, patřím do rodiny křeťanů z~celé Země.``

V~odpovědi na \textbf{otázku 27} \textit{,,Kdo je hlavou Boží církve?{}``} se dočítáme, že
\textit{,,{},Hlavou` Boží církve je Kristus, který ji vede, oživuje a řídí svým
Duchem skrze svědectví Písma svatého a svátosti jako své tělo.``} Především je
zde třeba větu syntakticky zjednoznačnit. Nabízí se totiž čtení, že se hovoří o
\textit{svátostech jako (např.) Kristovo tělo}. To je čtení mylné a větu je
třeba číst tak, že \textit{Kristus církev řídí jako své tělo}, a to
\textit{skrze svědectví Písma svatého a svátosti}. Plyne to z~protikladu hlavy a
těla, kterého se zde užívá jako metafory.

Výpověď, že Kristus vede, oživuje a řídí církev skrze svátosti, je do jisté míry
problematická. Církev je vedena Kristem tehdy, když v~ní působí Duch svatý, to
považuji za natolik zřejmé, že pro to snad netřeba dále argumentovat. Výrok
v~odpovědi na otázku 27 by tedy implikoval, že působení Ducha svatého je
podmíněno vykonáváním svátostí. Toto připomíná mechanické středověké uvažování,
jako např. že kdo neprodělal církevní obřad křtu, nemůže vstoupit do nebe.

Je samozřejmě možné interpretovat výrok i tak, že vykonáváním a přijímáním
svátostí s~upřímností a pokorou se disponujeme přijetí Ducha svatého a ten nás
pak může vést, oživovat a řídit. Také lze výrok interpretovat tak, že se zde
svátost myslí obecnější, v~duchu Tomáše Akvinského jako to, co skrytě obsahuje
svatost. Potom k~tomu můžeme zajisté zařadit i přijetí Ducha svatého a říci, že
skrze takovou svátost Kristus církev řídí.

Vzhledem k~tomu, že však tento rozbor směřuje zevnitř církve v~jejím rámci k~ní
samotné, nepokládám za vhodné uplatňovat shovívavý postoj. Naopak, domnívám se,
že je vhodné poukázat na potenciální nedostatek -- ať už ve formulaci Základů
víry nebo, což je pravděpodobnější, v~mém vlastním jejich pochopení.

Navrhoval bych odpověď upravit například do této podoby: \textit{,,,Hlavou` Boží
církve je Kristus, který ji vede, oživuje a řídí svým Duchem, jehož nalézáme
v~Písmu svatém a o něhož se ucházíme přijímáním svátostí.``}

Odpověď na \textbf{otázku 93} zní: \textit{,,Zvěst církve je nutno doplňovat a potvrzovat
svátostmi a svědectvím života.``} Domnívám se, že v~tomto zdánlivě banálním
výroku se skrývá velká moudrost. Opět by sice bylo možné vidět zde mechanické
uvažování a upevňování církevní moci diktátem povinnosti přijímat svátosti. To
by však v~tomto případě ukazovalo spíše na takové uvažování na straně čtenáře,
než na chybu ve formulaci.

Nutnost svátostí je formulována i u Akvinského, opět v~Sumě III/60, ovšem to
neznamená, že by byla nějak samozřejmou. Konstatovat nutnost svátostí je zcela
na místě, a to proto, že jimi člověk rozšiřuje svůj život o posvátný rozměr.
Křtem zasvěcuje celý život Bohu. Bez křtu může žít pro sebe, jako pokřtěný už má
žít Bohu. Bez manželství může mít partnera pro svoje potěšení nebo výhody.
Manželstvím před Bohem se vztah stává prostředkem vztahu s~Bohem. Pokáním
opouští svoji hříšnost a zavazuje se zůstat čist. Přijímáním těla a krve Páně se
sjednocuje s~Bohem. Pochopitelně prázdné obřady nic z~toho nezprostředkují --
tolik se za sebe odvažuji tvrdit, ať je to v~rozporu s~jakoukoliv naukou. Ale
přijímá-li člověk svátosti opravdově, všechno toto se děje. A bez toho se to
děje stěží, protože naše životy nás táhnou k~žití pro sebe. Z~toho důvodu plně
souhlasím, že svátostmi a svědectvím života je nutno zvěst církve doplňovat.

\section{Definice svátosti podle Základů viry}

Konečně se dostáváme k~vlastní definici pojmu svátosti podle Základů víry. Podle
odpovědi na \textbf{otázku 307}
\textit{,,Svátost je jednání, jímž se obecenství věřících v~Duchu svatém činně
podílí na milosti Božího Slova přítomného skrze slyšené svědectví Písma svatého
a svátostné úkony.``} Tato formulace si zaslouží podrobnější rozbor.

\textit{,,Svátost je jednání.``} A sice jednání obecenství věřících. Na jednu
stranu je tato formulace kategoricky konkrétnější než u Augustina a Tomáše.
Nejde už o zcela neurčité \textit{znamení (,,signum``)} nebo \textit{něco
(,,aliquid``)}. Jde o jednání věřících. Předměty, jednání Boží, stavy mysli,
místa, slova, dokonce slova evangelia, vše je z~pojmu vyňato. Nabízí se otázka,
jak autoři formulace k~definici přistupovali: Zda jim šlo o definici
vyčerpávající nebo spíše deskriptivní. Jinými slovy zda jim šlo o to, vymezit,
co všechno lze nazvat svátostí a co už nikoliv, nebo zda jim šlo o to, úkony,
které zveme svátostmi co nejlépe charakterizovat. Vzhledem k~tomu, že se jedná o
odpověď na otázku \textit{,,Co je svátost?``}, logicky bych čekal definici
vyčerpávající, tedy formulaci, která vymezí, co svátost je a co svátost není.
Ze způsobu odpovědi však usuzuji, že opdověď spíše charakterizuje ty úkony,
které zveme svátostmi. Každopádně potom je odpověď mnohem legitimnější než
v~případě vyčerpávající definice.

Pro demonstraci toho, že jako vyčerpávající definice formulace ze Základů víry
nepostačuje, si vezměme příklad mezního křtu: Dva nevěřící lidé Alice a
Bořek\footnote{Podle vzoru scénářů bezpečnostních rizik pojmenovávám postavy
jmény podle abecedy. V~angličtině se standardně používá Alice a Bob, viz
\url{https://en.wikipedia.org/wiki/Alice\_and\_Bob}.} jsou spolu sami a Alice je
v~ohrožení života. Alice si vzpomene na to, že babička kladla důraz na
náboženství a zatouží po křtu. Bořek křtu nepřikládá žádný význam, ale ze sympatie
Alici polije vodou a pronese křestní formuli. Alici se přitíží a za několik
minut se dostane na pokraj smrti. Tehdy se jí dostane mystická zkušenost a stane
se z~ní hluboce věřící člověk.

Jedná se zde o svátost nebo nikoliv? Podle zdravého úsudku jistě ano: Křest byl
církevně přípustný a vyslyšený Bohem. Podle definice ale nikoliv, protože nešlo
o jednání obecenství věřících: Během křtu byli oba dva nevěřící.
Přikloňme se tedy k~variantě, že formulace má popisovat to, co v~církvi zveme
svátostí. Vhodnější otázkou by tedy mohlo být: \textit{,,Čím jsou svátosti?{}``}

\textit{,,... podílí na milosti Božího Slova...``} Co znamená podílet se na
milosti Božího slova? Je zde míněno Boží slovo ve smyslu biblického textu nebo
ve smyslu otázky 52 jako Bůh sám? Pokud se míní Bůh sám, proč se užívá pojmu
Boží slovo? Co se míní milostí Božího slova? Je to milost, kterou udílí Boží
slovo? Nebo je to milost, ze které nám Boží slovo je dáno? V~jakém smyslu se
míní podílení? Jde o to, že my spoluudílíme milost s~Bohem? Nebo že jsme jedni
z~těch, kterým je milost udělena (máme na ní podíl)?

\textit{,,...v~Duchu svatém činně podílí...``} Podílet se v~Duchu svatém a
podílet se činně jsou dvě příslovečná určení. Mají deskriptivní nebo
restriktivní význam? To jest: Vypovídá se zde, že podílení je v~Duchu svatém a
je činné, nebo to znamená, že jiného podílení než činného a v~Duchu svatém se
tato věta netýká?

Abych rozdíl ilustroval, vezměme si příklad věty: \textit{,,Předej prosím tohle
svojí spanilé choti a tohle svojí starší dceři.``} Ve větě se vyskytují dvě
vazby podstatného jména s~přívlastkem, a to \textit{,,spanilé choti``} a
\textit{,,starší dceři``}. Je nám jasné, že oslovený má jedinou choť, a o té se
přívlastkem navíc dodává, že je spanilá. Naopak je jasné, že dcery má dotyčný
dvě a předmět je jen pro tu starší z~nich. Z~jazykového hlediska to může být
klidně i naopak: dotyčný může mít několik manželek a předmět je jen pro tu
spanilou z~nich. Stejně tak může mít jedinou dceru o které má mluvči potřebu
dodat, že už je poněkud starší. Oba tyto alternativní výklady budí pousmání,
protože znalost světa nám funkci přívlastku jasně rozliší: Zda je restriktivním,
čili takovým, který vymezuje, který z~myšlených předmětů má mluvčí na mysli,
nebo deskriptivní, tedy takový, který podstatné jméno rozvíjí.

V~našem případě vazby \textit{,,v~Duchu svatém [se] činně podílí``} se nejedná o
přívlastky nýbrž o příslovečná určení, ale rozdíl je přesto přítomen a je
potřeba si ujasnit, o který z~případů se jedná. V~případě deskriptivního rozvití
by se zde říkalo, že svátost je jednání, jímž se obecenství věřících
podílí na milosti Božího Slova, a o takovém podílení se praví, že je činné a že
je v~Duchu svatém: ,,Kdo se podílí na milosti Božího slova, ten tak činí v~Duchu
svatém.`` V~přínadě restriktivního rozvití by se zde říkalo: ,,Jestliže se někdo
na milosti Božího slova podílí jinak než činně a v~Duchu svatém, nejedná se o
svátost.``
Každé rozvití (\textit{,,činně``} a \textit{,,v~Duchu svatém``} také může mít
jinou funkci, jedno restriktivní a jedno deskriptivní.

Domnívám se, že u činného podílení se jde o přislovečné určení restriktivní.
Připouští se tedy, že se člověk může podílet i pasivně, nebo přesněji řečeno
jinak jež činně, a na takové podílení se definice svátosti pak nevztahuje. U
,,v~Duchu svatém``  toto dle mého názoru vůbec jasné není. Naopak, zda je možné,
aby se člověk činně podílel na milosti Božího slova jinak než v~Duchu svatém je
dosti závažná teologická otázka. A má významný dopad na tuto definici svátosti:
Je účast v~Duchu svatém podmínkou? Je-li podmínkou, pak jak se ověří? Jestliže
by např. křest neproběhl v~Duchu svatém, bude platný? Jak se to ověří? Co
vlastně znamená, že ,,obecenství se podílí v~Duchu svatém``? Stačí, aby se
jediný účastník podílel nehodně a je tím svátost zrušena? Stačí, aby se jediný
účastnil platně a je tím svátost potvrzena? Je potřeba většiny? Jestliže se
jedná do deskriptivní rozvití, nebylo by lépe ho pro jasnost a srozumitelnost
vynechat?

Přiznávám, že se zde dopouštím legalistického uvažování, ale zároveň se
domnívám, že formulace, která nemá jasný význam, by se  v~oficiálním
věroučném dokumentu neměla vyskytovat.

Kartézským součinem jen zde explicitně nastíněných variant bychom se dostali ke
dvaatřiceti různým čtením. Všechny je rozebrat je holý nesmysl. Pokusme se proto o
intuitivní zjednoznačnění -- jak byla formulace asi myšlena: Při čtení a kázání
se předává svědectví Písma svatého. Také se provádějí svátostné úkony. V~těchto
dvou formách je přítomné Boží slovo. Božím slovem se míní Bůh, který se
v~biblickém textu zjevuje. Slovo Boží je jednak předmětem milosti, tedy
z~milosti Bohem udělené je nám Boží slovo dáno, jednak původcem milosti, tedy
Boží slovo coby Bůh udílí milost. Víceznačnost ponecháváme a rozumíme oběma
způsoby. Obecenství věřících slyší čtení, kázání a účastní se svátostných úkonů.
Z~milosti Boží biblický text a svátostné úkony rozněcují Ducha svatého v~srdcích
účastníků. Tak se činně a v~Duchu svatém  na této  milosti podílejí. Svátosti
jsou takovýmto jednáním.

Co se tato definice vlastně snaží předat? Jaké je její poselství? Proč asi byla
zvolena právě takto? Kéž bych na tyto otázky dokázal uspokojivě odpovědět! Můj
osobní dojem je, že zde jde především o to, podat takovou formulaci, která by
jednotně pokryla všechny svátosti z~výčtu těch uznávaných, která by položila
důraz na působení Boží v~obecenství, a která by svátosti položila vedle četby a
kázání, aby tyto měly nemenší důležitost.

\section{Poměr svátosti a kázání}

Tomu, že definice svátosti podle Základů víry byla mimo jiné motivovaná tím,
vymezit poměr svátosti vůči kázání, nasvědčuje i fakt, že tomuto poměru se
věnují všechny další otázky o svátosti, než se přejde ke svátostem jednotlivým.
Zmiňuje se tu například, že narozdíl od kázání se na svátostech podílíme více
smysly. Například u svátosti smíření myšlenka na účast více smysly vnucuje
asociace na obligátní vtipy o necudnosti kněží, který ovšem díky Bohu a díky
dobrovolnosti celibátu u nás nemá obzvláštní opodstatnění.

Celkově mi ale důraz na mnohosmyslovost trochu uniká: Nejde přece o smyslový
prožitek, ale o vnitřním procesu, který jím má být doprovázen. Fakticky to však
realitě odpovídá, a třeba je zde důležitost, která mi uniká. Podobně skepticky
se musím postavit i k~další odpovědi, a sice že obsahem svátostí je slovo Boží.
Vynořuje se zde další řada otázek po konkrétním smyslu tohoto výroku v~kontextu
jednotlivých svátostí. Nechci ale opět zabředávat do analýzy, proto budiž řečeno
jen tolik, že při definici slova Božího jako Boha samého, lze tuto formulaci
obhájit, ale její konkrétní výpovědní hodnota trpí.

\vspace{15mm}
\chapter{Závěr}
\label{div:zaver}

