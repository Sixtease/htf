%%% Hlavní soubor. Zde se definují základní parametry a odkazuje se na ostatní části. %%%

%% Verze pro jednostranný tisk:
% Okraje: levý 40mm, pravý 25mm, horní a dolní 25mm
% (ale pozor, LaTeX si sám přidává 1in)
\documentclass[12pt,a4paper]{report}
\setlength\textwidth{145mm}
\setlength\textheight{247mm}
\setlength\oddsidemargin{15mm}
\setlength\evensidemargin{15mm}
\setlength\topmargin{0mm}
\setlength\headsep{0mm}
\setlength\headheight{0mm}
% \openright zařídí, aby následující text začínal na pravé straně knihy

%% Pokud tiskneme oboustranně:
% \documentclass[12pt,a4paper,twoside,openright]{report}
% \setlength\textwidth{145mm}
% \setlength\textheight{247mm}
% \setlength\oddsidemargin{14.2mm}
% \setlength\evensidemargin{0mm}
% \setlength\topmargin{0mm}
% \setlength\headsep{0mm}
% \setlength\headheight{0mm}
% \let\openright=\cleardoublepage

%% Vytváříme PDF/A-2u
\usepackage[a-2u]{pdfx}

%% Přepneme na českou sazbu a fonty Latin Modern
\usepackage[czech]{babel}
\usepackage{lmodern}
\usepackage[T1]{fontenc}
\usepackage{textcomp}

%% Použité kódování znaků: obvykle latin2, cp1250 nebo utf8:
\usepackage[utf8]{inputenc}

%%% Další užitečné balíčky (jsou součástí běžných distribucí LaTeXu)
% Abstrakt (doporučený rozsah cca 80-200 slov; nejedná se o zadání práce)


%% Balíček hyperref, kterým jdou vyrábět klikací odkazy v PDF,
%% ale hlavně ho používáme k uložení metadat do PDF (včetně obsahu).
%% Většinu nastavítek přednastaví balíček pdfx.
\hypersetup{unicode}
\hypersetup{breaklinks=true}

%% Definice různých užitečných maker (viz popis uvnitř souboru)

\begin{document}

\chapter*{Iterativní zdokonalování přepisu zvukových nahrávek s~využitím zpětné vazby posluchačů}

\section*{Abstrakt}

\thispagestyle{empty}

Tato disertační práce se zabývá zpřístupněním zvukových
záznamů jednoho mluvčího úzké i široké veřejnosti.

Motivací práce byla existence chátrajících nahrávek hovorů českého filozofa
ing. Karla Makoně na kazetách a kotoučích. Cílem je zachování materiálu pro
budoucí generace a zpřístupnění nahrávek pomocí digitálních technologií,
především přístupnosti nahrávek na internetu a možnosti vyhledávání v~nich.

Práce představuje tvorbu systému pro přepis velké sady zvukových záznamů
se zapojením laické komunity. Navržené řešení spočívá ve vytvoření základního
přepisu nízké kvality pomocí automatického rozpoznávání řeči a vyvinutí
aplikace, která umožní od členů komunity i nahodilých zájemců získávat opravy
automatického přepisu, použitelné jako trénovací data pro další zlepšování.

Popíše se samotný mluvený korpus. Představí se autor a
jeho dílo,
témata v~nahrávkách, nahrávání samotné, digitalizace a získané přepisy.
Dále se rozvede tvorba systému pro automatický
přepis korpusu od sběru dat, přes akustické a jazykové modelování, různé
provedené experimenty až k~vyhodnocení úspěšnosti. V~neposlední řadě se popíše
webová aplikace pro sběr manuálních přepisů. Zmíní se odlišnosti od ostatních
systémů, detaily návrhu a řešení, mechanismus pro kompenzaci vysokých nároků na kvalitu
přepisu a nízkých nároků na odbornost přispěvatelů a vyhodnocení funkčnosti
po osmi letech provozu.

\end{document}
