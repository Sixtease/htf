\chapter{Úvod}
\label{kap:uvod}

Provádět páry přechodem do manželského svazku je jedno ze základních poslání
církve. O svazku muže a ženy promlouvá Starý zákon (Gn~1,27; Gn~2,23-24;
Ex~21,7-11; Lv~18,18; Dt~21,10-17; Dt~24,5; Dt~25,5-10), Nový zákon (Mk~10,6-9;
Mt~19,3-11; J~2,1-2), křesťanská tradice: Již Sentence Petra Lombardského, který
jako první vyjmenovává sedm svátostí, pojednávají o manželství jako o svátosti.
I nezávislý moderní výzkum rozpoznává péči o partnerské vztahy jako jednu
z~klíčových rolí náboženského života, viz např. Seybold \& Hill
2001\cite{seybold2001role}.

Do jaké míry však církevní instituce svoji roli plní? Je církevní sňatek
skutečnou pomocí pro realizování svazku, ve kterém už nejsou dva, ale jedno
tělo? Svazku, ve kterém se sejdou dva ve jménu Jeho a On je prostřed nich?
V~akademické oblasti se tématu věnuje např. Mahoney et al.
1999\cite{mahoney1999marriage}.  Přesto na tyto otázky neexistuje jednoznačná,
globálně platná odpověď. 

Článek, který tato seminární práce recenzuje, se tématu věnuje z~praktického,
sekulárního úhlu pohledu: Jak efektivní mohou být kněží v~předávání zavedeného
předmanželského tréninku? Tento pohled zajisté nepokrývá problematiku komplexně,
nicméně to nezabraňuje relevanci výzkumu pro duchovního.

\vspace{15mm}
\chapter{Obsah článku}

Autoři představují výzkum v~oblasti předmanželského tréninku pro zvýšení šancí
na úspěšnost manželství\cite{stanley2001community}. Co definuje úspěšné
manželství, autoři nedefinují.  Vycházejí z~existujícího programu pro prevenci a
zlepšování vztahů (prevention and relationship enhancement program, PREP, patrně
se též jedná o aluzi na slovo ,,preparation``). Tento program se měl podle
citovaného článku (Markman \& Stanley \& Blumberg
1994\cite{markman1994fighting}) osvědčit v~klinických podmínkách.  Předmětem
tohoto článku je úspěšnost aplikace zmíněného programu duchovními v~jejich
náboženských obcích.

Ve svém domovském Denveru autoři rekrutovali 105 náboženských obcí různých
vyznání, z~nichž se 45 účastnilo experimentu až do konce. Každá z~těchto obcí
opět rekrutovala zasnoubené páry pro účast v~experimentu. Účast byla
pochopitelně dobrovolná, jak ze strany náboženských obcí, tak ze strany párů%,
%proto nelze vyloučit, že výsledky jsou nereprezentativní i pro Denverskou oblast
%v~roce 1994, neboť ochota k~účasti může být faktor korelující s~výsledky
. Každý
pár byl náhodně zařazen do jedné ze tří skupin:
\begin{enumerate}
  \item{ponechané bez zásahu,}
  \item{
    provedené programem pracovníky Denverské univerzity, která experiment
    prováděla a
  }
  \item{
    provedené programem duchovními, kteří byli zaškoleni pracovníky Denverské
    univerzity.
  }
\end{enumerate}

Prezentují se tři hypotézy a jejich experimentální ověření:
\begin{enumerate}
  \item{
    hypotéza: Úspěšnost párů se zvýší, když duchovní použijí PREP. Jinými slovy
    se předpokládá zlepšení ve 3. skupině (provedené duchovními) oproti 1.
    skupině (bez zásahu).
  }
  \item{
    hypotéza: Pracovníci univerzity budou efektivnější než zaškolení duchovní.
    Jinými slovy se hypotetizují lepší výsledky ve 2. skupině oproti 3. skupině.
  }
  \item{
    hypotéza: Jednotlivé skupiny se od sebe nebudou výrazně lišit v~jiných
    metrikách\footnote{%
      Pojmem metrika zde míním kvantitativní reflexi některého zkoumaného jevu.
      Např. je-li v~dotazníku výzva, aby respondent na stupnici od 1 do 10 určil
      svoji spokojenost ve vztahu, pak uvedené číslo je metrikou, která modeluje
      (do určité míry spolehlivě či nespolehlivě) spokojenost ve vztahu.
    }, než na které se trénink soustředí.
  }
\end{enumerate}

Použité metriky jsou četné a podložené výzkumem, ovšem ze značné části vlastním
výzkumem: Commitment Inventory (Stanley \& Markman
1992\cite{stanley1992assessing}), Confidence Scale (do té doby nepublikováno),
Relationship Dynamics Scale (Stanley \& Markman 1997\cite{stanley1997marriage}),
Communication Skills Test (Jenkins \& Saiz, DU, nepublikováno), Marital Agendas
Protocol (Notarius \& Vancetti 1983\cite{notarius1983marital}), Marital
Adjustment Test (Locke \& Wallace 1959\cite{locke1959short}), Interaction
Dynamics Coding System (Julien \& Markman \& Lindahl
1989\cite{julien1989comparison}). Kromě toho se sbíraly demografické údaje,
subjektivní hodnocení vlastní náboženskosti a spokojenost s~poskytnutým
tréninkem. Bohužel se neuvádějí výsledky všech metrik.

Postup je ten, že se vše vyhodnotí před zaškolením a po něm. Zaškolení trvá 12
hodin čistého času a je rozděleno do tří sezení s týdenními rozestupy. Není
uvedeno, jak dlouho po zaškolení se provádí aposteriorní vyhodnocení. 1. skupina
(bez zásahu) absolvovala různé stupně předmanželské přípravy podle zvyklostí
v~té které náboženské obci. Minimální přípravou bylo organizační dojednání
svatby, maximální kolem sedmi hodin obnostního vyhodnocení a zaškolení.

Výsledky jsou následující:
{\bf První hypotéza} se potvrdila. Vyhodnocovala se na metrice Interaction Dynamics
Coding System a zaznamenala F-score 11,89. To znamená, že variance mezi 1. a 3.
skupinou byla téměř dvanáctkrát větší než variance v~rámci skupin. Konkrétní
hodnoty na stupnici IDCS však nedokážu přesně vyčíst. Podle přiloženého obrázku se zdá, že
se u 3. skupiny v~kategorii pozitivní komunikace zvedla přibližně z~3,9 na 4,4,
zatímco u 1. skupiny \textbf{poklesla} přibližně ze 4,2 na 3,7. V~textu se však uvádí,
že ke statisticky významnému posunu došlo pouze u 3. skupiny (z~1. a 3.) a
konkrétní hodnoty nenalézám. Je tedy možné, že pokles byl statisticky
nevýznamný, což však vyvolává otázku po konkrétních datech, neboť diference u 1.
skupiny se nezdá řádově nižší.
Kromě IDCS se hodnotilo metodou Communication Skills Test. Rozdíl v~apriorních a
aposteriorních hodnotách autoři uvádějí pouze na ose Problem-Solving Skills.
Uvádí se F-score 7,99 se zvýšením hodnoty u třetí skupiny.
{\bf Druhá hypotéza} se nepotvrdila. U hodnot pode IDCS si vedli duchovní o něco lépe
než zaměstnanci Denverské univerzity. Zajímavým a asi nečekaným výsledkem však
bylo, že se zvýšila míra invalidace ze strany partnera mezi hodnotami před a po
školení, a to zejména vůči mužům. Autoři to připisují zátěži předsvatebními
přípravami a problematickými tématy, která byla vynesena účastí na programu.
{\bf Třetí hypotéza} se uvádí jako potvrzená nezjištěním rozdílů, které by ji
vyvracely.

Kromě potvrzení či vyvrácení hypotéz uvádějí autoři další poznatky: Páry v~druhé
a třetí skupině uváděly větší spokojenost s~poskytnutou předmanželskou přípravou
než ty v~1. skupině. Program tedy doznal pozitivního hodnocení od účastníků.
Účastníci svorně uváděli jako nejnápomocnější část výcviku komunikační
dovednosti, zejména techniku speaker/listener\footnote{Zdá se, že tato metoda
pochází taktéž od autorů Markman \& Stanley, ovšem původní publikaci jsem
nedokázal dohledat.}. Metrika Dedication Commitment se
zvýšila signifikantně více u žen než u mužů. Religiozita měřená subjektivním
sebehodnocením, jakož i věk, byly o něco nižší u druhé skupiny. Autoři to
připisují domněnce, že výcviku prováděném sekulární organizací se ochotněji
podrobí mladší a méně religiozní jedinci.

\vspace{15mm}
\chapter{Diskuze}

Aby bylo možné článek a jeho přínos reálně ohodnotit, je nutné se seznámit
minimálně s~přesným průběhem programu PREP a s~hodnoticí metrikou IDCS. Bohužel
citované zdroje ani k~jednomu nejsou k~dispozici. Samotný PREP je však
publikován v~článku Renick et al. 1992\cite{renick1992prevention}, který
dostupný je. Nelze tedy zhodnotit primární metriku, ale program samotný ano.

PREP se vymezuje jako program primární prevence, tedy cílící na páry, u nichž
ještě nedochází ke vztahovým poruchám. Spočívá ve dvanácti přednáškách spojených
s~párovými cvičeními. Témata přednášek jsou komunikační dovednosti, sebereflexe,
radost ve vztahu, schemata nedorozumění mezi pohlavími, upevňování vztahu,
duchovní rozměr vztahu a schopnost použití přednesené látky v~krizových
situacích. Cvičení jsou pod dohledem nebo samostatná, podle zvoleného formátu.

Uvádějí se významné pozitivní výsledky: zvýšení spokojenosti v~partnerství první
tři roky po absolvování; zpomalení poklesu spokojenosti čtyři a více let po
absolvování; 58\% pokles rozvodovosti pět let po absolvování. Slibnost programu
se tedy nedá popřít.

Recenzovanému článku se dají najít i výtky:

\begin{enumerate}
  \item{
    Hned první odstavec diskutuje obecně vnímanou potřebu ,,něco dělat``
    s~neuspokojivým stavem. Autoři se řečnicky táží: ,,Budou však tyto výzvy
    ,něco dělat` mít odezvu v~moudře vedených zásazích založených na empirických
    datech?{}`` Nemohu se ubránit protiotázce: Jsou empirická data nejlepším
    vodítkem? Není v~osobním vedení lidí za dlouhodobým, niterným a
    celobytostným cílem, lepším vodítkem lidská moudrost jednotlivce, v~němž
    jsou empirická data integrována a transformována do nadhledu a prozření? Je
    kvantitativně vyhodnotitelný přístup tím nejlepším, co jako lidstvo máme
    k~dispozici? Zajisté je to tím nejlepším z~hlediska kvantitativního
    vyhodnocení, už z~povahy věci. Nicméně považoval bych za nešťastné, kdyby se
    přístupy založené na jiných principech, považovaly za méně hodnotné, byť
    jejich účinnost nebude přímočaře reprodukovatelná jinými.
  }

  \item{
    V~článku se operuje s~duchovními jako vedoucími předmanželské intervence.
    Ovšem nijak se nezkoumá kompetence těchto duchovních k~danému úkolu. Jediným
    faktorem, který by se dal za tuto míru kompetence považovat, je délka
    vzdělání a délka praxe v~práci s~lidmi. Je však zřejmé, že desítkami let
    špatné praxe se může špatná praxe prohloubit a zhoršit. Vyhodnocení apriorní
    kompetence kněží bych považoval za stěžejní součást výzkumu, zkoumám-li vliv
    zavedení konkrétní metodiky na jejich efektivitu. S~tím souvisí absence
    podrobnějšího popisu ,,běžné praxe`` v~náboženských obcích, byť formou odkazu
    na předešlý výzkum. Jen v~citovaném článku\cite{renick1992prevention} se uvádí jako
    běžná praxe v~katolické církvi ,,Engaged Encounter``. Bližší prozkoumání
    tohoto programu považuji za nutnou budoucí práci v~návaznosti na tuto
    recenzi. Jedině pokud víme, s~čím přesně se navrhovaná metodika
    porovnává, můžeme výsledky přenést do dalších prostředí. Bílí dvacátníci
    z~Denveru devadesátých let nejsou ani při nejlepší vůli reprezentativním
    vzorkem globální populace snoubenců. Je zcela pochopitelné, že se autoři
    omezují na lokální výzkum, ale užitečnosti zajisté nepřidává, že kontext je
    málo popsán. Dá se to ovšem jistě přisoudit i omezením počtu stran publikace
    vydavatelem.
  }

  \item{
    Cituji ze 4. odstavce na 1. stránce:
    \begin{quote}
      Dělicí bod na cestě k~selhání manželství nastává, když jeden z~partnerů
      či oba začnou s přítomností protějšku spolehlivě asociovat bolest a stres
      místo podpory a bezpečí. Redukce negativních interakcí a zachovávání
      elementů pozitivního svazku jsou proto stěžejními cíli prevence.
    \end{quote}
    Tento závěr nelze zpochybňovat, ale opět považuji za nešťastné zůstávat na
    povrchu věci u pozorování a adresování behaviorálních jevů. Přístup se
    nápadně podobá naivnímu lékařství, kde se na akné aplikuje vysoušecí mast,
    místo aby se zjistila příčina např. v~disbalanci některého vnitřního orgánu.
    Je potřeba dávat si pozor, aby z~takzvaně vědeckého, kvantitativního
    přístupu, se nestala jakási modla.
  }

  \item{
    Práce budí dojem jakési uzavřenosti do vlastního vesmíru. Většina
    citovaných materiálů je z~dílny autorů, vyhodnocuje se na metrikách taktéž
    z~vlastní dílny. Výsledky se vyhodnocují bezprostředně po prodělání výcviku.
    Autoři přiznávají, že by bylo potřeba dlouhodobější pozorování, s~čímž nelze
    než souhlasit.
  }
\end{enumerate}

Navzdory těmto nedostatkům se nedá popřít přínos práce a domnívám se, že skrývá
velký potenciál i pro moderní české duchovní. Vidím zejména tyto pozitivní
přínosy:

\begin{enumerate}
  \item{
    Považuji za inspirativní zahrnout do předmanželské přípravy oporu o
    kvantitativní metriky. Jakkoliv jsem proti tomu, aby se postupovalo výlučně
    podle nich, vyhýbat se jim jako nástroji by bylo stejně tak chybné. Metody
    přednesené v~předcházející práci jsou slibné, ovšem výzkum
    v~psychologii za 26 let od publikace jistě pokročil, takže obrátit se na
    moderní metody vyvinuté v~našem prostředí se jeví smysluplnější.
  }

  \item{
    Za velice povzbudivý považuji fakt, že kněží byli v~experimentu navzdory
    hypotéze autorů úspěšnější než zběhlí zaměstnanci univerzity. Kéž by se
    tento výsledek stal motivací duchovním, aby se systematicky zdokonalovali ve
    svém působení na snoubence a páry ve své pastýřské péči!
  }
\end{enumerate}

\vspace{15mm}
\chapter{Závěr}

Recenzovaný článek přináší zajímavý a inspirativní výzkum v~oblasti prevence
rozvodovosti se zapojením náboženských obcí jako startovního můstku pro 75\%
prvomanželství ve zkoumané denverské oblasti devadesátých let. Program navazuje
na dlouholetý výzkum a prokazuje značnou účinnost. Systém frontálních přednášek
může být dnes zastaralý, ale to není na posouzení v~této recenzi. Vhled do problematiky a do
konkrétních postupů, které se jeví jako poctivá snaha o správnou vědeckou
metodu, jsou cennými klenoty na cestě k~vlastní praxi duchovního, kde působení
na novomanžele je jedním z~bodů obrovské zodpovědnosti a zároveň příležitostí
k~tomu, aby kněžský úřad prokazoval čest svému nebeskému povolání.
